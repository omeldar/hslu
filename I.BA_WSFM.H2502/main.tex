% !TeX program = tectonic
\documentclass[%
  paper=a4,
  fontsize=11pt,
  BCOR=8mm,
  DIV=12,
  headings=normal,1
  parskip=half,
  numbers=noendperiod
]{scrreprt}

% Fonts & language: Tectonic uses (Xe)TeX engine; use fontspec + polyglossia
\usepackage{fontspec}
\usepackage{polyglossia}
\setmainlanguage{german}
\setotherlanguage{english}

% Choose readable fonts
\setmainfont{TeX Gyre Pagella}
\setsansfont{TeX Gyre Heros}
\setmonofont{TeX Gyre Cursor}

% Encoding & micro-typography
\usepackage{microtype}

% Graphics, tables, math (lightweight defaults)
\usepackage{graphicx}
\usepackage{booktabs}
\usepackage{siunitx}
\usepackage{amssymb}
\usepackage{amsmath}

\usepackage{pifont}
\sisetup{locale=DE, detect-all}

% Bibliography with biblatex + biber
\usepackage[
  backend=biber,
  style=apa,
  natbib=true,
  maxcitenames=2,
  maxbibnames=99
]{biblatex}

\DeclareLanguageMapping{ngerman}{ngerman-apa}
\addbibresource{bib/references.bib}

% Load hyperlink setup
\usepackage{xcolor}
\usepackage{hyperref}
\hypersetup{
    colorlinks=true,
    linkcolor=blue,
    citecolor=black,
    filecolor=magenta,      
    urlcolor=blue,
    pdftitle={Implementierung eines automatisierten Löschprozesses für kategorisierte Daten},
    pdfauthor={Eldar Omerovic},
    pdfsubject={Bachelorarbeit – Hochschule Luzern, Informatik},
    pdfkeywords={Datenlöschung, Audit Logging, Scheduling, .NET, DSGVO},
    pdfpagemode=FullScreen,
}
\urlstyle{same}

% Redefine the url command for the TOC to remove styling
\AtBeginDocument{
  \addtocontents{toc}{\protect\hypersetup{linkcolor=.,citecolor=black,urlcolor=.}}
}

% Load project preamble/macros
% Figure paths
\graphicspath{{figures/}}

% Klickbar: kompletter Eintrag (Text + Seitenzahl)
\hypersetup{linktoc=all}

% Hierarchie mit Einrückungen
\KOMAoptions{toc=graduated}

% Dichtes Punktmuster für die Leader
\makeatletter
\renewcommand*\TOCLineLeaderFill{\leaders\hbox to .6em{\hss.\hss}\hfill}
\makeatother

% No indentation, space between paragraphs
\setlength{\parindent}{0pt}
\setlength{\parskip}{\baselineskip}

\usepackage{csquotes}
\setmainlanguage{german}

\usepackage{listings}
\usepackage{xcolor}

\usepackage{upquote}
\usepackage{microtype}

\usepackage{float}

% --------------------
% Fonts & Typografie
% --------------------
\usepackage[T1]{fontenc}
\usepackage{lmodern}
\usepackage{microtype}

\usepackage{setspace}
\setstretch{1}

% Table that wraps over pages
\usepackage{longtable}
\usepackage{paracol}
\usepackage{array}

% table of contents

\usepackage{tocloft}
\renewcommand{\listfigurename}{Abbildungsverzeichnis}
\renewcommand{\listtablename}{Tabellenverzeichnis}
\renewcommand{\lstlistlistingname}{Quellcodeverzeichnis}

\makeatletter
\newcommand{\onlylof}{\@starttoc{lof}}
\newcommand{\onlylot}{\@starttoc{lot}}
\newcommand{\onlylol}{\@starttoc{lol}} % List of Listings (listings)
\makeatother

\usepackage{seqsplit}

\newcolumntype{P}[1]{>{\raggedright\arraybackslash\sloppy}p{#1}}
\newcommand{\cw}[1]{\seqsplit{#1}} % code-wrap helper

\usepackage{tabularx}
\usepackage{pdfpages}
\lstdefinestyle{defaultcode}{
  basicstyle=\ttfamily\small,
  backgroundcolor=\color{gray!5},
  frame=single,
  rulecolor=\color{gray!40},
  breaklines=true,
  breakatwhitespace=true,
  tabsize=2,
  numbers=left,
  numberstyle=\tiny\color{gray!70},
  numbersep=8pt,
  captionpos=b,
  showstringspaces=false,
  columns=fullflexible,
  keepspaces=true,
  upquote=true
}

\lstdefinelanguage{json}{
  basicstyle=\ttfamily\small,
  morestring=[b]",
  stringstyle=\color{black},
  showstringspaces=false,
  breaklines=true,
  literate=
   *{0}{{0}}{1}
    {1}{{1}}{1}
    {2}{{2}}{1}
    {3}{{3}}{1}
    {4}{{4}}{1}
    {5}{{5}}{1}
    {6}{{6}}{1}
    {7}{{7}}{1}
    {8}{{8}}{1}
    {9}{{9}}{1}
    {:}{{:}}{1}
    {,}{{,}}{1}
    {\{}{{\{}}{1}
    {\}}{{\}}}{1}
    {[}{{[}}{1}
    {]}{{]}}{1}
}

\lstdefinestyle{codejson}{
  style=defaultcode,
  language=json
}

\lstdefinestyle{codesql}{
  style=defaultcode,
  language=SQL,
  morekeywords={IDENTITY,UNIQUEIDENTIFIER,NVARCHAR,DATETIME2,GETUTCDATE,NEWID},
}

\lstdefinestyle{codecsharp}{
  style=defaultcode,
  language=[Sharp]C,
  morekeywords={async,await,var,record,init,Task,ValueTask,IServiceCollection,ILogger},
}

\lstdefinestyle{coderazor}{
  style=defaultcode,
  language=HTML,
}

\lstset{style=defaultcode}
% Custom commands for your thesis
\newcommand{\wipro}{WIPRO}
\newcommand{\todo}[1]{\textbf{TODO:} #1}

\newcommand{\img}[3][]{%
  \centering
  \includegraphics[#1]{#2}
  \caption{#3}
}

% Title meta
\title{Technische Verfahren zur Bestimmung der Ähnlichkeit von Bildern}
\subtitle{Untersuchung von Metriken und Feature-basierten Methoden im Kontext prozedural generierter Texturen}
\author{Eldar Omerovic}
\date{\today}

\begin{document}

\thispagestyle{empty}
\begin{center}
    
    \begin{flushleft}
        \includegraphics[width=4cm]{titlepage/HSLU_Logo.png}
    \end{flushleft}

    \vspace{5cm}

    \noindent\makebox[\linewidth]{\rule{\textwidth}{1pt}} 
    
    \begin{center}
        \Large \textbf{3D Modellieren für Echtzeitanwendungen}\\[5pt]
        \large 3DMOD4RT von Eldar Omerovic\\
    \end{center}
    
    \noindent\makebox[\linewidth]{\rule{\textwidth}{1pt}} 
    
    \vspace{1cm}
    
    \begin{center}
    
    \textbf{Autor}\\
    \textit{Eldar Omerovic}
    
    \vspace{1cm}
    
    \textbf{Semester}\\
    \textit{HS25}
    
    \vspace{1cm}
    
    \textbf{Über diese Zusammenfassung}\\
    \textit{Diese Zusammenfassung wurde zur Prüfungsvorbereitung im HS25 erstellt.}
    
    \end{center}
    
    \vfill
    \begin{flushright}
        Januar 2026
    \end{flushright}
\end{center}

\newpage

\pagenumbering{Roman}
\setcounter{page}{1}  

\newpage

\tableofcontents

\clearpage
\pagenumbering{arabic}
\setcounter{page}{1}

\chapter{Ausgangslage, Problemstellung und Zielsetzung}
\section{Ausgangslage}

Digitale Bilder sind in vielen Bereichen der Informatik zentral, etwa in der Computergraphik, der Bildsuche oder der Qualitätssicherung. Dabei stellt sich häufig die Frage, wie ähnlich sich zwei Bilder sind. Während Menschen visuelle Ähnlichkeit intuitiv erfassen, muss sie für technische Systeme formal definiert und messbar gemacht werden.

Zur Bestimmung von Bildähnlichkeit existieren unterschiedliche Ansätze, etwa pixelbasierte Distanzmetriken oder Feature-basierte Verfahren. Diese unterscheiden sich hinsichtlich Komplexität, Robustheit und Aussagekraft. Ein klarer Vergleich ihrer Eigenschaften fehlt jedoch häufig.

\section{Problemstellung}

Die technische Erfassung von Bildähnlichkeit ist nicht eindeutig definiert. Unterschiedliche Metriken können dasselbe Bildpaar verschieden bewerten. Zudem ist unklar, welche Verfahren sich für welche Bildtypen eignen und wo ihre Grenzen liegen.

Ohne eine systematische Analyse besteht die Gefahr, ungeeignete Metriken einzusetzen oder Ergebnisse falsch zu interpretieren. Dies erschwert eine fundierte Bewertung visueller Ähnlichkeit und reduziert die Aussagekraft entsprechender Systeme.

\section{Zielsetzung der Arbeit}

Ziel dieser Arbeit ist es, ausgewählte technische Verfahren zur Bestimmung der Bildähnlichkeit systematisch zu untersuchen und hinsichtlich ihrer Eigenschaften zu vergleichen. Dabei werden sowohl einfache Distanzmetriken als auch Feature-basierte Ansätze analysiert.

Die Arbeit soll Unterschiede, Stärken und Schwächen der Methoden herausarbeiten und eine nachvollziehbare Grundlage für deren Einsatz schaffen.

\chapter{Forschungsfragen und Methodik}
\section{Forschungsfragen}

\textbf{Hauptforschungsfrage}: Welche technischen Verfahren eignen sich, um die Ähnlichkeit zweier digitaler Bilder quantitativ zu bestimmen, und wie unterscheiden sich diese Verfahren in ihrer Aussagekraft?

\textbf{Unterforschungsfragen}:

\begin{enumerate}
    \item Welche Unterschiede zeigen sich zwischen pixelbasierten Distanzmetriken und Feature-basierten Verfahren bei typischen Bildvariationen (z.B. Helligkeit, Skalierung, Rotation, Rauschen)?
    \item Welche Kriterien lassen sich ableiten, um die Eignung einer Methode für einen gegebenen Anwendungsfall nachvollziehbar zu begründen?
\end{enumerate}

\section{Methodik}

Zur Beantwortung der Forschungsfragen wird ein vergleichendes Vorgehen gewählt, das aus zwei Schritten besteht. Erstens erfolgt eine zielgerichtete Literaturrecherche, um gängige Metriken und Feature-basierte Ansätze zu identifizieren und deren Annahmen sowie typische Einsatzgebiete zu beschreiben.

Zweitens wird eine konzeptionelle Vergleichsanalyse durchgeführt. Dafür werden ausgewählte Verfahren anhand definierter Vergleichskriterien (z. B. Robustheit gegenüber Bildtransformationen, Interpretierbarkeit, Rechenaufwand, Voraussetzungen an Vorverarbeitung) gegenübergestellt. Ergänzend werden exemplarische Testfälle beschrieben (Bildpaare mit kontrollierten Veränderungen), um die erwarteten Stärken und Grenzen der Methoden nachvollziehbar zu diskutieren.

Dieses Vorgehen ist geeignet, weil es sowohl den Forschungsstand systematisch abbildet als auch eine begründete, an Kriterien orientierte Einordnung ermöglicht, ohne bereits eine vollständige Implementierung vorwegzunehmen.

\subsection{Verwendung von KI}

Für die Recherche wurden KI-gestützte Tools punktuell genutzt, insbesondere zur Unterstützung bei der Formulierung von Suchbegriffen und zur ersten thematischen Orientierung. Die Auswahl der wissenschaftlichen Quellen sowie die Überprüfung der bibliografischen Angaben erfolgte anschliessend manuell anhand der Originalquellen. Für sprachliche Überarbeitungen (Grammatik, Rechtschreibung, Satzstellung) wurden KI-Tools vereinzelt eingesetzt. Inhaltliche Aussagen, Argumentation und Struktur wurden eigenständig erarbeitet und abschliessend kritisch geprüft.

\chapter{Persönliche Reflexion des Vorgehens}
Bei der Themenwahl war mir wichtig, etwas zu nehmen, das zur Bachelorarbeit passt, aber im WSFM nicht bereits eine Umsetzung verlangt. Deshalb untersuche ich, wie Bildähnlichkeit technisch messbar gemacht wird, statt eine konkrete Pipeline zu bauen. So kann ich die Arbeit klar als Forschungsskizze abgrenzen.

Herausfordernd war die Eingrenzung: \enquote{Bildähnlichkeit} ist breit und hängt stark vom Anwendungsfall ab. Entsprechend schwierig war es, Forschungsfragen zu formulieren, die präzise sind, aber ohne grosse Experimente beantwortbar bleiben. Auch die Festlegung von Vergleichskriterien ist heikel, weil man damit ungewollt einzelne Verfahren bevorzugt.

Eine Limitation ist die Abhängigkeit von der verfügbaren Literatur und deren Evaluationsqualität. Ohne eigene Implementierung bleibt offen, wie sich Methoden in einer konkreten Pipeline verhalten. Rückblickend würde ich früher ein fixes Set an Bildvariationen und Kriterien definieren, um Recherche und Argumentation stringenter auszurichten.

\chapter{Literaturüberblick und Einordnung}
\section{Zentrale Fachliteratur}

\textcite{wang2004ssim} führen mit dem Structural Similarity Index (SSIM) eine Bildähnlichkeitsmetrik ein, die nicht primär den pixelweisen Fehler misst, sondern die Ähnlichkeit über Luminanz-, Kontrast- und Strukturanteile modelliert. Für meine Arbeit ist das besonders hilfreich, weil SSIM gut zeigt, warum \enquote{Pixelgleichheit} nicht automatisch \enquote{visuell ähnlich} bedeutet. SSIM ist zudem klar definierbar und damit für einen methodischen Vergleich geeignet: Ich kann die Metrik anhand Kriterien wie Interpretierbarkeit, Voraussetzungen (z. B. gleiche Bildausrichtung) und Robustheit gegenüber typischen Veränderungen einordnen. Gleichzeitig macht die Quelle auch eine zentrale Grenze sichtbar: Sobald Bilder geometrisch verändert sind (z. B. Rotation oder Verschiebung), kann SSIM trotz subjektiver Ähnlichkeit stark abfallen. Genau diese Spannung ist für meine Forschungsfrage relevant, weil sie erklärt, warum man je nach Anwendungsfall über Alternativen nachdenken muss.

\textcite{lowe2004sift} beschreibt mit SIFT einen Feature-basierten Ansatz, bei dem Bilder über lokale Keypoints und Deskriptoren verglichen werden. Diese Perspektive ist für meine Arbeit wichtig, weil sie Ähnlichkeit nicht als globalen Pixelvergleich, sondern als Übereinstimmung lokaler Merkmale operationalisiert. Das passt direkt zur Unterfrage, wie Verfahren auf Bildvariationen reagieren: SIFT zielt explizit auf Robustheit gegenüber Skalierung und Rotation ab, was bei klassischen Metriken ein Schwachpunkt ist. Für die Einordnung ist aber auch relevant, dass Feature-Matching nicht automatisch \enquote{semantische} oder \enquote{perzeptuelle} Ähnlichkeit garantiert: Zwei Bilder können viele Matches haben, obwohl sie inhaltlich anders wirken (oder umgekehrt bei texturartigen Bildern mit wenigen stabilen Keypoints). Daraus leite ich ab, dass die Wahl eines Verfahrens immer an Kriterien gekoppelt werden muss und nicht nur an \enquote{funktioniert irgendwie}.

\clearpage
\section{Methodenliteratur}

Als methodische Grundlage nutze ich \textcite{kitchenham2007slr}, die Leitlinien für systematische Literaturreviews im Software Engineering formulieren. Für mein Vorgehen ist das nützlich, weil ich damit transparent begründen kann, wie ich Literatur auswähle (Suchbegriffe, Datenbanken, In-/Exklusionskriterien) und wie ich die Ergebnisse strukturiere. Auch wenn WSFM keinen vollständigen SLR verlangt, helfen mir die Prinzipien (Nachvollziehbarkeit, Reproduzierbarkeit im kleinen Rahmen), die Auswahl der Quellen zu legitimieren und die Vergleichsanalyse weniger subjektiv zu machen. Damit stützt die Methodenliteratur direkt den Anspruch meiner Arbeit, nicht nur Verfahren aufzuzählen, sondern sie anhand definierter Kriterien einzuordnen.

\chapter{Verzeichnisse}
\raggedright
\section{Literaturverzeichnis}

\printbibliography[heading=none]

\section{Abkürzungsverzeichnis}
\begin{tabularx}{\textwidth}{@{}lX@{}}
\textbf{Abkürzung} & \textbf{Bedeutung} \\
\hline
WSFM & Wissenschaftliches Schreiben \& Forschungsmethodik \\
KI & Künstliche Intelligenz \\
GenKI & Generative Künstliche Intelligenz \\
SSIM & Structural Similarity Index Measure (Metrik zur Bildähnlichkeit) \\
SIFT & Scale-Invariant Feature Transform (Feature-basiertes Verfahren zur Bildähnlichkeit) \\
SLR & Systematic Literature Review (Methodik zur strukturierten Literaturrecherche) \\
BAA & Bachelorarbeit (Bachelor-Arbeit) \\
FS26 & Frühlingssemester 2026 \\
EBSE & Evidence-Based Software Engineering (Software Engineering basierend auf evidenzbasierten Methoden) \\
IEEE & Institute of Electrical and Electronics Engineers (Organisation, die den IEEE-Zitierstil definiert) 
\end{tabularx}

\chapter{Anhänge}

\section{Anhang A: Aufgabenstellung (WSFM)}

\textbf{Titel}: Technische Verfahren zur Bestimmung der Bildähnlichkeit

\textbf{Ausgangslage und Problemstellung}

In vielen Anwendungsfällen ist es notwendig, die Ähnlichkeit zweier digitaler Bilder quantitativ zu bestimmen, zum Beispiel für Bildsuche, Qualitätskontrolle oder den Vergleich von Bildversionen. Während Menschen Ähnlichkeit visuell intuitiv beurteilen, benötigen technische Systeme dafür formale Kriterien und messbare Kennzahlen. Je nach gewählter Methode (z. B. pixelbasierte Metriken oder Feature-basierte Ansätze) können identische Bildpaare unterschiedlich bewertet werden. Zudem ist oft unklar, welche Verfahren bei typischen Bildveränderungen (Helligkeit, Rauschen, Skalierung, Rotation) robust bleiben und welche schnell an Aussagekraft verlieren.

\textbf{Ziel der Arbeit und erwartete Resultate}

Ziel dieser Arbeit ist es, ausgewählte technische Verfahren zur Bestimmung der Bildähnlichkeit systematisch zu untersuchen und vergleichend einzuordnen. Dabei sollen Stärken, Schwächen und Einsatzgrenzen der Ansätze transparent herausgearbeitet werden. Als Resultat wird eine nachvollziehbare Gegenüberstellung erwartet, die Kriterien liefert, um je nach Anwendungskontext begründet zu entscheiden, welche Methode geeignet ist.

\textbf{Gewünschte Methoden / Vorgehen}

\begin{itemize}
  \item Zielgerichtete Literaturrecherche zu etablierten Metriken und Feature-basierten Verfahren zur     Bildähnlichkeit
  \item Strukturierte Vergleichsanalyse anhand definierter Kriterien (z. B. Robustheit gegenüber Transformationen, Interpretierbarkeit, Rechenaufwand, Voraussetzungen)
  \item Diskussion von Anwendungsfällen und Ableitung von Empfehlungen zur Methodenwahl
\end{itemize}

\textit{Hinweis: Die Bachelorarbeit-Aufgabenstellung wird im Anhang B beigefügt, da sie den Kontext und die Motivation für die Themenwahl liefert, jedoch nicht 1:1 Gegenstand dieser WSFM-Arbeit ist.}

\clearpage
\section{Anhang B: Bezug zur Bachelorarbeit und BAA Aufgabenstellung}

Die vorliegende Forschungsskizze wurde thematisch in Anlehnung an die Bachelorarbeit \enquote{Prozedurales und regelbasiertes Generieren von Texturen in der Computergraphik} erstellt. Dabei wird jedoch nicht die zentrale Fragestellung der Bachelorarbeit direkt als Thema übernommen. Stattdessen fokussiert diese Arbeit ein verwandtes Teilthema, das im Kontext der Bachelorarbeit als notwendiger Baustein bzw. erster Schritt relevant ist: die technische Bestimmung der Ähnlichkeit von Bildern.

Im Anschluss folgt die zugrundeliegende provisorische Aufgabenstellung, welche als Ausgangspunkt für die Forschungsskizze dient.

\includepdf[
  pages=-,
  scale=0.85,
  pagecommand={\thispagestyle{plain}}
]{tex/chapters/anhaenge/Aufgabenstellung_BAA_Omerovic_Eldar.pdf}

\end{document}
