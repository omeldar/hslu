\section{Zentrale Fachliteratur}

\textcite{wang2004ssim} führen mit dem Structural Similarity Index (SSIM) eine Bildähnlichkeitsmetrik ein, die nicht primär den pixelweisen Fehler misst, sondern die Ähnlichkeit über Luminanz-, Kontrast- und Strukturanteile modelliert. Für meine Arbeit ist das besonders hilfreich, weil SSIM gut zeigt, warum \enquote{Pixelgleichheit} nicht automatisch \enquote{visuell ähnlich} bedeutet. SSIM ist zudem klar definierbar und damit für einen methodischen Vergleich geeignet: Ich kann die Metrik anhand Kriterien wie Interpretierbarkeit, Voraussetzungen (z. B. gleiche Bildausrichtung) und Robustheit gegenüber typischen Veränderungen einordnen. Gleichzeitig macht die Quelle auch eine zentrale Grenze sichtbar: Sobald Bilder geometrisch verändert sind (z. B. Rotation oder Verschiebung), kann SSIM trotz subjektiver Ähnlichkeit stark abfallen. Genau diese Spannung ist für meine Forschungsfrage relevant, weil sie erklärt, warum man je nach Anwendungsfall über Alternativen nachdenken muss.

\textcite{lowe2004sift} beschreibt mit SIFT einen Feature-basierten Ansatz, bei dem Bilder über lokale Keypoints und Deskriptoren verglichen werden. Diese Perspektive ist für meine Arbeit wichtig, weil sie Ähnlichkeit nicht als globalen Pixelvergleich, sondern als Übereinstimmung lokaler Merkmale operationalisiert. Das passt direkt zur Unterfrage, wie Verfahren auf Bildvariationen reagieren: SIFT zielt explizit auf Robustheit gegenüber Skalierung und Rotation ab, was bei klassischen Metriken ein Schwachpunkt ist. Für die Einordnung ist aber auch relevant, dass Feature-Matching nicht automatisch \enquote{semantische} oder \enquote{perzeptuelle} Ähnlichkeit garantiert: Zwei Bilder können viele Matches haben, obwohl sie inhaltlich anders wirken (oder umgekehrt bei texturartigen Bildern mit wenigen stabilen Keypoints). Daraus leite ich ab, dass die Wahl eines Verfahrens immer an Kriterien gekoppelt werden muss und nicht nur an \enquote{funktioniert irgendwie}.

\clearpage
\section{Methodenliteratur}

Als methodische Grundlage nutze ich \textcite{kitchenham2007slr}, die Leitlinien für systematische Literaturreviews im Software Engineering formulieren. Für mein Vorgehen ist das nützlich, weil ich damit transparent begründen kann, wie ich Literatur auswähle (Suchbegriffe, Datenbanken, In-/Exklusionskriterien) und wie ich die Ergebnisse strukturiere. Auch wenn WSFM keinen vollständigen SLR verlangt, helfen mir die Prinzipien (Nachvollziehbarkeit, Reproduzierbarkeit im kleinen Rahmen), die Auswahl der Quellen zu legitimieren und die Vergleichsanalyse weniger subjektiv zu machen. Damit stützt die Methodenliteratur direkt den Anspruch meiner Arbeit, nicht nur Verfahren aufzuzählen, sondern sie anhand definierter Kriterien einzuordnen.