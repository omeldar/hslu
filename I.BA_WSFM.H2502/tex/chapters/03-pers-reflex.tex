Bei der Themenwahl war mir wichtig, etwas zu nehmen, das zur Bachelorarbeit passt, aber im WSFM nicht bereits eine Umsetzung verlangt. Deshalb untersuche ich, wie Bildähnlichkeit technisch messbar gemacht wird, statt eine konkrete Pipeline zu bauen. So kann ich die Arbeit klar als Forschungsskizze abgrenzen.

Herausfordernd war die Eingrenzung: \enquote{Bildähnlichkeit} ist breit und hängt stark vom Anwendungsfall ab. Entsprechend schwierig war es, Forschungsfragen zu formulieren, die präzise sind, aber ohne grosse Experimente beantwortbar bleiben. Auch die Festlegung von Vergleichskriterien ist heikel, weil man damit ungewollt einzelne Verfahren bevorzugt.

Eine Limitation ist die Abhängigkeit von der verfügbaren Literatur und deren Evaluationsqualität. Ohne eigene Implementierung bleibt offen, wie sich Methoden in einer konkreten Pipeline verhalten. Rückblickend würde ich früher ein fixes Set an Bildvariationen und Kriterien definieren, um Recherche und Argumentation stringenter auszurichten.