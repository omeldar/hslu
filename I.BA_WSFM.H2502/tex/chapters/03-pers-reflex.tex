Bei der Themenwahl war mir wichtig, etwas zu nehmen, das einerseits zur Bachelorarbeit passt, aber für WSFM nicht schon eine halbe Umsetzung verlangt. Darum habe ich den Fokus bewusst auf die Frage gelegt, wie Bildähnlichkeit technisch überhaupt messbar gemacht wird, statt direkt auf eine konkrete Implementierung oder Optimierung hinzuarbeiten. Das hat geholfen, die Arbeit als Forschungsskizze sauber abzugrenzen.

Schwierig fand ich vor allem die Eingrenzung: \enquote{Bildähnlichkeit} ist extrem breit und je nach Anwendungsfall bedeutet \enquote{ähnlich} etwas anderes. Die grösste Herausforderung war deshalb, Forschungsfragen so zu formulieren, dass sie nicht zu allgemein sind, aber trotzdem ohne umfangreiche Experimente beantwortbar bleiben. Methodisch war auch die Wahl der Vergleichskriterien anspruchsvoll, weil man schnell Kriterien wählt, die bereits ein bestimmtes Verfahren bevorzugen.

Eine Limitation sehe ich darin, dass meine Analyse stark davon abhängt, welche Literatur ich finde und wie gut die Methoden darin beschrieben und evaluiert sind. Ohne eigene Implementierung bleibt zudem offen, wie sich einzelne Verfahren in einer konkreten Pipeline tatsächlich verhalten. Im Nachhinein würde ich früher ein kleines, fixes Set an Bildvariationen und Bewertungskriterien definieren, damit die Recherche und die Argumentation noch besser aufeinander abgestimmt sind.