
\section{Anhang A: Aufgabenstellung (WSFM)}

\textbf{Titel}: Technische Verfahren zur Bestimmung der Bildähnlichkeit

\textbf{Ausgangslage und Problemstellung}

In vielen Anwendungsfällen ist es notwendig, die Ähnlichkeit zweier digitaler Bilder quantitativ zu bestimmen, zum Beispiel für Bildsuche, Qualitätskontrolle oder den Vergleich von Bildversionen. Während Menschen Ähnlichkeit visuell intuitiv beurteilen, benötigen technische Systeme dafür formale Kriterien und messbare Kennzahlen. Je nach gewählter Methode (z. B. pixelbasierte Metriken oder Feature-basierte Ansätze) können identische Bildpaare unterschiedlich bewertet werden. Zudem ist oft unklar, welche Verfahren bei typischen Bildveränderungen (Helligkeit, Rauschen, Skalierung, Rotation) robust bleiben und welche schnell an Aussagekraft verlieren.

\textbf{Ziel der Arbeit und erwartete Resultate}

Ziel dieser Arbeit ist es, ausgewählte technische Verfahren zur Bestimmung der Bildähnlichkeit systematisch zu untersuchen und vergleichend einzuordnen. Dabei sollen Stärken, Schwächen und Einsatzgrenzen der Ansätze transparent herausgearbeitet werden. Als Resultat wird eine nachvollziehbare Gegenüberstellung erwartet, die Kriterien liefert, um je nach Anwendungskontext begründet zu entscheiden, welche Methode geeignet ist.

\textbf{Gewünschte Methoden / Vorgehen}

\begin{itemize}
  \item Zielgerichtete Literaturrecherche zu etablierten Metriken und Feature-basierten Verfahren zur     Bildähnlichkeit
  \item Strukturierte Vergleichsanalyse anhand definierter Kriterien (z. B. Robustheit gegenüber Transformationen, Interpretierbarkeit, Rechenaufwand, Voraussetzungen)
  \item Diskussion von Anwendungsfällen und Ableitung von Empfehlungen zur Methodenwahl
\end{itemize}

\textit{Hinweis: Die Bachelorarbeit-Aufgabenstellung wird im Anhang B beigefügt, da sie den Kontext und die Motivation für die Themenwahl liefert, jedoch nicht 1:1 Gegenstand dieser WSFM-Arbeit ist.}

\clearpage
\section{Anhang B: Bezug zur Bachelorarbeit und BAA Aufgabenstellung}

Die vorliegende Forschungsskizze wurde thematisch in Anlehnung an die Bachelorarbeit \enquote{Prozedurales und regelbasiertes Generieren von Texturen in der Computergraphik} erstellt. Dabei wird jedoch nicht die zentrale Fragestellung der Bachelorarbeit direkt als Thema übernommen. Stattdessen fokussiert diese Arbeit ein verwandtes Teilthema, das im Kontext der Bachelorarbeit als notwendiger Baustein bzw. erster Schritt relevant ist: die technische Bestimmung der Ähnlichkeit von Bildern.

Im Anschluss folgt die zugrundeliegende provisorische Aufgabenstellung, welche als Ausgangspunkt für die Forschungsskizze dient.

\includepdf[
  pages=-,
  scale=0.85,
  pagecommand={\thispagestyle{plain}}
]{tex/chapters/anhaenge/Aufgabenstellung_BAA_Omerovic_Eldar.pdf}