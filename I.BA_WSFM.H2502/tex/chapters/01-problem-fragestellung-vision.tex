\section{Ausgangslage}

Digitale Bilder sind in vielen Bereichen der Informatik zentral, etwa in der Computergraphik, der Bildsuche oder der Qualitätssicherung. Dabei stellt sich häufig die Frage, wie ähnlich sich zwei Bilder sind. Während Menschen visuelle Ähnlichkeit intuitiv erfassen, muss sie für technische Systeme formal definiert und messbar gemacht werden.

Zur Bestimmung von Bildähnlichkeit existieren unterschiedliche Ansätze, etwa pixelbasierte Distanzmetriken oder Feature-basierte Verfahren. Diese unterscheiden sich hinsichtlich Komplexität, Robustheit und Aussagekraft. Ein klarer Vergleich ihrer Eigenschaften fehlt jedoch häufig.

\section{Problemstellung}

Die technische Erfassung von Bildähnlichkeit ist nicht eindeutig definiert. Unterschiedliche Metriken können dasselbe Bildpaar verschieden bewerten. Zudem ist unklar, welche Verfahren sich für welche Bildtypen eignen und wo ihre Grenzen liegen.

Ohne eine systematische Analyse besteht die Gefahr, ungeeignete Metriken einzusetzen oder Ergebnisse falsch zu interpretieren. Dies erschwert eine fundierte Bewertung visueller Ähnlichkeit und reduziert die Aussagekraft entsprechender Systeme.

\section{Zielsetzung der Arbeit}

Ziel dieser Arbeit ist es, ausgewählte technische Verfahren zur Bestimmung der Bildähnlichkeit systematisch zu untersuchen und hinsichtlich ihrer Eigenschaften zu vergleichen. Dabei werden sowohl einfache Distanzmetriken als auch Feature-basierte Ansätze analysiert.

Die Arbeit soll Unterschiede, Stärken und Schwächen der Methoden herausarbeiten und eine nachvollziehbare Grundlage für deren Einsatz schaffen.