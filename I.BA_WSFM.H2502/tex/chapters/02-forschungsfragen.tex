\section{Forschungsfragen}

\textbf{Hauptforschungsfrage}: Welche technischen Verfahren eignen sich, um die Ähnlichkeit zweier digitaler Bilder quantitativ zu bestimmen, und wie unterscheiden sich diese Verfahren in ihrer Aussagekraft?

\textbf{Unterforschungsfragen}:

\begin{enumerate}
    \item Welche Unterschiede zeigen sich zwischen pixelbasierten Distanzmetriken und Feature-basierten Verfahren bei typischen Bildvariationen (z.B. Helligkeit, Skalierung, Rotation, Rauschen)?
    \item Welche Kriterien lassen sich ableiten, um die Eignung einer Methode für einen gegebenen Anwendungsfall nachvollziehbar zu begründen?
\end{enumerate}

\section{Methodik}

Zur Beantwortung der Forschungsfragen wird ein vergleichendes Vorgehen gewählt, das aus zwei Schritten besteht. Erstens erfolgt eine zielgerichtete Literaturrecherche, um gängige Metriken und Feature-basierte Ansätze zu identifizieren und deren Annahmen sowie typische Einsatzgebiete zu beschreiben.

Zweitens wird eine konzeptionelle Vergleichsanalyse durchgeführt. Dazu werden ausgewählte Verfahren anhand definierter Kriterien (z. B. Robustheit gegenüber Transformationen, Interpretierbarkeit, Rechenaufwand und Vorverarbeitung) gegenübergestellt. Ergänzend werden exemplarische Testfälle mit kontrollierten Bildveränderungen beschrieben, um Stärken und Grenzen der Methoden nachvollziehbar zu diskutieren.

Dieses Vorgehen ist geeignet, weil es sowohl den Forschungsstand systematisch abbildet als auch eine begründete, an Kriterien orientierte Einordnung ermöglicht, ohne bereits eine vollständige Implementierung vorwegzunehmen.

KI-Tools wurden punktuell zur Orientierung bei der Recherche sowie für einzelne sprachliche Korrekturen genutzt. Inhaltliche Aussagen, Quellenwahl und Argumentation wurden eigenständig erstellt und anhand der Originalquellen überprüft.