\section{Forschungsfragen}

\textbf{Hauptforschungsfrage}: Welche technischen Verfahren eignen sich, um die Ähnlichkeit zweier digitaler Bilder quantitativ zu bestimmen, und wie unterscheiden sich diese Verfahren in ihrer Aussagekraft?

\textbf{Unterforschungsfragen}:

\begin{enumerate}
    \item Welche Unterschiede zeigen sich zwischen pixelbasierten Distanzmetriken und Feature-basierten Verfahren bei typischen Bildvariationen (z.B. Helligkeit, Skalierung, Rotation, Rauschen)?
    \item Welche Kriterien lassen sich ableiten, um die Eignung einer Methode für einen gegebenen Anwendungsfall nachvollziehbar zu begründen?
\end{enumerate}

\section{Methodik}

Zur Beantwortung der Forschungsfragen wird ein vergleichendes Vorgehen gewählt, das aus zwei Schritten besteht. Erstens erfolgt eine zielgerichtete Literaturrecherche, um gängige Metriken und Feature-basierte Ansätze zu identifizieren und deren Annahmen sowie typische Einsatzgebiete zu beschreiben.

Zweitens wird eine konzeptionelle Vergleichsanalyse durchgeführt. Dafür werden ausgewählte Verfahren anhand definierter Vergleichskriterien (z. B. Robustheit gegenüber Bildtransformationen, Interpretierbarkeit, Rechenaufwand, Voraussetzungen an Vorverarbeitung) gegenübergestellt. Ergänzend werden exemplarische Testfälle beschrieben (Bildpaare mit kontrollierten Veränderungen), um die erwarteten Stärken und Grenzen der Methoden nachvollziehbar zu diskutieren.

Dieses Vorgehen ist geeignet, weil es sowohl den Forschungsstand systematisch abbildet als auch eine begründete, an Kriterien orientierte Einordnung ermöglicht, ohne bereits eine vollständige Implementierung vorwegzunehmen.

\subsection{Verwendung von KI}

Für die Recherche wurden KI-gestützte Tools punktuell genutzt, insbesondere zur Unterstützung bei der Formulierung von Suchbegriffen und zur ersten thematischen Orientierung. Die Auswahl der wissenschaftlichen Quellen sowie die Überprüfung der bibliografischen Angaben erfolgte anschliessend manuell anhand der Originalquellen. Für sprachliche Überarbeitungen (Grammatik, Rechtschreibung, Satzstellung) wurden KI-Tools vereinzelt eingesetzt. Inhaltliche Aussagen, Argumentation und Struktur wurden eigenständig erarbeitet und abschliessend kritisch geprüft.