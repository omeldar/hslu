\section{Motivation: Warum braucht man Global Illumination?}
Klassische Echtzeit-Beleuchtung (z.\,B. Ambient + Diffus + Specular) betrachtet oft nur: \enquote{Lampe trifft direkt auf Oberfläche}.
In der Realität passiert aber viel mehr:
\begin{itemize}
  \item Licht \textbf{springt mehrfach} zwischen Oberflächen hin und her (indirektes Licht).
  \item Schatten werden dadurch \textbf{weicher}, dunkle Bereiche bekommen \textbf{Aufhellung}.
  \item Farben \textbf{bluten} ab (Color Bleeding): eine rote Wand färbt benachbarte Flächen leicht rot.
\end{itemize}
\textbf{Global Illumination (GI)} ist der Sammelbegriff für Methoden, die diese Mehrfachwechselwirkung mitmodellieren.

\textbf{Merksatz:} GI ist \enquote{Licht transportiert sich durch die ganze Szene}, nicht nur direkt von der Lampe.

\section{Begriffe: Reflectance, Radiance, und die Rendering Equation (Kajiya-Kontext)}
\subsection*{Radiance (Strahldichte) als Konzept}
\textbf{Radiance} ist die zentrale Grösse in der realistischen Beleuchtung.
Konzeptionell bedeutet sie:
\enquote{Wie viel Licht kommt aus einer bestimmten Richtung an einem Punkt an (oder geht in eine Richtung weg)?}

Wichtig für die mündliche Prüfung:
\begin{itemize}
  \item Radiance ist richtungsabhängig: aus welcher Richtung Licht kommt, ist entscheidend.
  \item Sie ist das, was eine Kamera letztlich \enquote{sieht} (Richtung zur Kamera).
\end{itemize}

\subsection*{Reflectance und BRDF (Einordnung)}
\textbf{Reflectance} beschreibt, wie ein Material eingehendes Licht in ausgehendes Licht umwandelt.
Das wird meist als \textbf{BRDF} modelliert:
\enquote{Welche Anteile aus Einfallsrichtungen werden in eine Ausfallsrichtung reflektiert?}

\subsection*{Rendering Equation (Kajiya): Was sie aussagt, ohne Mathe}
Die \textbf{Rendering Equation} (Kajiya) ist eine konzeptionelle \enquote{Bilanzgleichung}:
\begin{quote}
\enquote{Das Licht, das du in Richtung Kamera siehst, besteht aus selbst emittiertem Licht plus reflektiertem Licht aus allen Richtungen.}
\end{quote}

\textbf{Warum ist das schwierig?}
\begin{itemize}
  \item \enquote{Aus allen Richtungen} bedeutet: theoretisch unendlich viele Richtungen.
  \item Indirektes Licht bedeutet: Licht, das von anderen Punkten kommt, die selbst wieder von anderen Punkten beleuchtet werden.
  \item Das ist ein gekoppeltes Problem über die ganze Szene.
\end{itemize}

\textbf{Merksatz:} Die Rendering Equation ist \enquote{Lichtbilanz am Punkt} und erklärt, warum GI teuer ist.

\section{Direkte vs. indirekte Beleuchtung}
\subsection*{Direkte Beleuchtung}
\textbf{Direkt} heisst: Licht kommt ohne Zwischenreflexion von der Lichtquelle zum Punkt.
Konzeptionell:
\begin{itemize}
  \item Schatten entstehen hier durch \enquote{Lichtquelle sichtbar oder verdeckt}.
  \item Viele Raster- und einfache Ray-Tracing-Ansätze starten mit direktem Licht.
\end{itemize}

\clearpage
\subsection*{Indirekte Beleuchtung}
\textbf{Indirekt} heisst: Licht erreicht den Punkt erst nach einer oder mehreren Reflexionen an anderen Oberflächen.
Typische Effekte:
\begin{itemize}
  \item Aufhellung in Schattenbereichen
  \item Color Bleeding
  \item weichere, plausiblere Raumwirkung
\end{itemize}

\textbf{Merksatz:} Direkt ist \enquote{Lampe $\rightarrow$ Punkt}, indirekt ist \enquote{Lampe $\rightarrow$ andere Fläche(n) $\rightarrow$ Punkt}.

\section{Radiosity: Grundidee}
\textbf{Radiosity} ist eine GI-Methode, die besonders gut für \textbf{diffuse} (matte) Szenen passt.
Sie denkt nicht in einzelnen Strahlen, sondern in \textbf{Energieaustausch zwischen Flächen}.

\subsection*{Konzept}
\begin{itemize}
  \item Man teilt die Szene in Flächenstücke (Patches).
  \item Jedes Patch tauscht Lichtenergie mit anderen Patches aus.
  \item Das Ergebnis ist eine \enquote{stationäre} Verteilung von diffusem Licht (wie hell ist jedes Patch insgesamt).
\end{itemize}

\textbf{Warum ist das attraktiv?}
\begin{itemize}
  \item Sehr gut für Innenräume mit diffusem Licht.
  \item Liefert weiche Schatten und Color Bleeding auf natürliche Weise.
\end{itemize}

\textbf{Einschränkung:} Klassische Radiosity ist primär für diffuse Reflexion gedacht, nicht für spiegelnde Materialien.

\clearpage
\section{Formfaktoren (konzeptionell)}
\textbf{Formfaktoren} sind das Herz von Radiosity.
Sie sagen konzeptionell:
\enquote{Welcher Anteil des Lichtes von Fläche A erreicht Fläche B, nur aufgrund der Geometrie?}

Dabei spielen Dinge eine Rolle wie:
\begin{itemize}
  \item Abstand zwischen Flächen
  \item Ausrichtung (Winkel) der Flächen
  \item Sichtbarkeit (wenn etwas dazwischen liegt, ist der Anteil kleiner oder null)
\end{itemize}

\textbf{Merksatz:} Formfaktor = \enquote{Geometrischer Kopplungsgrad} zwischen zwei Flächen.

\section{Iterative Verfahren: progressive und hierarchical Radiosity}
Radiosity führt auf ein grosses gekoppeltes Gleichungssystem (viele Patches).
In der Praxis löst man das oft iterativ.

\subsection*{Progressive Radiosity (Grundidee)}
\begin{itemize}
  \item Statt alles auf einmal zu lösen, verteilt man Energie schrittweise.
  \item Man nimmt typischerweise das Patch mit der meisten \enquote{noch zu verteilenden} Energie und \enquote{schiesst} diese auf andere Patches.
  \item Vorteil: Man bekommt schnell ein brauchbares Zwischenbild und verbessert es laufend.
\end{itemize}
\textbf{Konzept:} \enquote{Erst das wichtigste Licht verteilen, dann verfeinern.}

\subsection*{Hierarchical Radiosity (Grundidee)}
\begin{itemize}
  \item Nicht alle Patch-Paare brauchen die gleiche Detailtiefe.
  \item Man arbeitet mit einer Hierarchie (grob $\rightarrow$ fein) und verfeinert nur dort, wo es nötig ist.
  \item Vorteil: Skalierbarer bei grossen Szenen.
\end{itemize}
\textbf{Konzept:} \enquote{Detail nur dort, wo es einen sichtbaren Unterschied macht.}

\section{Weitere Lösungsansätze: Photon Mapping, Monte-Carlo Ray Tracing}
\subsection*{Photon Mapping (Einordnung)}
Photon Mapping ist eine zweistufige, \enquote{energiegetriebene} Methode:
\begin{itemize}
  \item \textbf{Phase 1:} Man schickt viele \enquote{Photonen} von Lichtquellen in die Szene und speichert, wo sie auftreffen (Photon Map).
  \item \textbf{Phase 2:} Beim Rendern schätzt man indirektes Licht, indem man in der Umgebung eines Punktes in der Photon Map nach gespeicherter Energie sucht.
\end{itemize}

\textbf{Wofür ist das gut?}
\begin{itemize}
  \item Indirektes Licht effizient abschätzen
  \item Caustics (Lichtbündelungen durch Glas/Wasser) deutlich besser als viele einfache Verfahren
\end{itemize}

\textbf{Trade-off:} Qualität hängt von Anzahl Photonen und Suchradius ab; kann zu \enquote{Flecken} oder Glättung führen.

\subsection*{Monte-Carlo Ray Tracing (Einordnung)}
Monte-Carlo Methoden lösen die Rendering Equation durch \textbf{stochastisches Sampling}:
\begin{itemize}
  \item Statt \enquote{alle Richtungen} exakt zu integrieren, nimmt man zufällig verteilte Richtungs-Samples.
  \item Mit genug Samples nähert man die richtige Lösung an.
\end{itemize}

\textbf{Warum ist das der moderne Standard für Realismus?}
\begin{itemize}
  \item Sehr allgemein: funktioniert für diffuse, glossy, spiegelnde Materialien
  \item Kann indirektes Licht beliebiger Bounce-Tiefe modellieren (Path Tracing als bekanntes Beispiel)
\end{itemize}

\textbf{Hauptproblem:} Rauschen.
Wenige Samples $\rightarrow$ sichtbares Noise, viele Samples $\rightarrow$ teuer.
Deshalb nutzt man oft \textbf{Denoising} und \textbf{gute Sampling-Strategien}.

\textbf{Merksatz:} Monte-Carlo Ray Tracing ist \enquote{Rendering Equation durch Zufallssamples} -- flexibel, aber noisy.

\section{Einordnung zu WebGPU (als Referenz-Kontext)}
WebGPU ist eine Raster-First API, aber viele GI-Konzepte lassen sich einordnen:
\begin{itemize}
  \item \textbf{Monte-Carlo/Path Tracing} kann man als Compute-Shader implementieren (Rays pro Pixel, Akkumulation in Texturen).
  \item \textbf{Hybrid-Rendering} ist typisch: Rasterisierung für das Grundbild, stochastische Effekte (GI, Reflexionen) als Zusatz.
  \item \textbf{Radiosity- oder Photon-Ideen} kann man als Precompute oder als approximative Screen-/Probe-Techniken umsetzen.
\end{itemize}

\section{Typische konzeptionelle Prüfungsfragen (Kernantworten)}
\begin{itemize}
  \item \textbf{Was ist die Rendering Equation in einem Satz (Kajiya)?}
    -- Licht in Blickrichtung = Emission + reflektiertes Licht aus allen Richtungen, gewichtet durch Materialverhalten.
  \item \textbf{Direkte vs. indirekte Beleuchtung?}
    -- Direkt: Lichtquelle trifft direkt; indirekt: Licht kommt erst nach Reflexionen von anderen Flächen.
  \item \textbf{Wofür steht Radiance konzeptionell?}
    -- Wie viel Licht aus einer Richtung an einem Punkt ankommt bzw. in eine Richtung ausgeht; das ist, was die Kamera \enquote{sieht}.
  \item \textbf{Was macht Radiosity besonders?}
    -- Energieaustausch zwischen Flächen, besonders für diffuse Szenen, liefert weiche Schatten und Color Bleeding.
  \item \textbf{Was ist ein Formfaktor?}
    -- Geometrischer Anteil: wie viel Energie von Fläche A Fläche B erreicht (Abstand, Winkel, Sichtbarkeit).
  \item \textbf{Progressive vs. hierarchical Radiosity?}
    -- Progressive verteilt Licht schrittweise (früh brauchbar); hierarchical verfeinert adaptiv nur wo nötig.
  \item \textbf{Photon Mapping vs. Monte-Carlo Ray Tracing?}
    -- Photon Mapping speichert Lichtenergie-Treffer und schätzt später; Monte-Carlo sampelt Richtungen direkt und nähert die Rendering Equation an (mit Noise).
\end{itemize}
