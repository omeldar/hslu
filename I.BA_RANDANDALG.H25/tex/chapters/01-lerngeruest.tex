In diesem Kapitel befindet sich das Lerngerüst.

\section{SW1: Einleitung}

\begin{enumerate}[label=(\alph*)]
    \item Für welchen Zweck wird Zufall in der Informatik eingesetzt?
    \item Beispiel von SW1 zur Motivation
    \begin{enumerate}[label=(\roman*)]
        \item Kann ich das Problem erklären?
        \item Was ist die beste Kommunikationskomplexität deterministischer Algorithmen?
        \item Kenne ich den randomisierten Algorithmus?
        \item Was für Fehler können vorkommen?
        \item Habe ich grob die Idee der Abschätzung für die Fehlerwahrscheinlichkeit verstanden?
        \item Was ändert sich, wenn der randomisierte Algorithmus wiederholt wird?
    \end{enumerate}
\end{enumerate}

\section{SW2: Modellierung und MAX-SAT}

\begin{enumerate}[label=(\alph*)]
    \item Modellierung randomisierter Algorithmen
    \begin{enumerate}[label=(\roman*)]
        \item Was ist der Unterschied zwischen stochastischen und randomisierten Algorithmen?
        \item Welche zwei Modelle für die Randomisierung gibt es?
        \item Was ist die erwartete Zeit?
        \item Was ist die Erfolgs- bzw. Fehlerwahrscheinlichkeit?
    \end{enumerate}
    \item MAX-SAT
    \begin{enumerate}[label=(\roman*)]
        \item Kann ich das Problem erklären?
        \item Wie funktioniert der Algorithmus STICH?
        \item Nach welchem Modell ist STICH aufgebaut?
        \item Was ist die erwartete Zeit und die erwartete Anzahl wahrere Klauseln von STICH?
    \end{enumerate}
    \item Algorithmus RQS
    \begin{enumerate}[label=(\roman*)]
        \item Wie funktioniert der Algorithmus?
        \item Nach welchem Modell ist RQS aufgebaut?
        \item Was ist die erwartete Anzahl Vergleiche von STICH?
        \item Warum ist RQS besser als der analoge deterministische Algorithmus QS?
    \end{enumerate}
\end{enumerate}

\section{SW3: Anwendungsprobleme, LAS-VEGAS}

\begin{enumerate}[label=(\alph*)]
    \item Anwendungsprobleme
    \begin{enumerate}[label=(\roman*)]
        \item Für welche zwei typischen Probleme werden randomisierte Algorithmen entwickelt?
        \item Was bedeuten falsche Berechnungen?
    \end{enumerate}
    \item LAS-VEGAS-Algorithmen
    \begin{enumerate}[label=(\roman*)]
        \item Was sind LAS-VEGAS-Algorithmen?
        \item Was ist der Unterschied der Definitionen Version 1 und Version 2?
    \end{enumerate}
    \item \textbf{Beispiel Kommunikationsprotokoll}: Kenne ich die vier Phasen des LAS-VEGAS-Verfahrens nach Version 2?
    \item \textbf{Version 1 versus Version 2}: Wie entsteht aus einem LAS-VEGAS-Algorithmus der Version 2 ein LAS-VEGAS-Algorithmus der Version 1 und umgekehrt?
\end{enumerate}

\section{SW4: 1MC-, 2MC, MC-Algorithmen}

\begin{enumerate}[label=(\alph*)]
    \item 1MC-Algorithmen:
    \begin{enumerate}[label=(\roman*)]
        \item Was ist ein Entscheidungsproblem?
        \item Wie wird ein 1MC-Algorithmus für Entschiedungsprobleme definiert?
        \item Was haben 1MC*-Algorithmen für eine zusätzliche Eigenschaft?
        \item Weiss ich, wie 1MC-Algorithmen mit n Wiederholungen verwendet werden?
        \item Was hat diese Wiederholung für einen Einfluss auf die Zeitkomplexität und die Fehlerwahrscheinlichkeit?
    \end{enumerate}
    \item 2MC-Algorithmen:
    \begin{enumerate}[label=(\roman*)]
        \item Wie wird ein 2MC-Algorithmus für Berechnung einer Funktion definiert?
        \item Weiss ich, wie 2MC-Algorithmen mit n Wiederholungen verwendet werden?
    \end{enumerate}
    \item MC-Algorithmen:
    \begin{enumerate}[label=(\roman*)]
        \item Wie wird ein MC-Algorithmus für Berechnung einer Funktion definiert?
        \item Was kann bei MC-Algorithmen das Problem sein?
        \item Kann ich das Beispiel (Kommunikationsprotokoll mit zwei Phasen) erklären?
    \end{enumerate}
\end{enumerate}

\section{SW5: Optimierungsprobleme}

\begin{enumerate}[label=(\alph*)]
    \item Optimierungsprobleme:
    \begin{enumerate}[label=(\roman*)]
        \item Wie wird ein randomisierter Algorithmus für ein Optimierungsproblem durch Wiederholungen eingesetzt?
        \item Was können wir über die Wahrscheinlichkeit für die optimale Lösung sagen?
        \item Welche Eigenschaft soll eine zulässige Lösung haben?
        \item Wie wird ein Optimierungsproblem allgemein beschrieben?
    \end{enumerate}
    \newpage
    \item Beispiele:
    \begin{enumerate}[label=(\roman*)]
        \item Kann ich das TSP-Problem erklären?
        \item Kann ich das MIN-VCP-Problem erklären?
        \item Kann ich das MAX-SAT-Problem erklären?
        \item Kann ich erklären, was IPL-Probleme sind?
    \end{enumerate}
\end{enumerate}

\section{SW6: Eigenschaften von Algorithmen für Optimierungsprobleme}

\begin{enumerate}[label=(\alph*)]
    \item Eigenschaften von Algorithmen für Opimtierungsprobleme:
    \begin{enumerate}[label=(\roman*)]
        \item Wann ist ein Algorithmus für ein Optimierungsproblem zulässig?
        \item Was ist die Approximationsgüte?
        \item Was ist ein $\delta$-Approximations-Algorithmus und was bedeutet diese Eigenschaft konkret für ein Maximums- bzw. Minimumgsproblem?
        \item Kann ich den VAC-Algorithmus für das MIN-VCP-Problem erklären?
    \end{enumerate}
    \item Eigenschaften für randomisierte Algorithmen:
    \begin{enumerate}[label=(\roman*)]
        \item Was ist ein randomisierter $E[\delta]$-Approximations-Algorithmus?
        \item Was ist ein randomisierter $\delta$-Approximations-Algorithmus?
        \item Wie unterscheiden sich die zwei Arten von randomisierten Approximations-Algorithmen?
    \end{enumerate}
\end{enumerate}

\newpage
\section{SW7: Randomisierte Approximations-Algorithmen}

\begin{enumerate}[label=(\alph*)]
    \item Randomisierte Approximations-Algorithmen:
    \begin{enumerate}[label=(\roman*)]
        \item Wie muss $\delta$ angepasst werden, damit ein randomisierter $E[\delta]$-Approximations-Algorithmus die Eigenschaft eines randomisierten $\delta$*-Approximations-Algorithmus erfüllt?
        \item Was sagt die Markov-Ungleichung?
    \end{enumerate}
    \item Beispiel:
    \begin{enumerate}[label=(\roman*)]
        \item Welche Approximations-Eigenschaft hat der STICH Algorithmus für das MAX-Ek-SAT-Problem?
    \end{enumerate}
\end{enumerate}

\section{SW8: Paradigmen, Überlisten des Gegners, Online-Probleme}

\begin{enumerate}[label=(\alph*)]
    \item Paradigmen:
    \begin{enumerate}[label=(\roman*)]
        \item Welche Paradigmen für den Entwurf von randomisierten Algorithmen gibt es?
    \end{enumerate}
    \item Überlisten des Gegners:
    \begin{enumerate}[label=(\roman*)]
        \item Kann ich die Idee erklären?
        \item Was ist das Hashing-Problem?
        \item Warum ist Hashing ein Musterbeispiel für das überlisten des Gegners?
        \item Was sind geeignete Hashfunktionen?
    \end{enumerate}
    \item Online-Probleme:
    \begin{enumerate}[label=(\roman*)]
        \item Was ist ein Online Problem?
        \item Kenn ich Beispiele für Online Probleme?
    \end{enumerate}
\end{enumerate}

\newpage
\section{SW9: Online Probleme}

\begin{enumerate}[label=(\alph*)]
    \item Optimierungsprobleme als Online-Problem:
    \begin{enumerate}[label=(\roman*)]
        \item Was ist ein Online Algorithmus?
        \item Was ist ein Offline Algorithmus?
        \item Wie wird die Konkurrenzgüte definiert?
        \item Was ist ein $\delta$-konkurrenzfähiger Algorithmus und was bedeutet diese Eigenschaft konkret für ein Maximums- bzw. Minimumsproblem?
        \item Was ist ein $\delta$-schweres Online Problem?
    \end{enumerate}
    \item Paging:
    \begin{enumerate}[label=(\roman*)]
        \item Was ist das Paging Problem?
        \item Warum ist Paging für einen Cache-Speicher mit $k$ Bit ein $k$-schweres Problem (Beispiel mit $k=3$)?
    \end{enumerate}
    \item Randomisierte Online Algorithmen:
    \begin{enumerate}[label=(\roman*)]
        \item Warum sind Online Probleme Beispiele für das Überlisten des Gegners?
        \item Was ist die Konkurrenzgẗe und die Konkurrenzfähigkeit bei randomisierten Algorithmen?
    \end{enumerate}
    \item Beispiel:
    \begin{enumerate}[label=(\roman*)]
        \item Kann ich das Beispiel zur Arbeitsverteilung erklären?
        \item Verstehe ich die graphische Darstellung mit zwei Aufträgen?
        \item Kenne ich Algorithmus DIAG?
    \end{enumerate}
\end{enumerate}

\newpage
\section{SW10: Fingerabdrücke}

\begin{enumerate}[label=(\alph*)]
    \item Fingerabdrücke:
    \begin{enumerate}[label=(\roman*)]
        \item Kann ich die Idee erklären?
    \end{enumerate}
    \item Beispiel:
    \begin{enumerate}[label=(\roman*)]
        \item Kenne ich die Algorithmen PSet, PSchnitt und d-R für die Kommunikationsprotokolle und ihre Eigenschaften?
        \item Kenne ich den Algorithmus STRING für das Teilstring Problem und seine Eigenschaften?
        \item Kenne ich den Algorithmus FREIVALDS für die Verifikation der Matrizenmultiplikation und seine Eigenschaften?
    \end{enumerate}
\end{enumerate}

\section{SW11: Wahrscheinlichkeitsverstärkung}

\begin{enumerate}[label=(\alph*)]
    \item Beispiel:
    \begin{enumerate}[label=(\roman*)]
        \item Kenne ich den Algorithmus AQP für die Äquivalenz zweier Polynome und seine Eigenschaften?
    \end{enumerate}
    \item Wahrscheinlichkeitsverstärkung und Stichproben:
    \begin{enumerate}[label=(\roman*)]
        \item Kann ich die Idee erklären?
    \end{enumerate}
    \item Beispiel:
    \begin{enumerate}[label=(\roman*)]
        \item Kann ich das MIN-CUT Problem erklären?
        \item Kenne ich den Algorithmus KONTRAKTION und seine Eigenschaften?
        \item Warum ist der Algorithmus KONTRAKTION auch durch simples Wiederholen nicht brauchbar?
    \end{enumerate}
\end{enumerate}

\newpage
\section{SW12: Wiederholungen, 3SAT, Zeugen und Primzahl-Test}

\begin{enumerate}[label=(\alph*)]
    \item Gezielte Wiederholungen:
    \begin{enumerate}[label=(\roman*)]
        \item Mit welcher Grundidee versuchen den Algorithmus KONTRAKTION zu verbessern?
        \item Kenne ich die Algorithmen DETRAN und WBAUM und ihre Eigenschaften?
    \end{enumerate}
    \item Das 3SAT Problem:
    \begin{enumerate}[label=(\roman*)]
        \item Warum sind auch randomisierte Algorithmen mit exponentieller Laufzeit interessant?
        \item Kenne ich den Algorithmus SCHÖNING für das 3SAT Problem und seine Eigenschaften?
    \end{enumerate}
    \item Häufige Zeugen:
    \begin{enumerate}[label=(\roman*)]
        \item Kann ich die Idee erklären?
    \end{enumerate}
    \item Primzahl Test:
    \begin{enumerate}[label=(\roman*)]
        \item Was ist ein Primzahl Test?
        \item Kenn ich den Algorithmus NAIV undv erstehe ich, wann er unbrauchbar ist?
    \end{enumerate}
\end{enumerate}

\section{SW13: Wiederholungen, 3SAT, Zeugen und Primzahl-Test}

\begin{enumerate}[label=(\alph*)]
    \item Zeugenkandidaten:
    \begin{enumerate}[label=(\roman*)]
        \item Was sind Zeugenkandidaten und welche Eigenschaften sollen sie haben?
        \item Wie wird der Satz von Fermat für Zeugenkandidaten gebraucht?
        \item Wie werden die Zeugenkandidaten mit dem Satz von Euler bzw. dem Satz von Miller-Rabin optimiert?
        \item Kenne ich den Algorithmus PRIMZAHL und seine Eigenschaften?
    \end{enumerate}
    \newpage
    \item Zufälliges Runden:
    \begin{enumerate}[label=(\roman*)]
        \item Kann ich die Idee erklären?
        \item Wie können wir das MIN-VCP Problem in ein LP Problem übersetzen?
        \item Kann ich das MAX-KP Problem erklären?
        \item Wie können wir das MAX-KP Problem in ein LP Problem übersetzen?
    \end{enumerate}
\end{enumerate}

\section{SW14: MAX-SAT, LP}

\begin{enumerate}[label=(\alph*)]
    \item Beispiel:
    \begin{enumerate}[label=(\roman*)]
        \item Kann ich das MAX-SAT Problem in ein LP Problem übersetzen?
        \item Kenne ich die zwei Algorithmen RZR und KOMB und ihre Eigenschaften?
    \end{enumerate}
\end{enumerate}