% !TeX program = tectonic
\documentclass[%
  paper=a4,
  fontsize=11pt,
  BCOR=8mm,
  DIV=12,
  headings=normal,1
  parskip=half,
  numbers=noendperiod
]{scrreprt}

% Fonts & language: Tectonic uses (Xe)TeX engine; use fontspec + polyglossia
\usepackage{fontspec}
\usepackage{polyglossia}
\setmainlanguage{german}
\setotherlanguage{english}

% Choose readable fonts
\setmainfont{TeX Gyre Pagella}
\setsansfont{TeX Gyre Heros}
\setmonofont{TeX Gyre Cursor}

% Encoding & micro-typography
\usepackage{microtype}

% Graphics, tables, math (lightweight defaults)
\usepackage{graphicx}
\usepackage{booktabs}
\usepackage{siunitx}
\usepackage{amssymb}
\usepackage{amsmath}
\usepackage{enumitem}
\usepackage{multicol}

\usepackage{pifont}
\sisetup{locale=DE, detect-all}

% Bibliography with biblatex + biber
\usepackage[
  backend=biber,
  style=ieee
]{biblatex}
\addbibresource{bib/references.bib}

% Load hyperlink setup
\usepackage{xcolor}
\usepackage{hyperref}
\hypersetup{
    colorlinks=true,
    linkcolor=blue,
    citecolor=black,
    filecolor=magenta,      
    urlcolor=blue,
    pdftitle={Implementierung eines automatisierten Löschprozesses für kategorisierte Daten},
    pdfauthor={Eldar Omerovic},
    pdfsubject={Bachelorarbeit – Hochschule Luzern, Informatik},
    pdfkeywords={Datenlöschung, Audit Logging, Scheduling, .NET, DSGVO},
    pdfpagemode=FullScreen,
}
\urlstyle{same}

% Redefine the url command for the TOC to remove styling
\AtBeginDocument{
  \addtocontents{toc}{\protect\hypersetup{linkcolor=.,citecolor=black,urlcolor=.}}
}

% Load project preamble/macros
% Figure paths
\graphicspath{{figures/}}

% Klickbar: kompletter Eintrag (Text + Seitenzahl)
\hypersetup{linktoc=all}

% Hierarchie mit Einrückungen
\KOMAoptions{toc=graduated}

% Dichtes Punktmuster für die Leader
\makeatletter
\renewcommand*\TOCLineLeaderFill{\leaders\hbox to .6em{\hss.\hss}\hfill}
\makeatother

% No indentation, space between paragraphs
\setlength{\parindent}{0pt}
\setlength{\parskip}{\baselineskip}

\usepackage{csquotes}
\setmainlanguage{german}

\usepackage{listings}
\usepackage{xcolor}

\usepackage{upquote}
\usepackage{microtype}

\usepackage{float}

% --------------------
% Fonts & Typografie
% --------------------
\usepackage[T1]{fontenc}
\usepackage{lmodern}
\usepackage{microtype}

\usepackage{setspace}
\setstretch{1}

% Table that wraps over pages
\usepackage{longtable}
\usepackage{paracol}
\usepackage{array}

% table of contents

\usepackage{tocloft}
\renewcommand{\listfigurename}{Abbildungsverzeichnis}
\renewcommand{\listtablename}{Tabellenverzeichnis}
\renewcommand{\lstlistlistingname}{Quellcodeverzeichnis}

\makeatletter
\newcommand{\onlylof}{\@starttoc{lof}}
\newcommand{\onlylot}{\@starttoc{lot}}
\newcommand{\onlylol}{\@starttoc{lol}} % List of Listings (listings)
\makeatother

\usepackage{seqsplit}

\newcolumntype{P}[1]{>{\raggedright\arraybackslash\sloppy}p{#1}}
\newcommand{\cw}[1]{\seqsplit{#1}} % code-wrap helper

\usepackage{tabularx}
\usepackage{pdfpages}
\lstdefinestyle{defaultcode}{
  basicstyle=\ttfamily\small,
  backgroundcolor=\color{gray!5},
  frame=single,
  rulecolor=\color{gray!40},
  breaklines=true,
  breakatwhitespace=true,
  tabsize=2,
  numbers=left,
  numberstyle=\tiny\color{gray!70},
  numbersep=8pt,
  captionpos=b,
  showstringspaces=false,
  columns=fullflexible,
  keepspaces=true,
  upquote=true
}

\lstdefinelanguage{json}{
  basicstyle=\ttfamily\small,
  morestring=[b]",
  stringstyle=\color{black},
  showstringspaces=false,
  breaklines=true,
  literate=
   *{0}{{0}}{1}
    {1}{{1}}{1}
    {2}{{2}}{1}
    {3}{{3}}{1}
    {4}{{4}}{1}
    {5}{{5}}{1}
    {6}{{6}}{1}
    {7}{{7}}{1}
    {8}{{8}}{1}
    {9}{{9}}{1}
    {:}{{:}}{1}
    {,}{{,}}{1}
    {\{}{{\{}}{1}
    {\}}{{\}}}{1}
    {[}{{[}}{1}
    {]}{{]}}{1}
}

\lstdefinestyle{codejson}{
  style=defaultcode,
  language=json
}

\lstdefinestyle{codesql}{
  style=defaultcode,
  language=SQL,
  morekeywords={IDENTITY,UNIQUEIDENTIFIER,NVARCHAR,DATETIME2,GETUTCDATE,NEWID},
}

\lstdefinestyle{codecsharp}{
  style=defaultcode,
  language=[Sharp]C,
  morekeywords={async,await,var,record,init,Task,ValueTask,IServiceCollection,ILogger},
}

\lstdefinestyle{coderazor}{
  style=defaultcode,
  language=HTML,
}

\lstset{style=defaultcode}
% Custom commands for your thesis
\newcommand{\wipro}{WIPRO}
\newcommand{\todo}[1]{\textbf{TODO:} #1}

\newcommand{\img}[3][]{%
  \centering
  \includegraphics[#1]{#2}
  \caption{#3}
}

% Title meta
\title{Entwurf und Analyse randomisierter Verfahren}
\subtitle{RANDANDALG HS25}
\author{Eldar Omerovic}
\date{\today}

\begin{document}

\thispagestyle{empty}
\begin{center}
    
    \begin{flushleft}
        \includegraphics[width=4cm]{titlepage/HSLU_Logo.png}
    \end{flushleft}

    \vspace{5cm}

    \noindent\makebox[\linewidth]{\rule{\textwidth}{1pt}} 
    
    \begin{center}
        \Large \textbf{3D Modellieren für Echtzeitanwendungen}\\[5pt]
        \large 3DMOD4RT von Eldar Omerovic\\
    \end{center}
    
    \noindent\makebox[\linewidth]{\rule{\textwidth}{1pt}} 
    
    \vspace{1cm}
    
    \begin{center}
    
    \textbf{Autor}\\
    \textit{Eldar Omerovic}
    
    \vspace{1cm}
    
    \textbf{Semester}\\
    \textit{HS25}
    
    \vspace{1cm}
    
    \textbf{Über diese Zusammenfassung}\\
    \textit{Diese Zusammenfassung wurde zur Prüfungsvorbereitung im HS25 erstellt.}
    
    \end{center}
    
    \vfill
    \begin{flushright}
        Januar 2026
    \end{flushright}
\end{center}

\newpage

\pagenumbering{Roman}
\setcounter{page}{1}  

\newpage
\vspace*{0.1\textheight}

\begin{center}
  {\large\bfseries Abstract\par}

  \vspace{0.5cm}

  Diese Zusammenfassung wurde nach dem Lerngerüst des Dozenten aufgebaut. Es beinhaltet die wichtigsten Informationen, um sich auf die Prüfung des HS25 im Modul RANDANDALG vorzubereiten.

  \vspace{0.5cm}

  Die Kapitel sind so aufgebaut, dass ein Kapitel eine Schulwoche an Stoff abdeckt. In einem ersten Schritt wird im Kapitel alles erklärt und zusammengefasst dargestellt. Zu letzt pro Kapitel werden die Fragen 1:1 wie sie im Lerngerüst stehen beantwortet.

\end{center}

\newpage

\tableofcontents

\clearpage
\pagenumbering{arabic}
\setcounter{page}{1}

\chapter{Lerngerüst}
In diesem Kapitel befindet sich das Lerngerüst.

\section{SW1: Einleitung}

\begin{enumerate}[label=(\alph*)]
    \item Für welchen Zweck wird Zufall in der Informatik eingesetzt?
    \item Beispiel von SW1 zur Motivation
    \begin{enumerate}[label=(\roman*)]
        \item Kann ich das Problem erklären?
        \item Was ist die beste Kommunikationskomplexität deterministischer Algorithmen?
        \item Kenne ich den randomisierten Algorithmus?
        \item Was für Fehler können vorkommen?
        \item Habe ich grob die Idee der Abschätzung für die Fehlerwahrscheinlichkeit verstanden?
        \item Was ändert sich, wenn der randomisierte Algorithmus wiederholt wird?
    \end{enumerate}
\end{enumerate}

\section{SW2: Modellierung und MAX-SAT}

\begin{enumerate}[label=(\alph*)]
    \item Modellierung randomisierter Algorithmen
    \begin{enumerate}[label=(\roman*)]
        \item Was ist der Unterschied zwischen stochastischen und randomisierten Algorithmen?
        \item Welche zwei Modelle für die Randomisierung gibt es?
        \item Was ist die erwartete Zeit?
        \item Was ist die Erfolgs- bzw. Fehlerwahrscheinlichkeit?
    \end{enumerate}
    \item MAX-SAT
    \begin{enumerate}[label=(\roman*)]
        \item Kann ich das Problem erklären?
        \item Wie funktioniert der Algorithmus STICH?
        \item Nach welchem Modell ist STICH aufgebaut?
        \item Was ist die erwartete Zeit und die erwartete Anzahl wahrere Klauseln von STICH?
    \end{enumerate}
    \item Algorithmus RQS
    \begin{enumerate}[label=(\roman*)]
        \item Wie funktioniert der Algorithmus?
        \item Nach welchem Modell ist RQS aufgebaut?
        \item Was ist die erwartete Anzahl Vergleiche von STICH?
        \item Warum ist RQS besser als der analoge deterministische Algorithmus QS?
    \end{enumerate}
\end{enumerate}

\section{SW3: Anwendungsprobleme, LAS-VEGAS}

\begin{enumerate}[label=(\alph*)]
    \item Anwendungsprobleme
    \begin{enumerate}[label=(\roman*)]
        \item Für welche zwei typischen Probleme werden randomisierte Algorithmen entwickelt?
        \item Was bedeuten falsche Berechnungen?
    \end{enumerate}
    \item LAS-VEGAS-Algorithmen
    \begin{enumerate}[label=(\roman*)]
        \item Was sind LAS-VEGAS-Algorithmen?
        \item Was ist der Unterschied der Definitionen Version 1 und Version 2?
    \end{enumerate}
    \item \textbf{Beispiel Kommunikationsprotokoll}: Kenne ich die vier Phasen des LAS-VEGAS-Verfahrens nach Version 2?
    \item \textbf{Version 1 versus Version 2}: Wie entsteht aus einem LAS-VEGAS-Algorithmus der Version 2 ein LAS-VEGAS-Algorithmus der Version 1 und umgekehrt?
\end{enumerate}

\section{SW4: 1MC-, 2MC, MC-Algorithmen}

\begin{enumerate}[label=(\alph*)]
    \item 1MC-Algorithmen:
    \begin{enumerate}[label=(\roman*)]
        \item Was ist ein Entscheidungsproblem?
        \item Wie wird ein 1MC-Algorithmus für Entschiedungsprobleme definiert?
        \item Was haben 1MC*-Algorithmen für eine zusätzliche Eigenschaft?
        \item Weiss ich, wie 1MC-Algorithmen mit n Wiederholungen verwendet werden?
        \item Was hat diese Wiederholung für einen Einfluss auf die Zeitkomplexität und die Fehlerwahrscheinlichkeit?
    \end{enumerate}
    \item 2MC-Algorithmen:
    \begin{enumerate}[label=(\roman*)]
        \item Wie wird ein 2MC-Algorithmus für Berechnung einer Funktion definiert?
        \item Weiss ich, wie 2MC-Algorithmen mit n Wiederholungen verwendet werden?
    \end{enumerate}
    \item MC-Algorithmen:
    \begin{enumerate}[label=(\roman*)]
        \item Wie wird ein MC-Algorithmus für Berechnung einer Funktion definiert?
        \item Was kann bei MC-Algorithmen das Problem sein?
        \item Kann ich das Beispiel (Kommunikationsprotokoll mit zwei Phasen) erklären?
    \end{enumerate}
\end{enumerate}

\section{SW5: Optimierungsprobleme}

\begin{enumerate}[label=(\alph*)]
    \item Optimierungsprobleme:
    \begin{enumerate}[label=(\roman*)]
        \item Wie wird ein randomisierter Algorithmus für ein Optimierungsproblem durch Wiederholungen eingesetzt?
        \item Was können wir über die Wahrscheinlichkeit für die optimale Lösung sagen?
        \item Welche Eigenschaft soll eine zulässige Lösung haben?
        \item Wie wird ein Optimierungsproblem allgemein beschrieben?
    \end{enumerate}
    \newpage
    \item Beispiele:
    \begin{enumerate}[label=(\roman*)]
        \item Kann ich das TSP-Problem erklären?
        \item Kann ich das MIN-VCP-Problem erklären?
        \item Kann ich das MAX-SAT-Problem erklären?
        \item Kann ich erklären, was IPL-Probleme sind?
    \end{enumerate}
\end{enumerate}

\section{SW6: Eigenschaften von Algorithmen für Optimierungsprobleme}

\begin{enumerate}[label=(\alph*)]
    \item Eigenschaften von Algorithmen für Opimtierungsprobleme:
    \begin{enumerate}[label=(\roman*)]
        \item Wann ist ein Algorithmus für ein Optimierungsproblem zulässig?
        \item Was ist die Approximationsgüte?
        \item Was ist ein $\delta$-Approximations-Algorithmus und was bedeutet diese Eigenschaft konkret für ein Maximums- bzw. Minimumgsproblem?
        \item Kann ich den VAC-Algorithmus für das MIN-VCP-Problem erklären?
    \end{enumerate}
    \item Eigenschaften für randomisierte Algorithmen:
    \begin{enumerate}[label=(\roman*)]
        \item Was ist ein randomisierter $E[\delta]$-Approximations-Algorithmus?
        \item Was ist ein randomisierter $\delta$-Approximations-Algorithmus?
        \item Wie unterscheiden sich die zwei Arten von randomisierten Approximations-Algorithmen?
    \end{enumerate}
\end{enumerate}

\newpage
\section{SW7: Randomisierte Approximations-Algorithmen}

\begin{enumerate}[label=(\alph*)]
    \item Randomisierte Approximations-Algorithmen:
    \begin{enumerate}[label=(\roman*)]
        \item Wie muss $\delta$ angepasst werden, damit ein randomisierter $E[\delta]$-Approximations-Algorithmus die Eigenschaft eines randomisierten $\delta$*-Approximations-Algorithmus erfüllt?
        \item Was sagt die Markov-Ungleichung?
    \end{enumerate}
    \item Beispiel:
    \begin{enumerate}[label=(\roman*)]
        \item Welche Approximations-Eigenschaft hat der STICH Algorithmus für das MAX-Ek-SAT-Problem?
    \end{enumerate}
\end{enumerate}

\section{SW8: Paradigmen, Überlisten des Gegners, Online-Probleme}

\begin{enumerate}[label=(\alph*)]
    \item Paradigmen:
    \begin{enumerate}[label=(\roman*)]
        \item Welche Paradigmen für den Entwurf von randomisierten Algorithmen gibt es?
    \end{enumerate}
    \item Überlisten des Gegners:
    \begin{enumerate}[label=(\roman*)]
        \item Kann ich die Idee erklären?
        \item Was ist das Hashing-Problem?
        \item Warum ist Hashing ein Musterbeispiel für das überlisten des Gegners?
        \item Was sind geeignete Hashfunktionen?
    \end{enumerate}
    \item Online-Probleme:
    \begin{enumerate}[label=(\roman*)]
        \item Was ist ein Online Problem?
        \item Kenn ich Beispiele für Online Probleme?
    \end{enumerate}
\end{enumerate}

\newpage
\section{SW9: Online Probleme}

\begin{enumerate}[label=(\alph*)]
    \item Optimierungsprobleme als Online-Problem:
    \begin{enumerate}[label=(\roman*)]
        \item Was ist ein Online Algorithmus?
        \item Was ist ein Offline Algorithmus?
        \item Wie wird die Konkurrenzgüte definiert?
        \item Was ist ein $\delta$-konkurrenzfähiger Algorithmus und was bedeutet diese Eigenschaft konkret für ein Maximums- bzw. Minimumsproblem?
        \item Was ist ein $\delta$-schweres Online Problem?
    \end{enumerate}
    \item Paging:
    \begin{enumerate}[label=(\roman*)]
        \item Was ist das Paging Problem?
        \item Warum ist Paging für einen Cache-Speicher mit $k$ Bit ein $k$-schweres Problem (Beispiel mit $k=3$)?
    \end{enumerate}
    \item Randomisierte Online Algorithmen:
    \begin{enumerate}[label=(\roman*)]
        \item Warum sind Online Probleme Beispiele für das Überlisten des Gegners?
        \item Was ist die Konkurrenzgẗe und die Konkurrenzfähigkeit bei randomisierten Algorithmen?
    \end{enumerate}
    \item Beispiel:
    \begin{enumerate}[label=(\roman*)]
        \item Kann ich das Beispiel zur Arbeitsverteilung erklären?
        \item Verstehe ich die graphische Darstellung mit zwei Aufträgen?
        \item Kenne ich Algorithmus DIAG?
    \end{enumerate}
\end{enumerate}

\newpage
\section{SW10: Fingerabdrücke}

\begin{enumerate}[label=(\alph*)]
    \item Fingerabdrücke:
    \begin{enumerate}[label=(\roman*)]
        \item Kann ich die Idee erklären?
    \end{enumerate}
    \item Beispiel:
    \begin{enumerate}[label=(\roman*)]
        \item Kenne ich die Algorithmen PSet, PSchnitt und d-R für die Kommunikationsprotokolle und ihre Eigenschaften?
        \item Kenne ich den Algorithmus STRING für das Teilstring Problem und seine Eigenschaften?
        \item Kenne ich den Algorithmus FREIVALDS für die Verifikation der Matrizenmultiplikation und seine Eigenschaften?
    \end{enumerate}
\end{enumerate}

\section{SW11: Wahrscheinlichkeitsverstärkung}

\begin{enumerate}[label=(\alph*)]
    \item Beispiel:
    \begin{enumerate}[label=(\roman*)]
        \item Kenne ich den Algorithmus AQP für die Äquivalenz zweier Polynome und seine Eigenschaften?
    \end{enumerate}
    \item Wahrscheinlichkeitsverstärkung und Stichproben:
    \begin{enumerate}[label=(\roman*)]
        \item Kann ich die Idee erklären?
    \end{enumerate}
    \item Beispiel:
    \begin{enumerate}[label=(\roman*)]
        \item Kann ich das MIN-CUT Problem erklären?
        \item Kenne ich den Algorithmus KONTRAKTION und seine Eigenschaften?
        \item Warum ist der Algorithmus KONTRAKTION auch durch simples Wiederholen nicht brauchbar?
    \end{enumerate}
\end{enumerate}

\newpage
\section{SW12: Wiederholungen, 3SAT, Zeugen und Primzahl-Test}

\begin{enumerate}[label=(\alph*)]
    \item Gezielte Wiederholungen:
    \begin{enumerate}[label=(\roman*)]
        \item Mit welcher Grundidee versuchen den Algorithmus KONTRAKTION zu verbessern?
        \item Kenne ich die Algorithmen DETRAN und WBAUM und ihre Eigenschaften?
    \end{enumerate}
    \item Das 3SAT Problem:
    \begin{enumerate}[label=(\roman*)]
        \item Warum sind auch randomisierte Algorithmen mit exponentieller Laufzeit interessant?
        \item Kenne ich den Algorithmus SCHÖNING für das 3SAT Problem und seine Eigenschaften?
    \end{enumerate}
    \item Häufige Zeugen:
    \begin{enumerate}[label=(\roman*)]
        \item Kann ich die Idee erklären?
    \end{enumerate}
    \item Primzahl Test:
    \begin{enumerate}[label=(\roman*)]
        \item Was ist ein Primzahl Test?
        \item Kenn ich den Algorithmus NAIV undv erstehe ich, wann er unbrauchbar ist?
    \end{enumerate}
\end{enumerate}

\section{SW13: Wiederholungen, 3SAT, Zeugen und Primzahl-Test}

\begin{enumerate}[label=(\alph*)]
    \item Zeugenkandidaten:
    \begin{enumerate}[label=(\roman*)]
        \item Was sind Zeugenkandidaten und welche Eigenschaften sollen sie haben?
        \item Wie wird der Satz von Fermat für Zeugenkandidaten gebraucht?
        \item Wie werden die Zeugenkandidaten mit dem Satz von Euler bzw. dem Satz von Miller-Rabin optimiert?
        \item Kenne ich den Algorithmus PRIMZAHL und seine Eigenschaften?
    \end{enumerate}
    \newpage
    \item Zufälliges Runden:
    \begin{enumerate}[label=(\roman*)]
        \item Kann ich die Idee erklären?
        \item Wie können wir das MIN-VCP Problem in ein LP Problem übersetzen?
        \item Kann ich das MAX-KP Problem erklären?
        \item Wie können wir das MAX-KP Problem in ein LP Problem übersetzen?
    \end{enumerate}
\end{enumerate}

\section{SW14: MAX-SAT, LP}

\begin{enumerate}[label=(\alph*)]
    \item Beispiel:
    \begin{enumerate}[label=(\roman*)]
        \item Kann ich das MAX-SAT Problem in ein LP Problem übersetzen?
        \item Kenne ich die zwei Algorithmen RZR und KOMB und ihre Eigenschaften?
    \end{enumerate}
\end{enumerate}

\chapter{Einleitung}
Gibt es Zufall?

\vspace{0.25cm}

Ob es wahren Zufall gibt ist eine philosophische Frage. Für uns in der Informatik können wir Zufall als Quelle zur Effizienz nutzen!

\section{Ein Beispiel zur Motivation}

\begin{figure}[H]
    \img[width=1\textwidth]{figures/01/beispiel_rechner.png}
        {Beispiel mit Rechnern}
    \label{fig:rechnerbeispiel}
\end{figure}

Das Szenario

\begin{itemize}
    \item Zwei Rechner, $R_{I}$ und $R_{II}$, an entfernten Orten.
    \item Jeder besitzt eine riesige Datenbank, dargestellt als Bitstring $x$ (bei $R_{I}$) und $y$ (bei $R_{II}$).
    \item Aufgabe: Überprüfe, ob die Datenbanken identisch sind, d.h. ob $x=y$.
\end{itemize}

Dies ist für kleine Datenbanken einfach. Man überprüft einfach Bit für Bit beide Bitstrings. Die entscheidende Hürde: Die Skalierung. Mit $n = 10^{16}$ Bits entspricht dies etwa 1.11 Petabyte oder über einer Million Terabyte.

\subsection{Der intuitive Ansatz - und sein Scheitern}

\begin{figure}[H]
    \img[width=1\textwidth]{figures/01/det_ansatz_scheitern.jpg}
        {Scheitern des deterministischen Ansatz}
    \label{fig:detansatz_scheitern}
\end{figure}

Die deterministische Lösung:

\begin{itemize}
    \item $R_I$ sendet seine komplette Datenbank $x$ (Länge $n$) an $R_{II}$.
    \item $R_{II}$ vergleicht $x$ Bit für Bit mit seiner eigenen Datenbank $y$.
    \item Ergebnis: $100\%$ige Korrektheit. Kein Fehler möglich.
\end{itemize}

Die Kommunikationskomplexität ist jedoch enorm hoch. Die benötigte Kommunikation beträgt $n$ Bits.

\begin{itemize}
    \item Für $n = 10^{16}$ Bits würde die Übertragung selbst mit modernsten Netzwerken Jahrtausende dauern.
    \item Fazit: Jedes deterministische Protokoll, dass $100\%$ige Sicherheit garantiert, benötigt mindestens $n$ Bits Kommunikation. Die Aufgabe ist somit deterministisch praktisch unlösbar.
\end{itemize}

\clearpage
\subsection{Zufall als strategische Ressource}

Müssen wir $100\%$ige Sicherheit fordern? Was, wenn wir eine winzige, kontrollierbare Unsicherheit für einen gewaltigen Effizienzgewinn in Kauf nehmen?

\begin{figure}[H]
    \img[width=0.8\textwidth]{figures/01/sicherheit_und_effizienz.png}
        {Sicherheit und Effizienz}
    \label{fig:sicherheit_effizienz}
\end{figure}

Die philosophische Grundlage (Zitat aus Hromkovič):

\enquote{Das Gewebe dieser Welt ist aus Notwendigkeit und Zufall gebildet; die Vernunft des Menschen stellt sich zwischen beide und weiss sie zu beherrschen. [\dots] das Zufällige weiss sie zu lenken, zu leiten und zu nutzen.} - Johann Wolfgang von Goethe

\textbf{Die neue Idee}: Statt der gesamten Daten ($x$ und $y$) vergleichen wir einen kleinen, aber charakteristischen \enquote{digitalen Fingerabdruck}.

\subsubsection{Das randomisierte Protokoll}

\begin{enumerate}
    \item \textbf{Schritt 1: Von Bits zu Zahlen}\newline
        Die Bitstrings $x$ und $y$ werden als Ganzzahlen interpretiert: \newline
        $Nummer($x$) = \sum (x_i * 2^{n-i})$
    \item \textbf{Schritt 2: Die zufällige Komponente}\newline
        $R_I$ wählt zufällig eine Primzahl $p$ aus der Menge aller Primzahlen $\leq n^2$ (jede mit gleicher Wahrscheinlichkeit - also \enquote{uniform}).
    \item \textbf{Schritt 3: Der Fingerabdruck}\newline
        $R_I$ berechnet den Rest: $s = \text{Nummer}(x) \mod p$.
    \item \textbf{Schritt 4: Die Kommunikation}\newline
        $R_I$ sendet nur das Paar $(p, s)$ an $R_{II}$.
    \item \textbf{Schritt 5: Die Verifikation}\newline
        $R_{II}$ berechnet seinerseits $q = \text{Nummer}(y) \mod p$. $R_{II}$ prüft, ob $q = s$ ist, und gibt \enquote{gleich} oder \enquote{ungleich} aus.
\end{enumerate}

\subsection{Analyse der Kommunikationskomplexität}

\begin{multicols}{2}

Vergleich der beiden Methoden mit $n = 10^{16}$ Bits.

\begin{itemize}
    \item Gesendet werden zwei Zahlen: $p$ und $s$.
    \item Das $s < p \leq n^2$, können beide mit ca. $log_2(n^2)$ Bits dargestellt werden.
    \item Die Formel vereinfacht sich zu $log_2(n^2) = 2 \cdot log_2(n)$.
    \item Die Gesamtkommunikation (für $p$ und $s$) beträgt also ca. $4 \cdot log_2(n)$ Bits.
\end{itemize}

\begin{figure}[H]
    \img[width=0.4\textwidth]{figures/01/vergleich_kommunikation.png}
        {Vergleich Kommunikation}
    \label{fig:vergleich_kommunikation}
\end{figure}

\end{multicols}

\subsection{Die Möglichkeit eines Fehlers}

\subsubsection{Fall 1: Die Daten sind identisch}

Für den Fall, dass beide Bitstrings identisch sind: $x=y$:

\begin{itemize}
    \item $\text{Nummer}(x) = \text{Nummer}(y)$
    \item Daher gilt: $\text{Nummer}(x) \mod p = \text{Nummer}(y) \mod p$ für jede Primzahl $p$.
\end{itemize}

Ergebnis: Kein Fehler möglich. Das Protokoll antwortet immer korrekt \enquote{gleich}. Man spricht von einem \enquote{einseitigen Fehler} (1MC-Algorithmus), weil in Fall 2 ein Fehler auftreten kann.

\subsubsection{Fall 2: Die Daten sind verschieden}

Der Algorithmus sagt dann richtigerweise \enquote{ungleich}, wenn:

$$
\text{Nummer}(x) \mod p \neq \text{Nummer}(y) \mod p
$$

Wann sagt der Algorithmus trotz verschiedenen Bitstrings \enquote{gleich}?

Der Algorithmus sagt gleich, wenn: 
$$
\text{Nummer}(x) \mod p = \text{Nummer}(y) \mod p
$$

Wann kann dies passieren?

Da $x \neq y$, wäre diese Aussage falsch. Ein Fehler tritt also genau dann auf, wenn trotz verschiedenen Daten gilt:

$$
\text{Nummer}(x) \equiv \text{Nummer}(y) \quad (\bmod \text{ } p)
$$

Wie kann es zu dem kommen?

Aus der Zahlentheorie gilt: 

$$
a \equiv b \quad (\bmod \text{ } p) \iff p \mid (a-b)
$$

Also hier:

$$
p \mid (\text{Nummer}(x) - \text{Nummer}(y))
$$

Setzt man:

$$
w = | \text{Nummer}(x) - \text{Nummer}(y) |
$$

erhält man:

$$
p \mid w
$$

Das bedeutet, ein Fehler passiert nur, wenn die zufällig gewählte Primzahl $p$ ein Primteiler der festen Zahl $w$ (der Differenz von $\text{Nummer}(x)$ und $\text{Nummer}(y)$) ist.

\begin{itemize}
    \item Ist $p \nmid w \rightarrow$ Reste verschieden $\rightarrow$ Algorithmus sagt \enquote{ungleich} (korrekt).
    \item Ist $p \mid w \rightarrow$ Reste gleich $\rightarrow$ Algorithmus sagt \enquote{gleich} (Fehler).
\end{itemize}

\textbf{Kernaussage}: Bei verschiedenen Daten kann der Algorithmus nur dann fälschlich \enquote{gleich} sagen, wenn die zufällig gewählte Primzahl ein Teiler der Differenz der beiden Zahlen ist.

Das ist der einzige mögliche Fehlerfall - deshalb hat der Algorithmus einen \textbf{einseitigen Fehler} und ist somit ein \text{1MC-Algorithmus}.

\clearpage
\subsection{Fehlerwahrscheinlichkeit}

Setup:

$$
w := |\text{Nummer}(x) - \text{Nummer}(y)|
$$

Um das nochmals klar zu machen:

$$
\text{Nummer}(x) \equiv \text{Nummer}(y) \quad (\bmod \text{ } p) \iff p \mid (\text{Nummer}(x) - \text{Nummer}(y)) \iff p \mid w
$$

\subsubsection{Abschätzung}

\begin{itemize}
    \item Da $x,y$ n-Bit-Zahlen sind, gilt $w < 2^n$
    \item Eine Zahl $w < 2^n$ hat höchstens $n-1$ verschiedene Primteiler.
    \item Wählen wir $p$ zufällig unter allen Primzahlen $\leq n^2$, so gibt es davon etwa:\newline
\end{itemize}

Um die möglichen Primzahlen $\leq n^2$ zu approximieren:

$$
\pi(n^2) \approx \frac{n^2}{2 \ln n}
$$

Damit gilt:

$$
P(\text{Fehler}) \leq \frac{n - 1}{\pi(n^2)} \approx \frac{2 \ln n}{n}
$$

Für $n = 10^{16}$ ergibt sich:

$$
P(\text{Fehler}) \lesssim 7.4 \cdot 10^{-15} = 0.000000000000001
$$

also praktisch vernachlässigbar.

\subsection{Wahrscheinlichkeitsverstärkung}

Wird das Protokoll $k$-mal unabhängig mit neuen Primzahlen ausgeführt und nur dann \enquote{gleich} ausgegeben, wenn alle Vergleiche übereinstimmen, so gilt:

$$
P(\text{Fehler nach k Versuchen}) \leq (P(\text{Fehler}))^k
$$

Bereits für $k = 10$ und $n = 10^{16}$: $\quad P(\text{Fehler}) \approx 10^{-142}$

\section{Summary}

\begin{enumerate}[label=(\alph*)]
    \item Für welchen Zweck wird Zufall in der Informatik eingesetzt?\newline
    \textbf{Antwort:} Für die Effizienzsteigerung.

    \item Beispiel von SW1 zur Motivation
    \begin{enumerate}[label=(\roman*)]
        \vspace{-0.2cm}\item Kann ich das Problem erklären?\newline
        \textbf{Antwort:} Das Vergleichen und Austauschen von grossen Datenmengen ist ein Problem, das determnistisch kaum lösbar sind (wenn 100\%ige Korrektheit notwendig).

        \vspace{-0.2cm}\item Was ist die beste Kommunikationskomplexität deterministischer Algorithmen?\newline
        \textbf{Antwort:} n-Bits

        \vspace{-0.2cm}\item Kenne ich den randomisierten Algorithmus?\newline
        \textbf{Antwort:} Der Algorithmus kombiniert die einzelnen Daten in eine grosse Zahl. Durch Modulo-Operationen mit uniform gewählten Primzahlen wird diese verkleinert. Dieser Rest wird dann ausgetauscht und verglichen.

        \vspace{-0.2cm}\item Was für Fehler können vorkommen?\newline
        \textbf{Antwort:} Wenn die Daten nicht gleich sind und eine Primzahl gewählt wird, welche ein Primteiler der Differenz der beiden \enquote{Daten-Nummern} ist, wird der Algorithmus fälschlicherweise \enquote{gleich} antworten.

        \vspace{-0.2cm}\item Habe ich grob die Idee der Abschätzung für die Fehlerwahrscheinlichkeit verstanden?\newline
        \textbf{Antwort:} Man versucht zu verstehen, wie viele Primzahlen wir wählen können, und wie viele davon \enquote{schlechte Primzahlen} sind (also $p \mid w$). Die Anzahl der schlechten Primzahlen geteilt durch alle möglichen wählbaren Primzahlen ergibt die Fehlerwahrscheinlichkeit.

        \vspace{-0.2cm}\item Was ändert sich, wenn der randomisierte Algorithmus wiederholt wird?\newline
        \textbf{Antwort:} Durch das wiederholen des Algorithmus und der Bedingung, dass alle Wiederholungen \enquote{gleich} zurückgeben. Verkleinern wir die Fehlerwahrscheinlichkeit exponentiell ($\text{Fehlerwahrscheinlichkeit}^{\text{Anz. Wiederholungen}}$).
    \end{enumerate}
\end{enumerate}

\chapter{Modellierung und MAX-SAT}
\section{Modelle und Metriken}

Was ist ein \enquote{randomisierter Alogirthmus}?

Der Zufall in einem randomisierten Algorithmus liegt nicht in den Daten, sondern im Algorithmus selbst. Das Ziel ist es eine gute Leistung mit hoher Wahrscheinlichkeit für jede einzelne Eingabe zu erzielen. Der Algorithmus muss in jeder Situation relativ zuverlässig sein.

Was ist der Unterschied zu einem \enquote{stochastischen Algorithmus}?

Ein stochastischer Algorithmus nimmt an, dass die Eingabedaten aus einer bestimmten Wahrscheinlichkeitsverteilung stammen. Der Zufall liegt in den Daten, nicht im Algorithmus.

\subsection{Modell 1: Die Werkzeugkiste}

Ein randomisierter Algorithmus A ist eine Menge von deterministischen Algorithmen \\* ${A_1, A_2, \dots, A_n}$. Zu Beginn der Berechnung wird einmalig zufällig ein $A_i$ ausgewählt und ausgeführt.

Beispiel: Der STICH-Algorithmus für MAX-SAT folgt diesem Modell.

\subsection{Modell 2: Die Weggabelung}

Der Algorithmus trifft während seiner Ausführung zufällige Entscheidungen. Der Berechnungspfad ist nicht vorbestimmt.

Beispiel: Der RQS-Algorithmus (Randomized Quicksort) ist ein klassisches Beispiel hierfür.

\subsection{Metrik 1: Erwartete Zeit}

Nicht die Worst-Case-Laufzeit, sondern der gewichtete Durchschnitt der Laufzeiten über alle möglichen zufälligen Entscheidungen des Algorithmus für eine feste Eingabe.

Für einen Algorithmus A und eine Eingabe $w$ ist die erwartete Zeit definiert als:

$$
\text{Erwartete Zeit}_A(w) = \sum [\text{Wahr}({C_i}) * \text{Zeit}(C_i)]
$$

für alle möglichen Berechnungspfade $C_i$ von A auf Eingabe w.

\subsection{Metrik 2: Erfolgs- / Fehlerwahrscheinlichkeit}

Die Wahrscheinlichkeit, dass der Algorithmus für eine feste Eingabe das korrekte Ergebnis liefert.

Die Erfolgswahrscheinlichkeit ist $\text{Wahr}(A(x) = F(x))$, wobei $F(x)$ das korrekte Ergebnis ist. Die Fehlerwahrscheinlichkeit ist $1 - \text{Wahr}(A(x) = F(x))$.

\section{Das MAX-SAT Problem}

Problembeschreibung: Gegeben ist eine Formel in Konjunktiver Normalform (KNF). Finde eine Belegung der Variablen, die die maximale Anzahl von Klauseln wahr macht (MAX-SATISFIABILITY).

Beispiel:
$$
\phi = (x_1 \lor x_2) \land (\neg x_1 \lor x_3) \land (x_2 \lor \neg x_3) \land (\neg x_1 \lor \neg x_3)
$$

Wenn wir die Variablen wie folgt belegen: $x_1 = 0, x_2 = 1, x_3 = 0$, können wir alle 4 Klauseln erfüllen:

\begin{multicols}{2}
    \begin{itemize}
        \item $x_1 \lor x_2$: $0 \lor 1$
        \item $\neg x_1 \lor x_3$: $\neg 0 \lor 0$
    \end{itemize}
    \begin{itemize}
        \item $x_2 \lor \neg x_3$: $1 \lor \neg 0$
        \item $\neg x_1 \lor \neg x_3$: $\neg 0 \lor \neg 0$
    \end{itemize}
\end{multicols}

Warum ist das so schwer?

MAX-SAT ist ein NP-schweres Problem. Eine exakte Lösung erfordert in der Regel exponentiellen Aufwand, da man alle $2^n$ möglichen Belegungen prüfen müsste.

\section{Der STICH-Algorithmus für MAX-SAT}

Die Idee: \enquote{Wir raten einfach}

\begin{multicols}{2}
    Der Algorithmus:

    \begin{enumerate}
        \item Für jede Variable $x_i$ in der Formel:
        \begin{enumerate}
            \item Wirf eine faire Münze
            \item Bei \enquote{Kopf}, setze $x_i = 1$
            \item Bei \enquote{Zahl}, setze $x_i = 0$
        \end{enumerate}
        \item Gib die resultierende Belegung zurück.
    \end{enumerate}

    \begin{figure}[H]
        \img[width=0.4\textwidth]{figures/02/muenzen_werfen.png}
        {Münzen Werfen für STICH}
        \label{fig:muenzen_werfen}
    \end{figure}
\end{multicols}

Verbindung zum Modell: STICH folgt Modell 1. Jede der $2^n$ möglichen Münzwurf-Sequenzen entspricht einem einzigartigen deterministischen Algorithmus. Wir wählen davon zufällig aus.

\begin{figure}[H]
    \img[width=0.9\textwidth]{figures/02/stich_tree.jpg}
    {STICH Baum}
    \label{fig:stich_tree}
\end{figure}

\subsection{Analyse des STICH-Algorithmus}

Erwartete Zeit: Sehr effizient. Die Laufzeit ist linear in der Anzahl der Variablen (also $O(n)$), da für jede Variable nur eine einzige Zufallsentscheidung getroffen wird.

Erwartete Anzahl erfüllter Klauseln: Wie viele Klauseln sind im Durchschnitt erfüllt?

\subsubsection{Analyse einer einzelnen Klausel mit $k$ Literalen}

Die Wahrscheinlichkeit, dass ein bestimmtes Literal falsch ist, beträgt $1/2$. Für eine Klausel mit $k$ Literalen ist die Wahrscheinlichkeit, dass alle Literale falsch sind, also $(1/2)^k$. Also ist die Wahrscheinlichkeit, dass die Klausel $F_i$ wahr ist: $1 - (1/2)^k$.

\subsubsection{Linearität des Erwartungswertes}

Der erwartete Gesamtwert ist die Summe der einzelnen Erwartungswerte.

$$
E[\text{erfüllte Klauseln}] = \sum E[F_i \text{ ist Wahr}] = \sum \left(1 - \left(\frac{1}{2}\right)^{k_i}\right)
$$

Das Ergebnis: Da $k \geq 1$ ist, ist $E[F_i \text{ ist wahr}] \geq 1/2$. Damit ist die erwartete Anzahl wahrer Klauseln $\geq m/2$ (wobei $m$ die Gesamtanzahl der Klauseln ist). Das bedeutet: Im Erwartungswert sind $\geq 50\%$ der Klauseln erfüllt.

\section{Randomized Quicksort (RQS)}

Das Problem des deterministischen Quicksorts:

\begin{multicols}{2}
    \begin{itemize}
        \item Die Effizienz von Quicksort hängt entscheidend von der Wahl des Pivotelements ab.
        \item Ein \enquote{böswilliger Gegner} kann uns eine Eingabe geben, die eine schlechte Pivot-Strategie ausnutzt.
    \end{itemize}
    
    Worst Case Beispiel: Wenn das Pivot immer das erste Element ist und die Eingabe bereits sortiert ist (z.B. $[1, 2, 3, 4]$). Die Laufzeit degradiert zu $O(n^2)$.

    \begin{figure}[H]
        \img[width=0.4\textwidth]{figures/02/quicksort.png}
        {Worst Case Quicksort}
        \label{fig:quicksort_worstcase}
    \end{figure}
\end{multicols}

\clearpage
\subsubsection{Die Idee}

Nimm dem Gegner die Macht, indem du seine Kenntnis über den Algorithmus nutzlos machst.

\begin{multicols}{2}
    \begin{enumerate}
        \item Wenn die Liste mehr als ein Element hat:
        \begin{enumerate}
            \item Wähle ein Element zufällig und gleichverteilt als Pivot.
            \item Teile die restlichen Elemente in zwei Listen auf: kleiner als Pivot und grösser als Pivot.
            \item Wende dies rekursiv auf die beiden Teillisten an.
        \end{enumerate}
    \end{enumerate}

    \begin{figure}[H]
        \img[width=0.5\textwidth]{figures/02/rqs.png}
        {Randomized Quicksort}
        \label{fig:random_quicksort}
    \end{figure}
\end{multicols}

\textbf{Verbindung zum Modell}: RQS folgt Modell 2. Der Zufall wird wiederholt innerhalb des Algorithmus bei jedem rekursiven Aufruf eingesetzt. Jede Berechnung ist ein einzigartiger Pfad durch den Berechnungsbaum.

Siehe nächste Seite für Analyse von RQS.

\clearpage
\subsection{Analyse von RQS}

Die Methode: Wir nutzen wieder die Linearität des Erwartungswertes. Wir zählen nicht rekursiv die Ausführungsschritte, sondern paarweise: Werden die Paare im Verlauf von QuickSort jemals miteinander verglichen? Das ist der Zentrale Trick.

\textbf{Setup}

Seien $s_1, s_2, \dots, s_n$ die Elemente der Eingabe in sortierter Reihenfolge. Definiere eine Zufallsvariable $X_{ij}$ für jedes Paar $(s_i, s_j)$ mit $i < j$. $X_{ij} = 1$, wenn $s_i$ und $s_j$ während der Ausführung von RQS verglichen werden, sonst $X_{ij} = 0$. Gesamtzahl der Vergleiche $T = \sum_{i < j} X_{ij}$.

\textbf{Analyse}

Erwartungswert von $T$: $E[T] = E[\sum_{i \leq j} X_{ij}] = \sum_{i \leq j} E[X_{ij}]$. Dabei ist $E[X_{ij}]$ die Wahrscheinlichkeit $p_{ij}$, dass $s_i$ und $s_j$ verglichen werden.

Wann werden $s_i$ und $s_j$ verglichen? Nur dann, wenn das erste aus der Menge ${s_i, s_{i+1}, \dots, s_j}$ gewählte Pivot-Element entweder $s_i$ oder $s_j$ ist. Die Wahrscheinlichkeit hierfür ist $\frac{2}{j-i+1}$, da wir genau 2 günstige Pivots ($s_i$ oder $s_j$) und insgesamt $j-i+1$ mögliche Pivots haben.

\textbf{Das Ergebnis und Fazit}

Ergebnis: 
$$
E[T] = \sum_{i \le j} \frac{2}{j-i+1} \le 2n \cdot H_n \approx 2n \ln n
$$

wobei $H_n$ die n-te Harmonische Zahl ist.

Fazit: Die erwartete Anzahl der Vergleiche ist $O(n \log n)$ für jede Eingabe. RQS ist immun gegen Worst-Case-Daten.

\textbf{Information: Harmonische Zahlen}

Die n-te Harmonische Zahl $H_n$ ist definiert als die Summe der Kehrwerte der ersten n natürlichen Zahlen:

$$
H_n = 1 + \frac{1}{2} + \frac{1}{3} + \dots + \frac{1}{n}
$$

Wichtige Eigenschaft: Für grosse $n$ nähert sich $H_n$ dem natürlichen Logarithmus von n plus der Euler-Mascheroni-Konstante $\gamma$ (ca. 0.5772).

Für Laufzeitanalysen reicht: $H_n = \Theta(\log n)$

\section{STICH vs. RQS}

\begin{longtable}{|p{3cm}|p{5cm}|p{5cm}|}
    \caption{Kernkonzepte STICH vs RQS}
    \label{tab:stich_vs_rqs} \\

    \hline

    \textbf{Kriterium} & \textbf{STICH Algorithmus} & \textbf{RQS Algorithmus} \\

    \hline

    Problem & MAX-SAT (NP-schweres Optimierungsproblem) & Sortieren (Polynomial lösbares Problem) \\

    \hline

    Ziel des Zufalls & Eine \enquote{ziemlich gute} Lösung für ein schweres Problem im Erwartungswert finden. & Den Worst-Case eines ansonsten guten Algorithmus eliminieren. \enquote{Überlisten des Gegners}. \\

    \hline

    Modell & Modell 1: Einmalige zufällige Wahl einer deterministischen Strategie (einer Belegung) & Modell 2: Fortlaufende zufällige Entscheidungen während der Ausführung (Pivot-Wahl). \\

    \hline

    Analysefokus & Erwartete Qualität der Lösung (erwartete Anzahl erfüllter Klausen). & Erwartete Laufzeit (erwartete Anzahl von Vergleichen). \\

    \hline

    Ergebnis & Erwartungswert $\geq m/2$ erfüllte Klauseln & Erwartete Laufzeit von $O(n log n)$. \\

    \hline

\end{longtable}

\section{Klassifizierung randomisierter Algorithmen}

\subsection{Las Vegas Algorithmen (z.B. RQS)}

Liefern immer das korrekte Ergebnis.

Unsicherheit: Die Laufzeit ist eine Zufallsvariable.

\subsection{Monte Carlo Algorithmen (z.B. STICH-ähnliche Ansätze)}

Die Laufzeit ist oft deterministisch, aber das Ergebnis kann (mit geringer Wahrscheinlichkeit) falsch sein. STICH liefert zwar keine \enquote{falsche} Lösung, aber eine suboptimale. Das entspricht dem Geist von Monte-Carlo-Approximationen entspricht.

\clearpage
\section{Summary}

\begin{enumerate}[label=(\alph*)]
    \item Modellierung randomisierter Algorithmen
    \begin{enumerate}[label=(\roman*)]
        \item Was ist der Unterschied zwischen stochastischen und randomisierten Algorithmen?\newline
        \textbf{Antwort:} Stochastische Algorithmen nehmen an, dass die Eingabedaten aus einer bestimmten Wahrscheinlichkeitsverteilung stammen. Der Zufall liegt in den Daten, nicht im Algorithmus. Randomisierte Algorithmen hingegen verwenden Zufall innerhalb des Algorithmus selbst, unabhängig von der Eingabe.
        
        \item Welche zwei Modelle für die Randomisierung gibt es?\newline
        \textbf{Antwort:} Die Werkzeugkiste (Modell 1) und die Weggabelung (Modell 2).

        \item Was ist die erwartete Zeit?\newline
        \textbf{Antwort:} Der gewichtete Durchschnitt der Laufzeiten über alle möglichen zufälligen Entscheidungen des Algorithmus für eine feste Eingabe.

        \item Was ist die Erfolgs- bzw. Fehlerwahrscheinlichkeit?\newline
        \textbf{Antwort:} Die Wahrscheinlichkeit, dass der Algorithmus für eine feste Eingabe das korrekte Ergebnis liefert bzw. die Wahrscheinlichkeit, dass er ein falsches Ergebnis liefert.
    \end{enumerate}
    \item MAX-SAT
    \begin{enumerate}[label=(\roman*)]
        \item Kann ich das Problem erklären?\newline
        \textbf{Antwort:} Gegeben ist eine Formel in Konjunktiver Normalform (KNF). Finde eine Belegung der Variablen, die die maximale Anzahl von Klauseln wahr macht (MAX-SATISFIABILITY).

        \item Wie funktioniert der Algorithmus STICH?\newline
        \textbf{Antwort:} Für jede Variable wird eine faire Münze geworfen. Bei \enquote{Kopf} wird die Variable auf 1 gesetzt, bei \enquote{Zahl} auf 0. Die resultierende Belegung wird zurückgegeben.

        \item Nach welchem Modell ist STICH aufgebaut?\newline
        \textbf{Antwort:} STICH folgt Modell 1: Die Werkzeugkiste

        \item Was ist die erwartete Zeit und die erwartete Anzahl wahrere Klauseln von STICH?\newline
        \textbf{Antwort:} Die erwartete Zeit ist $O(n)$, da für jede Variable $n$ nur eine einzige Zufallsentscheidung getroffen wird. Die erwartete Anzahl wahrer Klauseln ist mindestens $m/2$, also mindestens 50\% der Klauseln werden im Erwartungswert erfüllt.

    \end{enumerate}
    \clearpage
    \item Algorithmus RQS
    \begin{enumerate}[label=(\roman*)]
        \item Wie funktioniert der Algorithmus?\newline
        \textbf{Antwort:} Wenn die Liste mehr als ein Element hat, wird ein Element zufällig als Pivot gewählt. Die restlichen Elemente werden in zwei Listen aufgeteilt: kleiner als Pivot und grösser als Pivot. Dies wird rekursiv auf die beiden Teillisten angewendet.

        \item Nach welchem Modell ist RQS aufgebaut?\newline
        \textbf{Antwort:} RQS folgt Modell 2: Die Weggabelung.

        \item Was ist die erwartete Anzahl Vergleiche von STICH?\newline
        \textbf{Antwort:} Die erwartete Anzahl der Vergleiche ist $O(n \log n)$ für jede Eingabe.

        \item Warum ist RQS besser als der analoge deterministische Algorithmus QS?\newline
        \textbf{Antwort:} RQS ist immun gegen Worst-Case-Daten, da die Pivot-Wahl zufällig erfolgt und somit nicht von der Eingabe abhängt. Dadurch wird die erwartete Laufzeit von $O(n \log n)$ beibehalten, unabhängig von der Eingabe.

    \end{enumerate}
\end{enumerate}

\chapter{Anwendungsprobleme \& LAS-VEGAS}
\section{Anwendungsprobleme \& \enquote{Falsche Berechnungen}}

Randomisierte Algorithmen werden typischerweise für zwei grosse Problemklassen entworfen:

\subsection{Entschiedungs- \& Funktionsprobleme}

Hier wird eine exakte Ja/Nein-Antwort oder ein Funktionswert berechnet.

\begin{itemize}
    \item Beispiele: Primzahltest, Äquivalenz von Datenbanken (Strings \enquote{x}, \enquote{y}), Verifikation von Matrizenmultiplikation.
    \item \enquote{Falsche Berechnung} bedeutet: Der Algorithmus liefert ein Ergebnis $s$, das nicht dem korrekten Wert $F(x)$ entspricht. (z.B. \enquote{Ja}, obwohl die Antwort \enquote{Nein} ist).
    \item Las-Vegas-Algorithmen: Liefern niemals eine falsche Berechnung in diesem Sinne.
    \item Monte-Carlo-Algorithmen: Erlauben falsche Berechnungen mit einer gewissen kleinen Wahrscheinlichkeit.
\end{itemize}

\subsection{Such- \& Optimierungsprobleme}

Hier wird eine Lösung gesucht, die eine bestimmte (oft optimale) Eigenschaft hat.

\begin{itemize}
    \item MAX-SAT (möglichst viele Klauseln erfüllen), Traveling Salesperson Problem (TSP).
    \item \enquote{Falsche Berechnung} ist hier seltener ein Thema. Meist liefert der Algorithmus eine zulässige, aber nicht-optimale Lösung. Die Qualität (Approximationsgüte) wird bewertet.
\end{itemize}

In diesem Kapitel fokussieren wir uns auf Algorithmen der Problemklasse A - die Las-Vegas-Algorithmen.

\subsection{Das Prinzip der garantierten Korrektheit}

Ein Las-Vegas-Algorithmus ist ein randomisierter Algorithmus, der niemals ein falsches Ergebnis ausgibt. Das Kernprinzip ist:

\begin{itemize}
    \item Ergebnis: Immer korrekt.
    \item Laufzeit: Eine Zufallsvariable. Die Analyse konzentriert sich auf die erwartete Laufzeit: $E[\text{Time}(x)]$.
\end{itemize}

\begin{figure}[H]
    \img[width=0.9\textwidth]{figures/03/las-vegas.png}
    {LAS-VEGAS Prinzip}
    \label{fig:lasvegasprinzip}
\end{figure}

\subsection{Die zwei Versionen von Las-Vegas-Algorithmen}

Die beiden Versionen von Las-Vegas-Algorithmen unterscheiden sich darin, wie sie mit \enquote{unglücklichen} Zufallswahlen umgehen.

Sie nächste Seite für Vergleichstabelle.

\clearpage
\begin{longtable}{|p{3cm}|p{5cm}|p{5cm}|}
    \caption{Vergleich der zwei Versionen von Las-Vegas-Algorithmen}
    \label{tab:versionen_lasvegas} \\

    \hline

    \textbf{Eigenschaft} & \textbf{Version 1 (Die Strikte)} & \textbf{Version 2 (Die Vorsichtige)} \\

    \hline

    Mögliche Ausgaben & Liefert immer das korrekte Ergebnis $F(x)$ & Liefert entweder das korrekte Ergebnis $F(x)$ oder ein spezielles Symbol wie \enquote{?} (\enquote{keine Antwort}) \\

    \hline

    Fehler- wahrscheinlichkeit & 0\%. Es wird niemals ein falsches Ergebnis $y \neq F(x)$ ausgegeben. & Es wird niemals ein falsches Ergebnis $y \neq F(x)$ ausgegeben. \\

    \hline

    Garantie & Die Berechnung endet immer mit dem korrekten Ergebnis. Die erwartete Laufzeit $E[\text{Time}(x)]$ ist endlich. & Wenn ein Ergebnis geliefert wird, ist es korrekt. Die Wahrscheinlichkeit für ein korrektes Ergebnis muss positiv sein, typischerweise $P(A(x) = F(x)) \geq 1/2$. \\

    \hline

    Beispiel & Randomisierter Quicksort (RQS). Sortiert eine Liste immer korrekt, aber die Anzahl der Vergleiche variiert. & Kommunikationsprotokoll: Meldet bei Unsicherheit lieber \enquote{weiss nicht} (?), als fälschlicherweise Gleichheit zu bestätigen. \\

    \hline

\end{longtable}

Merkhilfe:

\begin{itemize}
    \item Version 1: Garantiert ein Ergebnis, aber nicht die Laufzeit eines einzelnen Laufs.
    \item Version 2: Garantiert die Korrektheit des Ergebnisses, falls eines geliefert wird.
\end{itemize}

\clearpage
\subsection{Protokoll $\text{LV}_{10}$: Das 4-Phasen Las-Vegas (Version 2)}

Das Ziel: Gibt es mindestens einen Index $i$ mit $x_i = y_i$?
Wichtig: Das Ziel ist nicht, das alle Paare $x_i = y_i$ sein müssen. Wir wollen schauen, \textbf{ob es mindestens ein Paar gibt, das gleich ist}.

Rechner $R_I$ hat einen String $x$, Rechner $R_{II}$ hat einen String $y$. Beide wollen prüfen, ob $x = y$, ohne die gesamten Strings zu übertragen.

Idee: $R_I$ sendet nur einen kurzen \enquote{Fingerabdruck} von $x$. Bei Übereinstimmung der Fingerabdrücke muss zur Sicherheit der ganze String verifiziert werden, da es eine zufällige Kollision sein könnte.

\textbf{Ausgangslage}

$R_I$ hat 10 Strings $x_1, x_2, \dots, x_{10}$. $R_{II}$ hat auch 10 Strings $y_1, y_2, \dots, y_{10}$.


\textbf{Phase 1 : Zufällige Primzahlwahl}

$R_I$ wählt uniform zufällig $k$ Primzahlen $p_1, p_2, \dots, p_k$ aus der Menge aller Primzahlen $\leq n^2$.

\textbf{Phase 2: Fingerabdrücke von $x$ berechnen}

Für jedes $i = 1, \dots, k$ berechnet $R_I$ den Fingerabdruck:

$$
s_i = \text{Nummer}(x) \mod p_i
$$

$R_I$ sendet an $R_{II}$ die Paare $(p_i, s_i)$ für $i = 1, \dots, k$:

$$
(p_1, s_1), (p_2, s_2), \dots, (p_k, s_k)
$$

\textbf{Phase 3: Vergleich mit $y$}

$R_{II}$ berechnet für jedes $i$:

$$
q_i = \text{Nummer}(y) \mod p_i
$$

$R_{II}$ vergleicht nun alle $s_i$ und $q_i$.

Falls: $s_i \neq q_i$, dann weiss $R_{II}$ mit Sicherheit, dass $x_i \neq y_i$ gilt, und sendet \enquote{verwerfe} (0) an $R_I$.

Sonst: Es gibt mindestens ein $i$, für das $s_i = q_i$ gilt. $R_{II}$ nimmt den kleinsten Index $j$ mit $s_j = q_j$ (Jeder Index würde funktionieren, aber der kleinste damit es eindeutig und reproduzierbar ist). Der kleinste Index $j$ wird mit dem kompletten String $y_j$ an $R_I$ zurückgesendet.

\textbf{Phase 4: Entscheidung}

Falls $R_{II}$ \enquote{ungleich} zurücksendet, gibt $R_I$ ebenfalls \enquote{ungleich} zurück.

Falls kein ungleich kam, erhält $R_I$ den Index $j$ und den String $y_j$. $R_I$ vergleicht $x_j$ mit $y_j$:

$$
x_j \stackrel{?}{=} y_j
$$

Falls ja, gibt $R_I$ \enquote{gleich} zurück. Sonst \enquote{?}.

Warum geben wir im zweiten Fall \enquote{?} zurück und nicht \enquote{ungleich}?

$R_I$ weiss in diesem Fall wirklich nicht, ob für ein $k > j$ vielleicht $x_k = y_k$ gilt. Es könnte ja sein, dass die Fingerabdrücke für $k$ kollidiert sind, also $s_k = q_k$ gilt, obwohl $x_k \neq y_k$. In diesem Fall müsste $R_{II}$ ja eigentlich \enquote{ungleich} zurücksenden. Da $R_I$ dies nicht sicher wissen kann, sendet es lieber \enquote{?} zurück.

$R_I$ rät nicht und gibt keine falsche Antwort zurück. Es gibt nur dann eine definitive Antwort (\enquote{gleich} oder \enquote{ungleich}), wenn es diese mit Sicherheit geben kann. \textbf{Genau das macht es zu einem Las-Vegas-Algorithmus der Version 2}.

\subsection{Beispiele für $\text{LV}_{10}$ (alle 3 Fälle)}

\textbf{Fall 1: $R_{II}$ sendet \enquote{verwerfe}}

$R_I$ hat $x_1, x_2, x_3$: $1011, 1100, 0011$\newline
$R_{II}$ hat $y_1, y_2, y_3$: $1110, 1000, 0101$

Hier gilt für alle $x_i \neq y_i$ und somit $s_i \neq q_i$. $R_{II}$ gibt \enquote{verwerfe} zurück. Somit gibt $R_I$ auch \enquote{ungleich} zurück.

\textbf{Fall 2: $R_{II}$ sendet \enquote{gleich}, String sind identisch}

$R_I$ hat $x_1, x_2, x_3$: $1011, 1100, 0011$\newline
$R_{II}$ hat $y_1, y_2, y_3$: $1011, 1100, 0011$

Hier gilt für alle $x_i = y_i$ und somit $s_i = q_i$. $R_{II}$ sendet den kleinsten Index 1 und den String $y_1$ zurück. $R_I$ vergleicht $x_1$ mit $y_1$ und gibt \enquote{gleich} zurück (da Strings tatsächlich gleich).

\clearpage
\textbf{Fall 3: $R_{II}$ sendet \enquote{gleich}, Strings nicht identisch}

$R_I$ hat $x_1, x_2, x_3$: $1011, 1100, 0011$\newline
$R_{II}$ hat $y_1, y_2, y_3$: $0101, 1110, 0001$

Hier gilt nach der Berechnung der Fingerabdrücke (Kollision):

$$
s_1 = q_1, \quad s_2 \neq q_2, \quad s_3 \neq q_3
$$

Obwohl die Strings insgesamt nicht identisch sind, kollidieren die Fingerabdrücke für Index 1.

Es gibt also ein $j = 1$ mit $s_j = q_j$. $R_{II}$ sendet den Index 1 und den String $y_1$ zurück. $R_I$ vergleicht $x_1$ mit $y_1$ und gibt \enquote{?} zurück. 

Die Strings scheinen nicht gleich zu sein, dennoch hat $R_{II}$ nicht \enquote{ungleich} zurückgesendet, da es ja eine Kollision der Fingerabdrücke gab.

\chapter{1MC-, 2MC-, MC-Algorithmen}
\section{Entscheidungsproblem vs. Funktionsproblem}

\textbf{Entscheidungsproblem}

Ein Entscheidungsproblem ist ein Paar $\Sigma, L$:

\begin{itemize}
    \item Eingabe ist ein Wort $x \in \Sigma^{*}$
    \item Ausgabe ist 1 (Ja), wenn $x \in L$, sonst 0 (Nein).
\end{itemize}

\textbf{Funktionsproblem}

Hier berechnen wir eine Funktion $F(x)$ (nicht nur JA/NEIN), z.B. eine Zahl, ein String, ein Objekt. Genau dafür sind 2MC und MC formuliert.

\section{1MC-Algorithmen}

Intuition: Bei 1MC ist nur eine Fehlerseite erlaubt:

\begin{itemize}
    \item Wenn $x \notin L$: nie ein Fehler (immer korrekt \enquote{Nein}).
    \item Wenn $x \in L$: darf es passieren, dass der Algorithmus \enquote{Nein} sagt (Fehler), aber \enquote{Ja} kommt mit mindestens $1/2$.
\end{itemize}

\textbf{Formale Definition}: Algorithmus $A$ ist 1MC für $L$, wenn:

\begin{enumerate}
    \item Für alle $x \in L: Pr[A(x) = 1] \geq 1/2$
    \item Für alle $y \notin L: Pr[A(y) = 0] = 1$
\end{enumerate}

Was ist zusätzlich bei 1MC*? Ein 1MC*-Algorithmus hat die Extra-Eigenschaft:
\begin{itemize}
    \item Für $x \in L$ konvergiert $Pr[A(x) = 1]$ mit wachsender Eignabelänge gegen 1.
\end{itemize}

\subsection{Wiederholung bei 1MC}

Strategie: $k$-mal unabhängig laufen lassen und am Ende ODER verknüfen:

\begin{itemize}
    \item Gib 1 aus, sobald mindestens ein Lauf 1 liefert.
    \item Gib 0 aus, wenn alle Läufe 0 liefern.
\end{itemize}

Fehleranalyse:

\begin{itemize}
    \item Für $y \notin L$: Fehler bleibt 0 (weil niemals 1 ausgegeben wird).
    \item Für $x \in L$: Fehler passiert nur, wenn alle $k$ Läufe fälschlich 0 liefern:
    $$
    Pr[\text{Fehler}] \leq (1/2)^k = 2^{-k}
    $$
\end{itemize}

Einfluss auf Zeit \& Fehler:
\begin{itemize}
    \item Zeitkomplexität: multpliziert sich linear mit $k$
    \item Fehlerwahrscheinlichkeit: fällt exponentiell in $k$
\end{itemize}

\section{2MC-Algorithmen}

Intuition: \enquote{feste Lücke} (Gap) über 1/2

Bei 2MC darf der Fehler auf beiden Seiten passieren, aber wir haben eine garantierte Trefferzone:
$$
Pr[A(x) = F(x)] \geq 1/2 + c
$$
mit einer festen Konstante $c > 0$, die nicht von $|x|$ abhängt.

\textbf{Formale Definition}

$A$ ist 2MC für eine Funktion $F$, wenn es ein $c$ mit $0 < c \leq 1/2$ gibt, sodass für alle Eingaben $x$:
$$
Pr[A(x) = F(x)] \geq 1/2 + c
$$

\clearpage
\subsection{Wiederholung bei 2MC: Majority Vote}

Da Fehler beidseitig möglich sind, funktioniert bei 2MC nicht das ODER-Kriterium wie bei 1MC. Stattdessen: $t$ unabhängige Läufe und nimm das Resultat, das am häufigsten vorkommt (Mehrheitsentscheid / Majority Vote).

\textbf{Warum hilft das?} Weil der Erwartungs-\enquote{Vorsprung} $c$ bei vielen unabhängigen Läufen stabilisiert wird: die Mehrheit wird mit sehr hoher Wahrscheinlichkeit korrekt.

Typische Schranke: Die Fehlerwahrscheinlichkeit sinkt exponentiell in $t$, z.B. in der Form:
$$
Pr[\text{Fehler nach } t \text{ Läufen}] \leq (1 - 4c^2)^{t/2}
$$
Daraus folgt für gewünschte Fehlerschranke $\delta$:
$$
t \gtrsim \frac{2 \ln(\delta)}{\ln(1 - 4c^2)}
$$
Wichtig ist die Aussage: Wenn $c$ konstant ist, reichen konstant viele Wiederholungen für konstantes $\delta \rightarrow$ bleibt effizient.

\section{MC-Algorithmen}

\textbf{Formale Definition}: Ein Algorithmus $A$ ist MC für eine Funktion $F$, wenn für alle $x$:
$$
Pr[A(x) = F(x)] \geq 1/2
$$
Mehr wird nicht verlangt.

Der entscheidende Unterschied zu 2MC: \enquote{verschwindende Lücke}. Bei 2MC war der Abstand zu $1/2$ durch ein fixes $c$ gesichert. Bei MC darf die Lücke:
$$
c_x := Pr[A(x) = F(x)] - 1/2
$$
mit wachsender Eingabelänge gegen 0 gehen. Genau das ist die \enquote{verschwindende Lücke}.

Warum ist das ein Problem? Mehr Wiederholungen helfen nur, wenn man \enquote{genug Abstand} hat. Wenn $c_x$ winzig ist, brauchen wir extrem viele Läufe, um Mehrheitsentscheidung zuverlässig zu machen (typisch proportional zu $1 / c_x^2$). Wenn $c_x$ z.B. exponentiell klein ist, wird die nötige Wiederholungszahl exponentiell gross. Das ist ineffizient.

\section{Beispiel MC/UMC: Kommunikationsprotokoll mit zwei Phasen}

Idee (UMC-Protokoll):

\begin{itemize}
    \item Zwei Strings $x,y \in \{0, 1\}^n$. Wir wollen entscheiden, ob $x \neq y$ (Sprache $L_{\text{ungleich}}$).
    \item Phase 1: Wähle zufällig eine Position $j$, sende $j$ und $x_j$.
    \item Phase 2: Wenn $x_j \neq y_j$: akzeptiere sicher (weil dann $x \neq y$). Wenn $x_j = y_j$: entscheide mit einer leicht \enquote{biased} Münze.
\end{itemize}

Kernpunkt der Analyse:
\begin{itemize}
    \item Wenn $x = y$, ist die korrekte Antwort \enquote{gleich} (also verwerfen). Das passiert nur mit Wahrscheinlichkeit $> 1/2$, aber der Vorsprung über $1/2$ ist nur etwa in der Grössenordnung $1/n$.
    \item Wenn $x \neq y$, hängt die Erfolgswahrscheinlichkeit davon ab, wie viele Position sich unterscheiden. Im Worst Case (nur 1 Unterschied) ist die Chance, den Unterschied in Phase 1 zu treffen, nur $1/n$, und der Gesamtvorsprung über $1/2$ kann extrem klein werden (typisch Grössenordnung $1/n^2$).
\end{itemize}

\textbf{Take-away}:

Das Protokoll ist MC (immer knapp über $1/2$ korrekt), aber die Lücke wird mit grossem $n$ so klein, dass man exponentiell/astronomisch viele Wiederholungen bräuchte $\rightarrow$ \enquote{formal korrekt, aber praktisch oft nutzlos}.

\clearpage
\begin{enumerate}[label=(\alph*)]
    \item 1MC-Algorithmen:
    \begin{enumerate}[label=(\roman*)]
        \item Was ist ein Entscheidungsproblem?\newline
        \textbf{Antwort}: Ein Problem, bei dem zu einer Eingabe $x$ nur JA/NEIN entschieden wird (formal: Sprache $L \subseteq \Sigma^{*}$). Ausgabe 1 falls $x \in L$, sonst 0.

        \item Wie wird ein 1MC-Algorithmus für Entschiedungsprobleme definiert?\newline
        \textbf{Antwort}: Ein randomisierter Algorithmus $A$ mit einseitigem Fehler. Nur bei einer Antwort darf sich der Algorithmus irren.

        \item Was haben 1MC*-Algorithmen für eine zusätzliche Eigenschaft?\newline
        \textbf{Antwort}: Für $x \in L$ wird die Erfolgswahrscheinlichkeit mit wachsender Eingabelänge immer besser (typisch geht gegen 1).

        \item Weiss ich, wie 1MC-Algorithmen mit n Wiederholungen verwendet werden?\newline
        \textbf{Antwort}: Führe $A$ n-mal unabhängig aus und gib JA aus, sobald ein Lauf JA liefert (ODER-Regel). Sonst NEIN.

        \item Was hat diese Wiederholung für einen Einfluss auf die Zeitkomplexität und die Fehlerwahrscheinlichkeit?\newline
        \textbf{Antwort}: (1) Zeit wird $n$-mal so gross (linear in $n$). (2) Fehler nur möglich für z.B. $x \in L$, dann $Pr[\text{Fehler}] \leq (1/2)^n = 2^{-n}$ (exponentiell klein, je mehr Ausführungen desto kleiner).

    \end{enumerate}
    \item 2MC-Algorithmen:
    \begin{enumerate}[label=(\roman*)]
        \item Wie wird ein 2MC-Algorithmus für Berechnung einer Funktion definiert?\newline
        \textbf{Antwort}: Ein randomisierter Algorithmus $A$ berchnet $F(x)$ mit $Pr[A(x) = F(x)] \geq 1/2 + c$ für alle $x$, wobei $c > 0$ eine feste Konstante ist. Bedeutet: ,Ein randomisierter Algorithmus ist 2MC für $F$, wenn er für jede Eingabe $x$ den richtigen Wert $F(x)$ mit Wahrscheinlichkeit mindestens $1/2 + c$ liefert.

        \item Weiss ich, wie 2MC-Algorithmen mit n Wiederholungen verwendet werden?\newline
        \textbf{Antwort}: Führe $A$ n-mal unabhängig aus und gib den Wert aus, der am häufigsten vorkommt (Majority-Vote).

    \end{enumerate}
    \item MC-Algorithmen:
    \begin{enumerate}[label=(\roman*)]
        \item Wie wird ein MC-Algorithmus für Berechnung einer Funktion definiert?\newline
        \textbf{Antwort}: $Pr[A(x) = F(x)] > 1/2$ für alle $x$. (Kein fester Abstand über 1/2 gefordert). Bedeutet: Ein randomisierter Algorithmus ist MC für $F$, wenn er für jede Eingabe $x$ den richtigen Wert $F(x)$ mit Wahrscheinlichkeit grösser als $1/2$ liefert. Aber der Vorsprung über $1/2$ muss nicht fest sein (kann sehr klein werden).

        \clearpage
        \item Was kann bei MC-Algorithmen das Problem sein?\newline
        \textbf{Antwort}: Der Vorsprung über $1/2$ kann sehr klein sein und mit $|x|$ gegen 0 gehen (\enquote{verschwindende Lücke}). Dann braucht man extrem viele Wiederholungen $\rightarrow$ praktisch ineffizient.

        \item Kann ich das Beispiel (Kommunikationsprotokoll mit zwei Phasen) erklären?\newline
        \textbf{Antwort}: Phase 1 findet zufällig eine Position, an der sich $x$ und $y$ unterscheiden (klappt nur mit kleiner Wahrscheinlichkeit, z.B. $1/n$). Phase 2 nutzt dann eine leicht verzerrte Entscheidung. Insgesamt ist die Erfolgswahrscheinlichkeit zwar $> 1/2$, aber der Vorteil über $1/2$ wird für grosse $n$ winzig.

    \end{enumerate}
\end{enumerate}



\chapter{Optimierungsprobleme}
\section{Das 6-Tupel eines Optimierungsproblems}

Wie wird ein Optimierungsproblem allgemein beschrieben?

\begin{figure}[H]
    \img[width=1\textwidth]{figures/05/optimierungsproblem.png}
    {Optimierungsproblem}
    \label{fig:optimierungsproblem}
\end{figure}

Ein Optimierungsproblem $U$ ist ein 6-Tupel: 
$$
U = (\Sigma_I, \Sigma_O, L, M, \text{cost}, \text{goal})
$$
Dabei trennt man sauber (1) wie Instanzen/Lösungen dargestellt werden von (2) was als zulässige Lösung gilt und (3) was \enquote{gut} heisst.

\begin{itemize}
    \item $\Sigma_I$: (Input-)Alphabet, aus dem die Zeichen bestehen, mit denen Eingaben kodiert werden (z.B. $\{0,1\}$ oder ein ASCII-Alphabet). Instanzen sind dann Wörter aus $\Sigma_I^{*}$.
    \item $\Sigma_O$: (Output-)Alphabet, zur Kodierung von Ausgaben (also von Lösungen). Lösungen werden als Wörter aus $\Sigma_O$ dargestellt.
    \item $L \subseteq \Sigma_I^{*}$: Sprache der zulässigen Eingaben. Menge aller sinnvollen Eingaben/Instanzen. Ein $x \in L$ heisst Problemfall/Instanz. Wichtig: Man betrachtet hier bewusst das Optimierungsproblem und nimmt an, dass Eingaben ausserhalv on $L$ nicht vorkommen.
    \item $M$: Menge zulässiger Lösungen. Funktion; $M: L \rightarrow \mathcal{P}(\Sigma_O^{*})$. Für jede Instanz $x$ ist $M(x)$ die Menge aller zulässigen Lösungen (die alle Constraints von $x$ erfüllen).
    \item $\text{cost}$: Kostenfunktion. Bewertet eine zulässige Lösung $\alpha \in M(x)$ für die Instanz $x$ durch einen nichtnegativen rationalen Wert (typisch: \enquote{Länge}, \enquote{Gewicht}, \enquote{Anzahl}, \enquote{Zeit}).
    \item $\text{goal} \in \{\text{Minimum, Maximum}\}$: Optimierungsziel. Left fest, ob die kleinsten oder grössten Kosten gesucht werden (Minimierungs- vs. Maximierungsproblem).
\end{itemize}

Damit ist eine Lösung $\alpha \in M(x)$ optimal, wenn sie das Ziel erreicht:
$$
\text{cost}(\alpha, x) = \text{Opt}_U(x) = \text{goal} \{ \text{cost}(\beta, x) | \beta \in M(x) \}
$$
Ein Algorithmus \enquote{löst} $U$, wenn er für jedes $x \in L$ eine zulässige Lösung aus $M(x)$ ausgibt und diese optimal ist.

\section{Zulässige Lösungen und das Optimum}

\begin{multicols}{2}
    \raggedcolumns
    \subsection{Zulässigkeit (Validity)}
    Ein Algorithmus $A$ ist nur dann zulässig für $U$, wenn für jede Eingabe $x \in L$ die Ausgabe $A(x)$ ein Element von $M(x)$ ist.

    Das bedeutet: Die Lösung muss alle Constraints (Einschränkungen) erfüllen.
    \vspace{0.5cm}
    $\phantom{a}$
    \columnbreak
    \begin{figure}[H]
        \img[width=0.4\textwidth]{figures/05/filter.png}
        {Filter der Zulässigkeit}
        \label{fig:filter}
    \end{figure}
\end{multicols}

\vspace{0.25cm}

\begin{multicols}{2}
    \raggedcolumns
    \subsection{Optimalität (Optimality)}
    Eine zulässige Lösung $\alpha \in M(x)$ heisst optimal, wenn ihre Kosten den extremsten Wert annehmen.
    $$
    \text{cost}(\alpha, x) = \text{Opt}_U(x)
    $$ $$ 
    = \text{goal}\{\text{cost}(\beta, x) | \beta \in M(x)\}
    $$
    \vspace{0.5cm}
    $\phantom{a}$
    \columnbreak
    \begin{figure}[H]
        \img[width=0.4\textwidth]{figures/05/optimal.png}
        {Optimalität der Lösung}
        \label{fig:optimal}
    \end{figure}
\end{multicols}

\section{Randomisierung mit Wiederholung}

\begin{figure}[H]
    \img[width=1\textwidth]{figures/05/randomisierung.png}
    {Randomisierung durch Wiederholung}
    \label{fig:randomisierung}
\end{figure}

Man lässt $A(x)$ mehrmals unabhängig auf derselben Instanz $x$ laufen, bekommt Lösungen $y_1, \dots, y_k$, berechnet jeweils ihren Kostenwert $\text{cost}(y_i, x)$ und gibt am Ende einfach die beste Lösung gemäss $\text{goal}$ (Minimum/Maximum) aus.

Voraussetzung ist, dass $A$ ein zulässiger Algorithmus ist, d. h. jede Ausgabe eine zulässige Lösung: $A(x) \in M(x)$ für alle $x \in L$. Dann bleibt auch das Endergebnis zulässig, weil wir nur unter zulässigen Kandidaten den besten auswählen.

Was bringt die Wiederholung? Wenn ein einzelner Lauf die optimale Lösung mit Wahrscheinlichkeit $p$ findet, dann ist die Wahrscheinlichkeit, dass nach $k$ unabhängigen Läufen mindestens einmal eine optimale Lösung dabei ist:
$$
1 - (1 - p)^k
$$
Das ist genau die Idee der Wahrscheinlichkeitsverstärkung durch Wiederholungen: Die Erfolgswahrscheinlichkeit steigt schnell, währendn die Laufzeit nur um den Faktor $k$ (linear) wächst.

\clearpage
\section{Wahrscheinlichkeit der optimalen Lösung}

Angenommen, ein randomisierter Algorithmus $A$ findet in einem Lauf auf einer Instanz $x$ eine optimale Lösung nur mit Wahrscheinlichekit
$$
p = \frac{1}{|x|} \quad (\text{mit } |x| = \text{ Eingabelänge})
$$
Dann ist die Wahrscheinlichkeit, das Optimum in einem Lauf nicht zu finden:
$$
1 - p = 1 - \frac{1}{|x|}
$$
Bei $k$ unabhängigen Wiederholungen ist die Wahrscheinlichkeit, das Optimum nie zu finden:
$$
(1-p)^k = \left( 1 - \frac{1}{|x|} \right)^k
$$
Wählt man $k = |x|$, so gilt:
$$
\left(1 - \frac{1}{|x|}\right)^{|x|} \leq e^{-1} = \frac{1}{e}
$$
denn allgemein folgt aus $ln(1-z) \leq -z$ (für $0 < z < 1$) die Schranke $(1 - z)^m \leq e^{-zm}$. Damit ist die Erfolgswahrscheinlichkeit, das Optimum mindestens einmal zu treffen:
$$
1- \left(1 - \frac{1}{|x|}\right)^{|x|} \geq 1 - \frac{1}{e} \approx 0,6321
$$
Also: 63,21\%.

\begin{multicols}{2}
    Allgemeiner bei $k = c|x|$ Wiederholungen erhält man:
    $$
    Pr[\text{mind. ein Treffer}] \geq 1 - e^{-c}
    $$
    also steigt die Erfolgswahrscheinlichkeit durch lineare Wiederholung schnell gegen 1 (bei nur linearem Laufzeitfaktor).

    \begin{figure}[H]
        \img[width=0.4\textwidth]{figures/05/wachstum.png}
        {Erfolgswahrscheinlichkeit}
        \label{fig:wachstum}
    \end{figure}
\end{multicols}

\section{Anwendungsbeispiele}

\subsection{Traveling Salesman Problem (TSP)}

Gegeben sind Städte und Distanzen. Gesucht ist eine Rundreise, die jede Stadt genau einmal besucht und zum Start zurückkehrt, mit minimaler Gesamtlänge. Grob löst man es durch \enquote{Suche im Raum aller Touren} (exponentiell viele), daher nutzt man in der Praxis Heuristiken/Approximationen wie Nearest Neighbor, Local Search oder (bei metrischen TSP) Christofides.

\begin{multicols}{2}
    
    \textbf{Eingabe}:\newline
    Ein gewichteter, vollständiger Graph $(G, c)$. $G = (V, E)$, Kostenfunktion $c: E \rightarrow \N$.
    
    \textbf{Einschränkungen}:\newline
    $M(G,c)$: Die Menge aller Hamiltonschen Kreise (Permutation aller Knoten, jeder genau einmal besucht).
    
    \textbf{Kosten}:\newline
    Summe der GEwichte aller Kanten im Kreis.
    
    \textbf{Ziel}: Minimum

    \begin{figure}[H]
        \img[width=0.4\textwidth]{figures/05/tsp.png}
        {Traveling Salesman Problem}
        \label{fig:tsp}
    \end{figure}

\end{multicols}

\subsection{MIN-VCP (Vertex Cover)}

Gegeben ist ein Graph. Gesucht ist eine kleinste Menge von Knoten, sodass jede Kante mindestens einem ausgewählten Knoten \enquote{hängt}. Ein klassischer grober Ansatz ist die 2-Approximation über ein maximales Matching: Nimm zu jeder gewählten Matching-Kante beide endpunkte ins Cover. Dies garantiert hächstens Faktor 2 vom Optimum.

\begin{multicols}{2}
    
    \textbf{Eingabe}:\newline
    Ein ungerichteter Graph $G = (V, E)$
    
    \textbf{Einschränkungen}:\newline
    $M(G)$. Eine Teilmenge von Knoten $U \subseteq V$, sodass jede Kante in $E$ mit mindestens einem Knoten aus $U$ verbunden ist.
    
    \textbf{Kosten}:\newline
    Die Anzahl der Knoten in $U (|U|)$.
    
    \textbf{Ziel}: Minimum

    \begin{figure}[H]
        \img[width=0.4\textwidth]{figures/05/vertexcover.png}
        {MIN-VCP - Vertex Cover}
        \label{fig:min-vcp}
    \end{figure}

\end{multicols}

\clearpage
\subsection{MAX-SAT (Maximale Erfüllbarkeit)}

Gegeben ist eine CNF-Formel (UND von Klauseln). Gesucht ist eine Begelung der Variablen, die möglichst viele Klauseln erfüllt. Grob funktioniert ein randomisierter Ansatz so: Wähle eine zufällige Belegung und verbessere sie ggf. durch lokale Suche/Random Walks oder Greedy-Schritte.

\begin{multicols}{2}
    
    \textbf{Eingabe}:\newline
    Eine Formel $\Phi$ in Konunktiver Normalform (KNF).
    
    \textbf{Einschränkungen}:\newline
    $M(\Phi)$. Jede mögliche Belegung der Variablen mit Wahrheitswerten ${0, 1}$.
    
    \textbf{Kosten}:\newline
    Anzahl der erfüllten (wahren) Klauseln.
    
    \textbf{Ziel}: Maximum. Unterschied zum SAT: WIr suchen die beste Belegung, auch wenn nicht alle Klauseln erfüllbar sind.

    \begin{figure}[H]
        \img[width=0.4\textwidth]{figures/05/maxsat.png}
        {MAX-SAT}
        \label{fig:max-sat}
    \end{figure}

\end{multicols}

\subsection{ILP (Integer Linear Programming)}

Gesucht sind ganzzahlige Variablenwerte, die lineare Nebenbedingungen erfüllen und eine lineare Zielfunktion minimieren/maximieren. Grob löst man ILPs typischerweise mit Branch-and-Bound/Branch-and-Cut: Man löst erst die LP-Relaxation, verzweigt bei nicht-ganzzahligen Lösungen und schneidet mit zusätzlichen Ungleichungen (Cuts) den Suchraum ein.

\begin{multicols}{2}
    
    \textbf{Eingabe}:\newline
    Matrix $A (m \times n)$, Vektoren $b$ und $c$. Werte sind ganze Zahlen.
    
    \textbf{Einschränkungen}:\newline
    $M(A, b, c)$. Alle Vektoren $x \in \N^n$, die das Gleichungssystem $Ax = b$ (oder $Ax \leq b$) erfüllen. Wichtig: Lösungen müssen ganzzahlig sein.
    
    \textbf{Kosten}:\newline
    Lineare Funktion $c^Tx$.
    
    \textbf{Ziel}: Minimum (oder auch Maximum).

    \begin{figure}[H]
        \img[width=0.4\textwidth]{figures/05/ilp.png}
        {ILP (Integer Linear Programming)}
        \label{fig:ilp}
    \end{figure}

\end{multicols}

\clearpage  
\begin{enumerate}[label=(\alph*)]
    \item Optimierungsprobleme:
    \begin{enumerate}[label=(\roman*)]
        \item Wie wird ein randomisierter Algorithmus für ein Optimierungsproblem durch Wiederholungen eingesetzt?
        \item Was können wir über die Wahrscheinlichkeit für die optimale Lösung sagen?
        \item Welche Eigenschaft soll eine zulässige Lösung haben?
        \item Wie wird ein Optimierungsproblem allgemein beschrieben?
    \end{enumerate}
    \item Beispiele:
    \begin{enumerate}[label=(\roman*)]
        \item Kann ich das TSP-Problem erklären?
        \item Kann ich das MIN-VCP-Problem erklären?
        \item Kann ich das MAX-SAT-Problem erklären?
        \item Kann ich erklären, was IPL-Probleme sind?
    \end{enumerate}
\end{enumerate}

\chapter{Analyse von randomisierten Algo's für Optimierungsprobleme}
\section{Approximations-Güte}

Wir vergleichen:

\begin{itemize}
    \item $\text{cost}(A(x))$ = Wert/Kosten der Lösung des Algorithmus
    \item $\text{Opt}(x)$ = Wert/Kosten der optimalen Lösung
\end{itemize}

Da es Minimierungs- und Maximierungsprobleme gibt, nutzt man eine Definition, die immer $\geq 1$ ist:
$$
\text{Güte}_A(x) = \max\left\{\frac{\mathrm{cost}(A(x))}{\mathrm{Opt}(x)},\ \frac{\mathrm{Opt}(x)}{\mathrm{cost}(A(x))}\right\}
$$

Intuition:

\begin{itemize}
    \item $1.0$ bedeutet: Algorithmus ist optimal.
    \item Je grösser, desto schlechter (weiter weg vom Optimum).
\end{itemize}

Warum dieses \enquote{max}?

\begin{itemize}
    \item Beim Minimum ist $\text{cost}(A)$ normalerweise $\geq \text{Opt}$, dann ist $\frac{\text{cost}(A)}{\text{Opt}} \geq 1$.
    \item Beim Maximum ist $\text{cost}(A)$ normalerweise $\leq \text{Opt}$,
    dann ist $\frac{\text{Opt}}{\text{cost}(A)} \geq 1$.
\end{itemize}

\clearpage
\section{Was ist ein $\delta$-Approximationsalgorithmus}

Ein Algorithmus $A$ ist ein $\delta$-Approximationsalgorithmus, wenn seine Gẗe für jede Eingabe höchstens $\delta$ ist:
$$
\text{Güte}_A(x) \leq \delta, \quad \forall x
$$

\textbf{Minimum-Problem (z.B. \enquote{Kosten minimieren})}

Hier gilt typischerweise $\text{cost}(A(x)) \geq \text{Opt}(x)$. Dann bedeutet $\delta$-Approximation:
$$
\text{cost}(A(x)) \leq \delta \cdot \text{Opt}(x)
$$
Sprich: Die Lösung ist höchstens Faktor $\delta$ teurer/schlechter als optimal.

\textbf{Maximum-Problem (z.B. \enquote{Gewinn maximieren})}

Hier gilt typischerweise $\text{cost}(A(x)) \leq \text{Opt}(x)$. Dann bedeutet $\delta$-Approximation:
$$
\text{Opt}(x) \leq \delta \cdot \text{cost}(A(x))
$$
Umgestellt (oft intuitiver):
$$
\text{cost}(A(x)) \geq \frac{1}{\delta} \cdot \text{Opt}(x)
$$
Sprich: Du erreichst mindestens neinen $\frac{1}{\delta}$-Anteil des Optimums.

\section{VAC-Algorithmus für MIN-VCP (Vertex Cover)}

\textbf{Problem (MIN-VCP)}

Gegeben ein ungerichteter Graph $G = (V, E)$.

Gesucht: möglichst wenige Knoten $C \subseteq V$, so dass jede Kante mindestens einen Endpunkt in $C$ hat.

Ziel: minimiere $|C|$.

\clearpage
\textbf{VAC-Idee}

Algorithmus:

\begin{enumerate}
    \item Wähle irgendeine Kante ${u,v}$
    \item Nimm beide Knoten $u$ und $v$ ins Cover $C$.
    \item Entferne alle Kanten, die an $u$ oder $v$ hängen (die sind jetzt abgedeckt).
    \item Wiederhole, bis keine Kanten mehr übrig sind.
\end{enumerate}

\textbf{Warum ist das eine 2-Approximation?}

Denk an jede Kante, die VAC auswählt. Für diese Kante ${u, v}$ gilt:

\begin{itemize}
    \item Jedes gültige Vertex Cover (also auch das Optimum) muss mindestens einen der beiden Knoten nehmen, sonst wäre diese Kante nicht abgedeckt.
    \item VAC nimmt immer beide. \enquote{bezahlt} 2 statt mindestens 1.
\end{itemize}

Wenn VAC insgesamt $k$ Kanten auf diese Weise \enquote{auswählt}, nimmt VAC $2k$ Knoten. Das Optimum braucht mindestens $k$ Knoten (weil pro ausgewählter Kante mindestens 1 nötig ist). Also:
$$
|C_{VAC}| \leq 2 \cdot |C_{\text{Opt}}|
$$

\section{Zwei Arten von Garantien}

Bei randomisierten Algorithmen ist das Ergebnis (und damit die Güte) zufällig (also eine Zufallsvariable).

\textbf{Typ 1: Randomisierter $E[\delta]$-Approximationsalgorithmus}

Eigenschaften:

\begin{enumerate}
    \item Liefert immer eine zulässige Lösung.
    \item Der Erwartungswert der Güte ist höchstens $\delta$:
\end{enumerate}
$$
\mathbb{E}[\text{Güte}_A(x)] \leq \delta
$$

Bedeutung: Im Durchschnitt gut, aber es können selten auch sehr schlechte Ausreisser passieren.

\textbf{Typ 2: Randomisierter $\delta$-Approximationsalgorithmus}

Eigenschaften:

\begin{enumerate}
    \item Liefert immer eine zulässige Lösung.
    \item Mit Wahrscheinlichkeit mindestens $1/2$ ist die Güte wirklich höchstens $\delta$:
\end{enumerate}
$$
Pr(\text{Güte}_A(x) \leq \delta) \geq \frac{1}{2}
$$
Bedeutung: Mindestens in 50\% der Läufe bekommst du eine Lösung, die den Faktor $\delta$ wirklich einhält (eine \enquote{Trefferquote}-Garantie).

\subsection{Unterschied Typ 1 vs. Typ 2}

Typ 1 (Erwartungswert):
\begin{itemize}
    \item bewertet den Durchschnitt.
    \item kann durch wenige extreme Ausreisser \enquote{versaut} werden.
\end{itemize}

Typ 2 (Wahrscheinlichkeit / Trefferquote):
\begin{itemize}
    \item garantiert, dass du mit $\geq$ 50\% Wahrscheinlichkeit wirklich \enquote{gut genug} bist.
    \item sagt nichts darüber, wie schlimm die anderen $\leq$ 50\% sein können.
\end{itemize}

Wenn 10 von 12 Läufen super sind (Güte 2), aber 2 Läufe extrem schlecht (Güte 50), dann kann der Erwartungswert ziemlich gross werden, obwohl die Trefferquote (10/12) sehr gut ist.

\clearpage
\section{Summary}

\begin{enumerate}[label=(\alph*)]
    \item Eigenschaften von Algorithmen für Opimtierungsprobleme:
    \begin{enumerate}[label=(\roman*)]
        \item Wann ist ein Algorithmus für ein Optimierungsproblem zulässig?\newline
        \textbf{Antwort}: Ein Algorithmus $A$ ist für ein Optimierungsproblem zulässig, wenn er für jede Eingabe $x$ immer eine gültige Lösung ausgibt. Also: $A(x)$ erfüllt alle Nebenbedingungen des Problems (auch wenn die Lösung nicht optimal ist).

        \item Was ist die Approximationsgüte?\newline
        \textbf{Antwort}: Die Approximationsgüte misst, wie weit die Algorithmuslösung vom Optimum entfernt ist. Damit es für Minimum und Maximum gleich \enquote{funktioniert}, definiert man sie so, dass sie immer $\geq 1$ ist. Siehe Kapitel 7.1.

        \item Was ist ein $\delta$-Approximations-Algorithmus und was bedeutet diese Eigenschaft konkret für ein Maximums- bzw. Minimumgsproblem?\newline
        \textbf{Antwort}: $A$ ist ein $\delta$-Approximationsalgorithmus, wenn für alle Eingaben gilt: $\text{Güte}_A(x) \leq \delta$. Konkret:\newline
        Minimierungsproblem (z.B. Kosten minimieren): Der Algorithmus ist höchstens Faktor $\delta$ teurer/schlechter als optimal.\newline
        Maximierungsproblem (z.B. Gewinn maximieren):
        Der Algorithmus erreicht mindestens einen $1 / \delta$-Anteil des Optimums.

        \item Kann ich den VAC-Algorithmus für das MIN-VCP-Problem erklären?\newline
        \textbf{Antwort}: Beim Vertex-Cover wählt VAC wiederholt eine beliebige Kante ${u, v}$, nimmt beide Endpunkte ins Cover und entfernt alle dadurch abgedeckten Kanten. Am Ende sind alle Kanten abgedeckt, also ist die Lösung zulässig. Es ist eine 2-Approximation, weil pro ausgewälter Kante jedes Cover mindestens einen Endpunkt braucht, da VAC aber immer zwei nimmt, wird es höchstens doppelt so gross wie optimal.

    \end{enumerate}
    Weiter auf nächster Seite.
    \clearpage
    \item Eigenschaften für randomisierte Algorithmen:
    \begin{enumerate}[label=(\roman*)]
        \item Was ist ein randomisierter $E[\delta]$-Approximations-Algorithmus?\newline
        \textbf{Antwort}: Ein randomisierter Algorithmus ist ein $E[\delta]$-Approximationsalgorithmus, wenn er immer zulässige Lösungen liefert und der Erwartungswert der Approximationsgüte höchstens $\delta$ ist, also im Durchschnitt Faktor-$\delta$-gut (trotz möglicher Ausreisser).

        \item Was ist ein randomisierter $\delta$-Approximations-Algorithmus?\newline
        \textbf{Antwort}: Das ist ein randomisierter Algorithmus, der immer zulässig ist und mit Wahrscheinlichkeit mindestens $1/2$ tatsächlich eine Lösung mit Güte $\leq \delta$ liefert (also eine \enquote{Trefferquote}-Garantie für eine $\delta$-gute Lösung).

        \item Wie unterscheiden sich die zwei Arten von randomisierten Approximations-Algorithmen?\newline
        \textbf{Antwort}: $E[\delta]$ garantiert \enquote{gut im Mittel} (kann selten sehr schlechte Ergebnisse haben), während der randomisierte $\delta$-Approximationsalgorithmus \enquote{oft gut} garantiert (mindestens 50\% der Läufe sind wirklich $\delta$-gut), aber nichts darüber sagt, wie schlecht die restlichen Läufe sein können.
    \end{enumerate}
\end{enumerate}

\chapter{Randomisierte Approximations-Algorithmen}
In diesem Kapitel wurden die Fragen des Lerngerüsts direkt im Kapitel beantwortet. Somit gibt es kein Summary.

Auch gilt für dieses Kapitel:

\begin{itemize}
    \item $X$ die Approximationsgüte (oder allgemein irgendeine nichtnegative Zufallsgrösse, die aussagt \enquote{wie schlecht} der Lauf war.)
    \item $E[X] \leq \delta$: Im Erwartungswert ist die Güte höchstens $\delta$. Wenn wir den Algorithmus sehr oft laufen lassen und die Güte mitteln, wäre sie im Durchschnitt $\leq \delta$.
    \item $P(X \geq 2 \delta) \leq \frac{E[X]}{2 \delta} \leq \frac{\delta}{2 \delta} = \frac{1}{2}$: Bedeutet, dass in mindestens 50\% der Läufe die Güte höchstens $2 \delta$ ist. Deshalb sagt man: \enquote{Aus einem randomisierten $E[\delta]$- Approximationsalgorithmus wird ein randomisierter $\delta^{*}$-Approximationsalgorithmus mit $\delta^{*} = 2 \delta$}.
\end{itemize}

\section{Wie muss $\delta$ angepasst werden, damit aus $E[\delta]$ ein $\delta^{*}$-Algorithmus wird?}

Wenn ein randomisierter Algorithmus nur eine Erwartungswert-Garantie hat (also $E[\text{Güte}] \leq \delta$), wollen wir oft eine \enquote{Trefferquote}-Garantie: mit Wahrscheinlichkeit $\geq 1/2$ ist die Güte wirklich gut. Das erreichen wir, indem wir den Faktor verdoppeln:
$$
\delta^{*} = 2 \delta
$$
Dann gilt: Mit Wahrscheinlichkeit mindestens $1/2$ ist die Güte $\leq 2 \delta$ (also \enquote{nicht zu schlecht}). Das ist genau der Trick \enquote{von links nach rechts} (Erwartung $\rightarrow$ Wahrscheinlichkeit).

\clearpage
\section{Was sagt die Markov-Ungleichung?}

Für eine nicht-negative Zufallsvariable $X \geq 0$ gilt für jedes $t > 0$:
$$
P(X \geq t) \leq \frac{E[X]}{t}
$$
Bedeutung: Wenn der Durchschnitt $E[X]$ klein ist, dann kann $x$ nicht oft sehr gross sein. Genau damit begründet man oben die Verdopplung: Setze $t = 2 \delta$ und nutze $E[X] \leq \delta$, dann ist $P(X \geq 2\delta) \leq 1/2$ also: $P(X \leq 2\delta) \geq 1/2$.

\section{Welche Approximationseigenschaften hat STICH für MAX-Ek-SAT?}

Problem (MAX-Ek-SAT): Wir haben eine KNF-Formel mit $m$ Klauseln, jede Klausel hat genau $k$ Literale. Ziel: Maximiere die Anzahl erfüllter Klauseln.

Algorithmus STICH: Setze jede Variable unabhängig zufällig: mit Wahrscheinlichkeit $1/2$ auf wahr, sonst falsch (fairer Münzwurf).

Analyse einer Klausel: Eine Klausel ist nur dann falsch, wenn alle $k$ Literale falsch werden. Das passiert mit Wahrscheinlichkeit $(1/2)^k = 2^{-k}$. Also ist die Klausel mit Wahrscheinlichkeit
$$
P(\text{Klausel wahr}) = 1 - 2^{-k}
$$
Interpretation: Im Schnitt erfüllt STICH den Anteil $(1 - 2^{-k})$ aller Klauseln.

Approximationsgarantie (als $E[\delta]$-Approximation): Das Optimum ist höchstens $m$ (mehr als alle Klauseln geht nicht). Daher ist die erwartete Approximationsgüte (für ein Maximumproblem) höchstens:
$$
\delta = \frac{m}{m(1-2^{-k})} = \frac{1}{1 - 2^{-k}} = \frac{2^k}{2^k-1}
$$
Beispiel: $k = 1 \Rightarrow \delta \leq 2, \quad k = 2 \Rightarrow \delta \leq 4/3, \quad k= 3 \Rightarrow \delta \leq 8/7$.

\begin{figure}[H]
    \img[width=1\textwidth]{figures/markov.png}
    {Markov Summary}
    \label{fig:markovsummary}
\end{figure}

\begin{figure}[H]
    \img[width=1\textwidth]{figures/STICH.png}
    {STICH Summary}
    \label{fig:stichsummary}
\end{figure}

Wobei $m$ die Anzahl Klauseln in der Formel ist. $m \cdot (1 - 2^{-k})$ ist somit die erwartete Anzahl erfüllter Klauseln.

\chapter{Paradigmen, Überlisten des Gegners, Online-Probleme}
\section{Paradigmen für randomisierte Algorithmen (Überblick)}

\begin{enumerate}
    \item Überlisten des Gegners
    \item Fingerprinting
    \item Wahrscheinlichkeitsverstärkung
    \item Häufige Zeugen
    \item Random Rounding
\end{enumerate}

\section{Überlisten des Gegners}

\textbf{Deterministisch}: Der Gegner kennt deinen Algorithmus genau und kann die Eingabe so wählen, dass du garantiert im Worst Case landest.

\textbf{Randomisiert}: Der Gegner kennt zwar den Algorithmus, aber nicht deine Zufallsentscheidungen. Dadurch kann er den Worst Case nicht mehr zuverlässig \enquote{treffen}. Ziel: Worst-Case (gegen Gegner) wird zu Average-Case (über Zufall).

Wichtig: Das heisst nicht \enquote{kein Worst Case existiert}, sondern: Er wird unwahrscheinlich, weil der Gegner nicht weiss, welchen zufälligen Pfad der Algorithmus nimmt.

\clearpage
\section{Hashing-Problem (Musterbeispiel fürs Überlisten)}

\textbf{Was ist das Hashing Problem?}

\begin{figure}[H]
    \img[width=1\textwidth]{figures/hashing.png}
    {Hashing Problem}
    \label{fig:hashingproblem}
\end{figure}

Wir haben ein grosses Universum $U$ möglicher Schlüssel (z.B. alle möglichen IDs), aber nur eine Tabelle $T$ mit $m$ Slots ($0$ bis $m-1$). In der Anwendung liegt jeweils eine aktuelle Menge $S$ von Schlüsseln vor, typischerweise mit $n = |S|$. Wir wählen die Hashfunktion $h$, die jeden Schlüssel $x$ auf einen Slot abbildet: $h: U \rightarrow {0, \dots, m-1}$.

\textbf{Problem}: Weil $U$ viel grösser als $T$ ist, sind Kollisionen unvermeidbar (verschiedene Schlüssel landen im gleichen Slot). Dann hängen in einem Slot mehrere Elemente (typisch als Liste/Chain), und Suchen/Einfügen/Löschen wird so langsam wie die Listenlänge.

\subsection{Warum scheitert deterministisches Hashing}

Wenn wir eine feste Hashfunktion $h$ benutzen, kann ein Gegner (der $h$ kennt) gezielt Schlüssel wählen, die alle in denselben Slot fallen. Dann wird eine $O(1)$-Suche zu einer lineare Liste der Länge $n$, also $O(n)$. Genau das ist \enquote{Gegner erzwingt Worst Case}.

\subsection{Warum ist Hashing ein Musterbeispiel fürs Überlisten}

Wir machen den Gegner mit der Randomisierung blind, indem wir die Hashfunktion zufällig auswählen (aus einer geeigneten Familie). Dann kann der Gegner zwar Schlüssel wählen, aber er weiss nicht, welche Hashfunktion verwendet wird, also kann er Kollisionen nicht zuverlässig erzwingen.

\section{Geeignete Hashfunktionen: Universelles Hashing}

\textbf{Grundidee}: Nicht \enquote{eine perfekte Hashfunktion finden}, sondern: eine Familie $H$ von Hashfunktionen definieren und beim Start zufällig $h \in H$ wählen.

\textbf{Definition}: Eine Familie $H$ heisst (in diesem Kontex) universell, wenn für jedes Paar verschiedener Schlüssel $x \neq y$ gilt:
$$
Pr_h[ h(x) = h(y)] \leq \frac{1}{m}
$$
wobei die Wahrscheinlichkeit über die zufällige Wahl von $h$ aus $H$ läuft.

Intuition: Für jedes feste Paar $x,y$ ist die Kollisionschance höchstens so klein, als würden wir \enquote{komplett zufällig} verteilen (ungefähr $1/m$).

\textbf{Warum bringt das erwartete $O(1)$?}

Wenn Kollisionen pro Paar so selten sind, ist die erwartete Kettenlänge pro Slot ungefähr $n/m$ (das ist der \enquote{Load Factor}). Wenn wir also $m$ proportional zu $n$ wählen (z.B. $m \approx n$), dann ist die erwartete Listenlänge konstant $\Rightarrow$ Search/Insert/Delete sind erwartungsgemäss $O(1)$. Das ist genau der gewünschte Effekt.

\textbf{Konkretes Beispiel einer universellen Familie}

Eine praktische universelle Klasse ist:

\begin{itemize}
    \item Wähle eine Primzahl $p > |U|$ (damit Rechnen \enquote{sauber} ist).
    \item Wähle zufällig Parameter $a \in {1, \dots, p-1}$ und $b \in {0, \dots, p-1}$.
    \item Definiere: $h_{a,b}(x) = ((a \cdot x + b) \mod p) \mod m$
\end{itemize}

\clearpage
Was bedeutet dies?

\begin{itemize}
    \item $a \cdot x + b$: lineare \enquote{Durchmischung} des Schlüssels $x$ ($a$: Steigung, $b$: Offset).
    \item $\bmod p$: Wir arbeiten in einem Bereich $0..p-1$ (Primzahl sorgt dafür, dass die Abbildung sich gut verteilt)
    \item $\bmod m$: Am Ende wird auf die Tabellengrösse $m$ reduziert (Slot $0..m-1$).
\end{itemize}

Das ist \enquote{geeignet}, weil man zeigen kann: diese Familie ist universell (Kollisionswahrscheinlichkeit bleibt $\lesssim 1/m$) und $h$ ist schnell berechenbar.

\section{Online-Probleme: Was ist das (und Beispiele)?}

Ein Online-Problem heisst: Die Eingabe kommt stückweise/sequenziell. Wir müssen nach jedem Teil sofort entscheiden, und die Entscheidung ist unwiderruflich (wir könne sie später nicht einfach rückgängig machen), obwohl wir die Zukunft nicht kennen. Der Gegner kann dabei sogar das Timing/\enquote{was als Nächstes kommt} so wählen, dass unsere früheren Entscheidungen schlecht aussehen.

Typische Beispiele:

\begin{itemize}
    \item Paging/Caching: Cache ist voll, neue Seite kommt $\rightarrow$ welche Seite werfen wir raus? (Zukunft unbekannt).
    \item Ski-Rental: Kaufen oder mieten, ohne zu wissen, wie oft wir noch fahren (Fixkosten vs. laufende Kosten).
    \item Scheduling: Jobs kommen nacheinander $\rightarrow$ welche Maschine/zeitliche Einplanung, ohne alle zukünftigen Jobs zu kennen.
\end{itemize}

\clearpage
\section{Summary}
\begin{enumerate}[label=(\alph*)]
    \item Paradigmen:
    \begin{enumerate}[label=(\roman*)]
        \item Welche Paradigmen für den Entwurf von randomisierten Algorithmen gibt es?\newline
        \textbf{Antwort}: (1) Überlisten des Gegners, (2) Fingerprinting, (3) Wahrscheinlichkeitsverstärkung, (4) Häufige Zeugen und (5) Random Rounding
    \end{enumerate}
    \item Überlisten des Gegners:
    \begin{enumerate}[label=(\roman*)]
        \item Kann ich die Idee erklären?\newline
        \textbf{Antwort}: Ein Gegner kann bei deterministischen Algorithmen eine Eingabe so wählen, dass immer der Worst Case passiert. Mit Randomisierung kennt der Gegner die Zufallsentscheidungen nicht und kann den Worst Case nicht gezielt erzwingen. Dadurch bekommen wir eine gute Leistung im Erwartungswert statt garantiert schlecht.

        \item Was ist das Hashing-Problem?\newline
        \textbf{Antwort}: Viele mögliche Schlüssel müssen auf wenige Tabellenplätze abgebildet werden. Dabei entstehen Kollisionen (mehrere Schlüssel landen im gleichen Slot), was Suche/Einfügen langsam machen kann, wenn sich zu viele in einem Slot sammeln.

        \item Warum ist Hashing ein Musterbeispiel für das überlisten des Gegners?\newline
        \textbf{Antwort}: Wenn die Hashfunktion fix ist, kann ein Gegner Schlüssel so wählen, dass sie alle kollidieren. Wählen wir die Hashfunktion zufällig, kann der Gegner diese Kollision nicht zuverlässig planen.

        \item Was sind geeignete Hashfunktionen?\newline
        \textbf{Antwort}: Solche, die Schlüssel gleichmässig verteilen und Kollisionen selten machen. Typisch: zufällige Wahl aus einer universellen Hashfamilie, sodass für jedes Schlüsselpaar die Kollisionswahrscheinlichkeit klein ist.

    \end{enumerate}
    \item Online-Probleme:
    \begin{enumerate}[label=(\roman*)]
        \item Was ist ein Online Problem?\newline
        \textbf{Antwort}: Die Eingabe kommt schrittweise, und wir müssen sofort Entscheidungen treffen, ohne die Zukunft zu kennen. Entscheidungen sind meistens nicht oder nur teuer rückgängig zu machen.

        \item Kenn ich Beispiele für Online Probleme?\newline
        \textbf{Antwort}: (1) Caching/Paging (welche Seite rauswerfen), (2) Ski-Rental (mieten vs. kaufen) und (3) Online-Scheduling (Jobs kommen nach und nach und müssen direkt eingeplant werden).
    \end{enumerate}
\end{enumerate}

\chapter{Online-Probleme}
\begin{figure}[H]
    \img[width=1\textwidth]{figures/online-vs-offline.png}
    {Online vs. Offline}
    \label{fig:onlinevsoffline}
\end{figure}

Offline Algorithmen bekommen die ganze Eingabe komplett am Anfang. Somit kennt man \enquote{die Zukunft} und kann dadurch perfekt planen.

Online Algorithmen bekommen die Eingabe stückweise: erst $x_1$, dann $x_2$, .... Wir müssen dann oft sofort entscheiden, ohne zu wissen, was als nächstes kommt. Die Entscheidungen sind unumkehrbar (man darf später nicht zurückspulen).

\clearpage
\section{Konkurrenzgüte}

Die Grundidee: Wir vergleichen den Online-Algorithmus mit einem hypothetischen \enquote{perfekten} Offline-Algorithmus $\text{Opt}$, der alles vorher weiss.

\begin{figure}[H]
    \img[width=1\textwidth]{figures/konkurrenzgüte.png}
    {Konkurrenzgüte}
    \label{fig:konkurrenzgüte}
\end{figure}

Kosten:

\begin{itemize}
    \item $\text{cost}(A(I))$: Kosten, die Algorithmus $A$ auf Eingabe $I$ verursacht.
    \item $\text{cost}(\text{Opt}(I))$: minimale (optimale) Kosten mit voller Zukunftskenntnis.
\end{itemize}

\textbf{Konkurrenzfaktor / Competitive Ratio}

Typisch (für Minimierungsprobleme), also \enquote{Kosten klein machen}:
$$
\text{cost}(A(I)) \leq c \cdot \text{cost}(\text{Opt}(I)) + b
$$
wobei:
\begin{itemize}
    \item $c$ = Konkurrenzfaktor (je kleiner, desto besser)
    \item $b$ = kleine Startkonstante (manchmal weggelassen, wenn unwichtig)
\end{itemize}

Wenn $b = 0$, sieht man oft auch einfach:
$$
\frac{\text{cost}(A(I))}{\text{cost}(\text{Opt}(I))} \leq c
$$

\clearpage
Was bedeutet \enquote{$\delta$-konkurrenzfähig}?

\begin{itemize}
    \item Ein Algorithmus heisst $\delta$-konkurrenzfähig, wenn er für alle Eingaben $I$ einen Konkurrenzfaktor $\leq \delta$ garantiert.
    \item Für Minimierung: \enquote{nicht mehr als $\delta$-mal so teuer}
    \item Für Maximierung (Gewinn gross machen) dreht sich der Vergleich um: Man will mindestens ein $1 / \delta$-Anteil des Optimums erreichen (gleiche Idee, nur andersherum formuliert).
\end{itemize}

\section{$\delta$-schwere Online Probleme}

Ein Online-Problem heisst $\delta$-schwer, wenn kein Online-Algorithmus eine bessere Konkurrenzgüte als $\delta$ erreichen kann (also $\delta$ ist eine echte Untergrenze).

Warum passiert das?

\begin{multicols}{2}

    \begin{figure}[H]
        \img[width=0.4\textwidth]{figures/chess.png}
        {Schach}
        \label{fig:schach}
    \end{figure}

    Weil ein Widersacher (Gegner, Adversary) die Eingabe so wählt, dass sie unserem Algorithmus maximal schadet:

\begin{itemize}
    \item Er kennt unseren deterministischen Algorithmus
    \item Er wartet unsere Entscheidung ab
    \item Dann gibt er genau den nächsten Input, der uns \enquote{bestraft}
\end{itemize}
    
\end{multicols}

\clearpage
\section{Beispiel: Paging (Cache-Verwaltung)}

\textbf{Problemidee}: Wir haben einen Cache (schnell), der nu $k$ Seiten speichern kann. Es kommt eine Anfrage nach einer Seite:

\begin{itemize}
    \item Wenn sie im Cache ist: Treffer, kein Problem.
    \item Wenn nicht: Page Fault $\rightarrow$ Seite muss aus dem langsamen Speicher geladen werden (teuer).
    \item Wenn der Cache voll ist müssen wir eine Seite rauswerfen (Eviction).
\end{itemize}

Ziel: Minimieren der Anzahl Page Faults.

\subsection{Warum ist Paging für deterministische Online-Algorithmen $k$-schwer?}


\begin{multicols}{2}
    Intuition (klassische Gegner-Falle):

    \begin{enumerate}
        \item Der Cache enthält anfangs $k$ Seiten, z.B. $p_1, \dots, p_k$
        \item Der Gegner fordert eine neue Seite $p_{k+1}$ an. Der Cache ist voll, wir müssen eine Seite rauswerfen (z.B. immer die $p_j$ mit dem kleinsten $j$).
        \item Sofort danach fordert der Gegner genau die Seite $p_j$ an, die wir gerade entfertn haben.
        \item Das wiederholt er so, dass wir bei jedem Schritt einen Page Fault erhalten.
    \end{enumerate}

    \begin{figure}[H]
        \img[width=0.4\textwidth]{figures/paging-problem.jpg}
        {Paging Problem}
        \label{fig:pagingproblem}
    \end{figure}
\end{multicols}

Ergebnis:

\begin{itemize}
    \item Unser Online-Algorithmus kann zu $k$ Faults gezwungen werden,
    \item während $\text{Opt}$ oft mit 1 Fault auskommt (weil wir beim ersten Rauswerfen \enquote{richtig} planen). Damit ist das Verhältnis $\frac{\text{cost}(A)}{\text{cost}(\text{Opt})} = k$. Also: $k$-schwer.
\end{itemize}

\subsection{Randomisierte Online-Algorithmen}

Idee: Den Gegner \enquote{entwaffnen}. Wenn der Algorithmus zufällig entscheidet, kann der Gegner nicht mehr sicher vorhersagen, was passiert.

Typisches Modell (fair für Randomisierung): \textbf{Oblivious Adversary}. Die Eingabe musst festgelegt werden, bevor der Algorithmus die zufällige Entscheidung trifft.

\textbf{Neue Messlatte: Erwartungswert}

Statt \enquote{immer garantiert} schaut man auf die erwarteten Kosten:
$$
\mathbb{E}[\text{cost}(A(I))] \leq c \cdot \text{cost}(\text{Opt}(I))
$$
\begin{itemize}
    \item $\mathbb{E}[\cdot]$ heisst: Erwartungswert (Durchschnitt über viele Münzwürfe).
    \item $c$ heisst dann oft erwartete Konkurrenzgüte
\end{itemize}

Warum passt das zum Paradigma \enquote{Überlisten des Gegners}?

\begin{itemize}
    \item Wir haben viele mögliche deterministische Strategien
    \item Wir wählen zufällig eine davon
    \item Der Gegner kann nicht mehr \enquote{massschneidern}, welche Eingabe exakt unsere Strategie zerstört.
\end{itemize}

\section{Beispiel 2: Arbeitsverteilung (Unit Job-Scheduling)}

Problem in Worten:

\begin{itemize}
    \item Es gibt zwei Jobs (Aufträge)
    \item Es gibt $m$ Maschinen-Schritte pro Job (jeder Job braucht jede Maschine genau einmal, aber in eigener Reihenfolge).
    \item Konflikt: Beide Jobs wollen gleichzeitig dieselbe Maschine. Also muss einer warten.
\end{itemize}

\clearpage
Man stellt die Situation als Weg von $(0, 0)$ nach $(m, m)$ dar:

\begin{itemize}
    \item x-Achse: Fortschritt Job 1
    \item y-Achse: Fortschritt Job 2
    \item Ein Schritt diagonal bedeutet: beide arbeiten gleichzeitig (ideal).
    \item Ein Hindernis/Block bedeutet: diagonal geht nicht. Einer muss horizontal/vertikal \enquote{ausweichen} (Warten).
\end{itemize}

\begin{figure}[H]
    \img[width=0.8\textwidth]{figures/grid.png}
    {Gitter-Bild. Quelle: \cite[S. 145]{hromkovic2004randomisierte}}
    \label{fig:jobs-grid}
\end{figure}

Die Kostenformel (einfach):
$$
\text{cost} = m + \text{Delay}
$$
\begin{itemize}
    \item $m$ ist \enquote{Grundkosten} (man braucht insgesamt $m$ Diagonal-Schritte, wenn alles perfekt wäre)
    \item $\text{Delay}$ ist die zusätzliche Wartezeit durch Konflikte
\end{itemize}

\subsection{Die deterministische Falle}

Wenn wir deterministisch ausweichen, kann der Gegner die Hindernisse so legen, dass wir immer wieder ins nächste Hindernis laufen. Dadurch gäbe es sehr viel Delay. Die Competitive Ration kann sehr schlechten werden (linear in $m$).

\subsection{Algorithmus DIAG (randomisiert)}

Ziel: Nicht vorhersagbar sein, auf welcher \enquote{Spur} (Diagonal-Nähe) man läuft.

DIAG in 3 einfachen Schritten:

\begin{enumerate}
    \item Random: Wähle zufällig eine Diagonale $D_i$ in einem Bereich der Hauptdiagonale.
    \item Follow: Laufe so gut wie möglich auf dieser Diagonale (so viele Diagonalschritte wie möglich).
    \item Recover: Wenn ein Hindernis kommt: kurz ausweichen (1 horizontal + 1 vertikal), dann zurück zur Diagonale.
\end{enumerate}

Warum das hilft:

\begin{itemize}
    \item Der Gegner kann Hindernisse nicht so platzieren, das alle möglichen zufällig gewählten Diagonalen \enquote{gleich schlimm} sind.
    \item Dadurch wird die erwartete Zusatzwartezeit klein: ungefähr in der Grössenordnung $\sqrt{m}$
\end{itemize}

Präzise gemäss Buch: Die erwarteten Kosten liegen etwa bei $m + O(\sqrt{m})$, und damit ist die erwartete (normierte) Konkurrenz sehr gut.

\begin{figure}[H]
    \img[width=0.7\textwidth]{figures/grid_diag_moved.png}
    {Gitter, neue Diagonale. Quelle: \cite[S. 148]{hromkovic2004randomisierte}}
    \label{fig:jobs-grid-moved}
\end{figure}

Key-Facts:

\begin{itemize}
    \item Oft meint man mit \enquote{Faktor} in der Intuition den Warte-Teil (Delay): deterministisch kann Delay $\approx m$ sein, randomisiert wird erwarteter Delay $\approx \sqrt{m}$.
    \item Wenn man den Gesamt-Cost $m + \text{Delay}$ betrachtet, ist das Verhältnis zu $\text{Opt} \geq m$ sogar nahe bei $1$ (weil beide mindestens m zahlen müssen).
\end{itemize}

\clearpage
\begin{enumerate}[label=(\alph*)]
    \item Optimierungsprobleme als Online-Problem:
    \begin{enumerate}[label=(\roman*)]
        \item Was ist ein Online Algorithmus?\newline
        \textbf{Antwort}: Ein Algorithmus, der die Eingabe Schritt für Schritt bekommt und nach jedem Schritt sofort entscheiden muss, ohne die Zukunft zu kennen.

        \item Was ist ein Offline Algorithmus?\newline
        \textbf{Antwort}: Ein Algorithmus, der die komplette Eingabe von Anfang an kennt und dadurch optimal planen kann.

        \item Wie wird die Konkurrenzgüte definiert?\newline
        \textbf{Antwort}: Man vergleicht die Kosten (oder den Gewinn) des Online-Algorithmus mit dem optimalen Offline-Algorithmus $\text{Opt}$. Für Minimierung typisch:\\$\text{cost}(A(I)) \leq c \cdot \text{cost}(\text{Opt}(I))$

        \item Was ist ein $\delta$-konkurrenzfähiger Algorithmus und was bedeutet diese Eigenschaft konkret für ein Maximums- bzw. Minimumsproblem?\newline
        \textbf{Antwort}: Ein Algorithmus ist $\delta$-konkurrenzfähig, wenn er für jede Eingabe höchstens um Faktor $\delta$ schlechter ist als $\text{Opt}$. Bei (1) Minimumsproblem höchstens $\delta$-mal so hohe Kosten wie $\text{Opt}$ und bei (2) Maximumsproblem mindestens $1 / \delta$ vom optimalen Gewinn.

        \item Was ist ein $\delta$-schweres Online Problem?\newline
        \textbf{Antwort}: Ein Online-Problem ist $\delta$-schwer, wenn kein Online-Algorithmus eine bessere Konkurrenzgüte als $\delta$ garantieren kann ($\delta$ ist eine echte Untergrenze).
    \end{enumerate}
    \item Paging:
    \begin{enumerate}[label=(\roman*)]
        \item Was ist das Paging Problem?\newline
        \textbf{Antwort}: Man hat einen Cache mit Platz für $k$ Seiten. Bei einer Anfrage: ist die Seite nicht im Cache, gibt es einen Page Fault und man muss sie laden. Ist der Cache voll, muss man eine Seite auswerfen. Ziel: möglichst wenige Page Faults.

        \item Warum ist Paging für einen Cache-Speicher mit $k$ Bit ein $k$-schweres Problem (Beispiel mit $k=3$)?\newline
        \textbf{Antwort}: Ein Gegner kann Anfragen so wählen, dass wir fast immer genau die Seite braucheny, die wir gerade rausgeworfen haben. Bei $k = 3$: Cache enthält z.B. ${1,2,3}$. Anfrage: $4 \rightarrow$ Wir werfen eine raus (z.B. 1). Nächste Anfrage ist wieder $1 \rightarrow$ Page Fault. Dann wieder eine neue Seite, wir werfen sie raus, Gegner fragt genau diese wieder an. So kann unser Algorithmus ungefähr 3-mal so viele Faults haben wie $\text{Opt}$. Daher ist das Problem $k$-schwer.

    \end{enumerate}
    \clearpage
    \item Randomisierte Online Algorithmen:
    \begin{enumerate}[label=(\roman*)]
        \item Warum sind Online Probleme Beispiele für das Überlisten des Gegners?\newline
        \textbf{Antwort}: Weil man oft mit einem Gegner denkt, der die Eingabe so wählt, dass der Online-Algorithmus schlecht aussieht. Randomisierung hilft, weil der Gegner nicht genau vorhersagen kann, welche Entscheidungen der Algorithmus trifft.

        \item Was ist die Konkurrenzgẗe und die Konkurrenzfähigkeit bei randomisierten Algorithmen?\newline
        \textbf{Antwort}: Man bewertet nicht die festen Kosten, sondern die erwarteten Kosten über die Zufallsentscheidungen: $\text{Erwartete Kosten}(\text{Algorithmus}) \leq c \cdot \text{Kosten}(\text{Opt})$. Dann heisst der Algorithmus $c$-konkurrenzfähig \enquote{im Erwartungswert}.

    \end{enumerate}
    \item Beispiel:
    \begin{enumerate}[label=(\roman*)]
        \item Kann ich das Beispiel zur Arbeitsverteilung erklären?\newline
        \textbf{Antwort}: Zwei Aufträge müssen jeweils durch $m$ Maschinen-Schritte. Wenn beide gleichzeitig dieselbe Maschine brauchen, muss einer warten. Ziel ist minimale Gesamtzeit bzw. minimale Verzögerung.

        \item Verstehe ich die graphische Darstellung mit zwei Aufträgen?\newline
        \textbf{Antwort}: Man zeichnet ein $m \times m$-Gitter. Ein Diagonalschritt bedeutet \enquote{beide Jobs machen Fortschritt gleichzeitig}. Wenn das wegen Konflikt nicht geht, muss man horizontal/vertikale ausweichen (einer wartet). Hindernisse markieren Konflikte.

        \item Kenne ich Algorithmus DIAG?\newline
        \textbf{Antwort}: DIAG wählt zufällig eine Diagonale (eine \enquote{Spur} nahe der Hauptdiagonalen) und versucht, möglichst entlang dieser Spur zu laufen. Bei einem Hindernis weicht er kurz aus und kehrt zur Spur zurück. Die Zufallswahl macht es schwer, ihn systematisch in viele Konflikte zu erzwingen, daher ist die erwartete Verzögerung klein.

    \end{enumerate}
\end{enumerate}

\chapter{Fingerabdrücke}
Oft sind Objekte sehr gross (z.B. lange Bitstrings, ganze Matrizen). Ein direkter Vergleich wäre teuer: man müsste im schlimmsten Fall alles vollständig übertragen / ausrechnen / vergleichen.

Trick: Statt das ganze Objekt zu vergleichen, berechnet man einen kleinen \enquote{Fingerabdruck} (eine kurze Zahl oder einen kurzen Vektor). Danach vergleicht man nur diese Fingerabdrücke.

\begin{itemize}
    \item Wenn zwei Objekte gleich sind, sollen ihre Fingerabdrücke immer gleich sein.
    \item Wenn zwei Objekte verschieden sind, sollen ihre Fingerabdrücke meistens verschieden sein (mit hoher Wahrscheinlichkeit).
\end{itemize}

Das heisst: Man akzeptiert eine kleine Fehlerwahrscheinlichkeit (typisch Monte-Carlo) oder man macht einen zusätzlichen sicheren Check, sobald Fingerabdrücke gleich aussehen (dann ist es Las-Vegas und immer korrekt).    

\section{Fingerabdruck}

Wir betrachten Bitstrings $x \in \{0, 1\}^n$. Man interpretiert den String als Zahl:
$$
\text{Nummer}(x) = \sum^{n}_{i=1} 2^{n-i} \cdot x_i
$$
Also wie eine Binärzahl.

Dann wählt man zufällig eine Primzahl $p$ aus einer grossen Menge (z.B. alle Primzahlen $\leq n^2$) und definiert den Fingerabdruck:
$$
\text{Finger}_p(x) = \text{Nummer}(x) \mod p
$$

\clearpage
\textbf{Warum funktioniert das?}

\begin{itemize}
    \item Falls $x = y$, dann ist \text{Nummer}(x) = \text{Nummer}(y), also auch für jedes $p$:
    $$
    \text{Finger}_p(x) = \text{Finger}_p(y)
    $$
    \item Falls $x \neq y$, dann ist $d = \text{Nummer}(x) - \text{Nummer}(y) \neq 0$. Ein Fehler passiert nur, wenn $p$ den Unterschied \enquote{versteckt}, also wenn $p \mid d$. Es gibt aber nur wenige Primzahlen, die eine feste Zahl $d$ teilen können (maximal so viele wie $d$ Primfaktoren hat; in dieser Anwendung ergibt sich eine sehr kleine obere Schranke auf \enquote{schlechte Primzahlen}). Daraus folgt: mit hoher Wahrscheinlichkeit ist $\text{Finger}_p(x) \neq \text{Finger}_p(y)$.
\end{itemize}

\section{Kommunikationsprotokolle: PSet, PSchnitt, d-R}

\subsection{PSet - \enquote{Ist $x$ in der Menge $U$?}}

Situation:

\begin{itemize}
    \item Rechner $R_I$ hat einen String $x$
    \item Rechner $R_{n}$ hat eine Menge $U = \{u_1, \dots, u_k\}$ (alle Länge $n$). 
    \item Ziel: Entscheiden, ob $x \in U$, ohne dass $R_I$ den ganzen String $x$ schicken muss.
\end{itemize}

Protokoll PSet (vereinfacht):

\begin{enumerate}
    \item $R_I$ wählt zufällig eine Primzahl $p \in \text{PRIM}(n^2)$.
    \item $R_I$ berechnet $s = \text{Nummer}(x) \mod p$ und sendet $p$ und $s$.
    \item $R_n$ berechnet für jedes $u_i$ den Rest $q_i = \text{Nummer}(u_i) \mod p$.
    \begin{itemize}
        \item Falls $s$ in $\{ q_1, \dots, q_k \}$ vorkommt: Ausgabe \enquote{$x \in U$}
        \item sonst: Ausgabe $x \notin U$
    \end{itemize}
\end{enumerate}

\clearpage
Fehlerart und Eigenschaft:

\begin{itemize}
    \item Wenn wirklich $x \in U$: dann ist $x = u_j$ für ein $j$ und der Rest passt für jejdes $p$. Fehlerwahrscheinlichkeit ist 0.
    \item Wenn $x \notin U$: Fehler nur möglich, wenn für irgendein $u_i$ zufällig:
    $$
    \text{Nummer}(x) \equiv \text{Nummer}(u_i) (\bmod p)
    $$
    obwohl die Zahlen verschieden sind. Das kann passieren, aber selten.
\end{itemize}

Eine typische Abschätzung der Fehlerwahrscheinlichkeit ist:
$$
Pr[\text{Fehler}] \leq k \cdot \frac{2 \ln n}{n}
$$
Solange $k$ nicht zu gross ist, ist das klein (z.B. für $k \leq \frac{n}{4 \ln n}$ ist der Fehler $\leq \frac{1}{2}$).

Einordnung: einseitiger Monte-Carlo (1MC): nie fälschlich \enquote{nicht drin}, wenn $x \in U$; aber manchmal fälschlich \enquote{drin}, wenn $x \notin U$.

\subsection{PSchnitt - \enquote{Ist $U \cap V$ leer?}}

Situation:

\begin{itemize}
    \item $R_I$ hat $V = \{ v_1, \dots, v_l \}$
    \item $R_n$ hat $U = \{ u_1, \dots, u_k \}$
    \item Ziel: Entscheiden, ob $U \cap V = \empty$ oder nicht.
\end{itemize}

Protokoll PSchnitt (vereinfacht):

\begin{enumerate}
    \item $R_I$ wählt zufällig Primzahl $p \in \text{PRIM}(n^2)$.
    \item $R_I$ sendet $p$ und alle Reste $s_i = \text{Nummer}(v_i) \mod p$.
    \item $R_n$ berechnet $q_j = \text{Nummer}(u_j) \mod p$ und prüft, ob sich Restmengen schneiden.
\end{enumerate}

\clearpage
Fehlerart und typische Schranke

\begin{itemize}
    \item Wenn $U \cap V \neq \empty$: Es gibt wirklich gleiche Strings $\rightarrow$ Reste passen für jedes $p$. Fehler ist 0.
    \item Wenn $U \cap V = \empty$: Fehler nur durch \enquote{Rest-Kollisionen}.
\end{itemize}

$$
Pr[\text{Fehler}] \leq l \cdot k \cdot \frac{2 \ln n}{n}
$$

Damit geht die Fehlerwahrscheinlichkeit gegen 0, wenn $l \cdot k = o(\frac{n}{\ln n})$.

Einordnung: Dies ist ein 1MC* in dem Sinn, dass bei wachsenden $n$ die Fehlerwahrscheinlichkeit gegen 0 geht (unter den Grössenbedingungen).

\subsection{d-R - \enquote{Gleiche Strings} mit stärkerer Zuverlässigkeit}

Hier ist die Idee: Statt Primzahlen $\leq n^2$ nimmt man Primzahlen $\leq n^d$ mit $d > 2$.

\textbf{Warum hilft das?}

Die Anzahl \enquote{schlechter Primzahlen} bleibt im wesentlichen ähnlich (für feste Eingabe), aber die Gesamtmenge möglicher Primzahlen wird viel grösser. Das heisst der Anteil schlechter Primzahlen wird sehr klein:
$$
Pr[\text{Fehler}] \leq \frac{n-1}{|\text{PRIM}(n^d)|} \approx \frac{d \ln n}{n^{d-1}}
$$
Trade-off (wichtig):
\begin{itemize}
    \item Kommunikation steigt linear mit $d$ (weil $p$ mehr Bits braucht).
    \item Fehler sinkt sehr schnell mit wachsendem $d$.
\end{itemize}

\clearpage
\section{STRING - Teilstringproblem (Las-Vegas mit Fingerabdrücken)}

\textbf{Problem}: Gegeben Muster $x$ (Länge $n$) und Text $Y$ (Länge $m$), finde die kleinste Position $r$, so dass $x$ gleich dem Teilstring $Y[r..r + n - 1]$ ist, oder \enquote{nicht vorhanden}.

\textbf{Naiv}: Vergleiche $x$ mit jedem Teilstring $O(n \cdot m)$.

\textbf{Idee}: Vergleiche zuerst nur Fingerabdrücke $\bmod p$. Nur wenn Fingerabdrücke gleich sind, mache einen echten Zeichen-für-Zeichen-Vergleich.

Algorithmus STRING (vereinfacht):

\begin{enumerate}
    \item Wähle zufällig eine Primzahl $p$ aus einer geeigneten Menge
    \item Berechne $\text{Finger}_p(x)$.
    \item Laufe über alle Fenster $Y[r..r + n - 1]$:
    \begin{itemize}
        \item Berechne $\text{Finger}_p(Y[r..r + n - 1])$
        \item Falls Fingerabdrücke verschieden: sicher kein Match, also weiter.
        \item Falls gleich: verifiziere echt durch direkten Vergleich; wenn gleich, gib $r$ aus.
    \end{itemize}
\end{enumerate}

Wichtiger Trick: \enquote{Rolling Hash} (Fingerabdruck schnell updaten)

Man kann den Fingerabdruck des nächsten Fensters aus dem vorherigen in $O(1)$ updaten (statt jedes Fenster neu in $O(n)$ zu berechnen). Im Buch steht dafür eine Update-Formel (in Bits mit Basis 2).

Eigenschaften
\begin{itemize}
    \item Las-Vegas: Ergebnis ist immer korrekt, weil bei Fingerabdruck-Treffern zusätzlich wirklich verglichen wird.
    \item Erwartete Laufzeit kann (mit passender Wahl von $p$) linear in $m$ sein, weil falsche Fingerabdruck-Treffer selten sind.
\end{itemize}

\clearpage
\section{FREIVALDS - Verifikation von Matrixmultiplikationen}

\textbf{Problem}: Drei $n \times n$ Matrizen $A, B, C$. Prüfen, ob wirklich $A \cdot B = C$.

Deterministisch \enquote{einfach}: $A \cdot B$ ausrechnen und vergleichen. Das ist teuer (klassisch $O(n^3)$ arithmetische Operationen).

Freivalds-Idee: Matrizen wirken wie Funktionen auf Vektoren. Wenn $A \cdot B = C$, dann gilt für jeden Vektor $a$:
$$
A(Ba) = Ca
$$
Also testen wir das nur für einen zufälligen Vektor $a$.

Algorithmus FREIVALDS:

\begin{enumerate}
    \item Wähle zufällig $a \in \{0,1\}^n$
    \item Berechne $\beta = A(Ba)$ und $\gamma = Ca$.
    \item Wenn $\beta = \gamma$: gib \enquote{gleich} aus, sonst \enquote{ungleich}.
\end{enumerate}

Fehlerwahrscheinlichkeit (einseitig)

\begin{itemize}
    \item Falls $A \cdot B = C$: Gleichheit gilt für alle $a \rightarrow$ Fehler 0.
    \item Falls $A \cdot B \neq C$: Dann unterscheiden sich die Matrizen in mindestens einer Zeile/Eintrag. Es lässt sich zeigen: mindestens die Hälfte aller $a \in \{0,1\}^n$ \enquote{entlarvt} den Unterschied, d.h. liefert $\beta \neq \gamma$. Also:
    $$
    Pr[\text{Fehler}] \leq \frac{1}{2}
    $$
    und durch $t$ unabhängige Wiederholungen wird das $\leq 2^{-t}$.
\end{itemize}

Eine Matrix-Vektor-Multiplikation kostet $O(n^2)$. Es sind konstant viele davon, also insgesamt $O(n^2)$ arithmetische Operationen.

Hinweis zur Konsistenz: An einer STelle steht \enquote{$O(n^3)$}, aber gleichzeitig ist das erklärte Ziel ein $O(n^2)$-Verfahren. Mit der üblichen Kostenrechnung (Matrix-Vektor) ist $O(n^2)$ korrekt.

\clearpage
\section{Typische Stolperfallen}

\begin{itemize}
    \item \enquote{Gleicher Fingerabdruck $\Rightarrow$ gleiches Objekt} ist falsch (Kollision möglich). Bei Monte-Carlo akzeptiert man dann eine kleine Fehlerrate, bei Las-Vegas überprüft man danach sicher.
    \item Fehlerwahrscheinlichkeit hängt von der Grösse der Kandidatenmenge ab: Bei PSet/PSchnitt wirds schlechter, wenn $k$ bzw. $kl$ gross wird.
    \item Fehler senken geht auf zwei Arten:
    \begin{enumerate}
        \item Wiederholen (unabhängig): Fehler fällt exponentiell.
        \item Grössere Primzahlen zulassen (d-R): Fingerabdruck länger, aber Fehler sinkt stark.
    \end{enumerate}
\end{itemize}

\clearpage
\begin{enumerate}[label=(\alph*)]
    \item Fingerabdrücke:
    \begin{enumerate}[label=(\roman*)]
        \item Kann ich die Idee erklären?\newline
        \textbf{Antwort}: Ein Fingerabdruck ist eine kurze Repräsentation eines grossen Objekts (z. B. eines langen Bitstrings). Statt zwei Objekte komplett zu vergleichen, vergleicht man ihre Fingerabdrücke. Gleiche Objekte haben immer gleiche Fingerabdrücke; verschiedene Objekte haben mit hoher Wahrscheinlichkeit verschiedene Fingerabdrücke (kleine Kollisions-/Fehlerwahrscheinlichkeit).
    \end{enumerate}
    \item Beispiel:
    \begin{enumerate}[label=(\roman*)]
        \item Kenne ich die Algorithmen PSet, PSchnitt und d-R für die Kommunikationsprotokolle und ihre Eigenschaften?\newline
        \textbf{Antwort}: PSet: Prüft, ob ein String $x$ in einer Menge $U$ liegt, indem man zufällig ein Primzahl-Modul $p$ wählt und nur den Rest (Fingerabdruck) vergleicht. Einseitiger Fehler: Wenn $x \in U$, ist das ERgebnis sicher richtig. Falsch-Positive sind möglich, aber selten.

        PSchnitt: Prüft, ob zwei Mengen $U$ und $V$ einen gemeinsamen String haben, über Fingerabdrücke $\mod p$. Einseitiger Fehler: Wenn es wirklich einen gemeinsamen String gibt, wird er sicher erkannt. Ansosnsten sind falsch-positive durch Kollisionen möglich.

        d-R: Gleiche Grundidee, aber man wählt $p$ aus einer grösseren Menge (Primzahlen bis $n^d$). Effekt: viel kleinere Fehlerwahrscheinlichkeit, aber etwas mehr Kommunikationsaufwand (weil $p$ mehr Bits hat).

        \item Kenne ich den Algorithmus STRING für das Teilstring Problem und seine Eigenschaften?\newline
        \textbf{Antwort}: Findet, ob ein Muster $x$ in einem Text $Y$ vorkommt, indem man Fingerabdrücke von \enquote{Fenstern} im Text vergleicht (schnell per Rolling Update). Bei Fingerabdruck-Treffern wird immer noch echt nachgeprüft. Eigenschaft: Las-Vegas (immer korrekt), typischerweise sehr schnell, weil selten nachgeprüft werden muss.

        \item Kenne ich den Algorithmus FREIVALDS für die Verifikation der Matrizenmultiplikation und seine Eigenschaften?\newline
        \textbf{Antwort}: Testet $A \cdot B = C$, indem man einen zufälligen 0/1-Vektor $a$ wählt und prüft, ob $A(Ba) = Ca$. Eigenschaften:
        \begin{itemize}
            \item Wenn $A \cdot B = C$. immer \enquote{richtig} (kein Fehler).
            \item Wenn $A \cdot B \neq C$: Fehlerchance pro Test höchstens $1/2$; durch Wiederholen fällt sie auf $2^{-t}$.
            \item Laufzeit pro Test: $O(n^2)$ (nur Matrix-Vektor-Produkte)
        \end{itemize}

    \end{enumerate}
\end{enumerate}

\chapter{Wahrscheinlichkeitsverstärkung}
\clearpage
\begin{enumerate}[label=(\alph*)]
    \item Beispiel:
    \begin{enumerate}[label=(\roman*)]
        \item Kenne ich den Algorithmus AQP für die Äquivalenz zweier Polynome und seine Eigenschaften?
    \end{enumerate}
    \item Wahrscheinlichkeitsverstärkung und Stichproben:
    \begin{enumerate}[label=(\roman*)]
        \item Kann ich die Idee erklären?
    \end{enumerate}
    \item Beispiel:
    \begin{enumerate}[label=(\roman*)]
        \item Kann ich das MIN-CUT Problem erklären?
        \item Kenne ich den Algorithmus KONTRAKTION und seine Eigenschaften?
        \item Warum ist der Algorithmus KONTRAKTION auch durch simples Wiederholen nicht brauchbar?
    \end{enumerate}
\end{enumerate}

\chapter{Wiederholungen, 3SAT, Zeugen und Primzahl-Test}
\clearpage
\begin{enumerate}[label=(\alph*)]
    \item Gezielte Wiederholungen:
    \begin{enumerate}[label=(\roman*)]
        \item Mit welcher Grundidee versuchen den Algorithmus KONTRAKTION zu verbessern?
        \item Kenne ich die Algorithmen DETRAN und WBAUM und ihre Eigenschaften?
    \end{enumerate}
    \item Das 3SAT Problem:
    \begin{enumerate}[label=(\roman*)]
        \item Warum sind auch randomisierte Algorithmen mit exponentieller Laufzeit interessant?
        \item Kenne ich den Algorithmus SCHÖNING für das 3SAT Problem und seine Eigenschaften?
    \end{enumerate}
    \item Häufige Zeugen:
    \begin{enumerate}[label=(\roman*)]
        \item Kann ich die Idee erklären?
    \end{enumerate}
    \item Primzahl Test:
    \begin{enumerate}[label=(\roman*)]
        \item Was ist ein Primzahl Test?
        \item Kenn ich den Algorithmus NAIV undv erstehe ich, wann er unbrauchbar ist?
    \end{enumerate}
\end{enumerate}

\chapter{Zeugenkandidaten, Zufälliges Runden}


\clearpage
\section{Summary}
\begin{enumerate}[label=(\alph*)]
    \item Zeugenkandidaten:
    \begin{enumerate}[label=(\roman*)]
        \item Was sind Zeugenkandidaten und welche Eigenschaften sollen sie haben?
        \item Wie wird der Satz von Fermat für Zeugenkandidaten gebraucht?
        \item Wie werden die Zeugenkandidaten mit dem Satz von Euler bzw. dem Satz von Miller-Rabin optimiert?
        \item Kenne ich den Algorithmus PRIMZAHL und seine Eigenschaften?
    \end{enumerate}
    \newpage
    \item Zufälliges Runden:
    \begin{enumerate}[label=(\roman*)]
        \item Kann ich die Idee erklären?
        \item Wie können wir das MIN-VCP Problem in ein LP Problem übersetzen?
        \item Kann ich das MAX-KP Problem erklären?
        \item Wie können wir das MAX-KP Problem in ein LP Problem übersetzen?
    \end{enumerate}
\end{enumerate}

\chapter{MAX-SAT, LP}
\section{Was ist MAX-SAT}

Wir haben boolesche Variablen $x_1, \dots, x_n$ jede ist wahr (1) oder falsch (0). Eine Klausel ist ein Oder-Ausdruck aus Literalen z.B.
$$
x_1 \lor \neg x_2 \lor x_3
$$
Ein Literal ist entweder $x_i$ oder $\neg x_i$. Eine Formel in CNF/KNF ist ein Und aus Klauseln:
$$
\Phi = C_1 \land C_2 \land \dots \land C_m
$$

Ziel (MAX-SAT): Finde eine Belegung der Variablen, die möglichst viele Klauseln erfüllt. Wichtig: MAX-SAT ist NP-schwer $\rightarrow$ wir suchen effiziente Approximation.

\section{Schritt 1: Übersetzung von MAX-SAT in ein ILP / 0-1-LP}

Die Idee: Wir bauen ein lineares Optimierungsproblem, das \enquote{zählt}, wie viele Klauseln erfüllt sind.

\textbf{Variablen}

\begin{itemize}
    \item Für jede boolesche Variable $x_i$: \newline
    $x_i \in \{0, 1\}$ (0 = falsch, 1 = wahr)
    \item Für jede Klausel $C_j$:
    $y_j \in \{0, 1\}$ (1 = Klausel erfüllt, 0 = nicht erfüllt)
\end{itemize}

\textbf{Wie modelliert man Literale als Zahlen?}

\begin{itemize}
    \item Positives Literal $x_i$ entspricht einfach $x_i$.
    \item Negiertes Literal $\neg x_i$ entspricht $1 - x_i$.\newline
    (Denn wenn $x_i = 0$, dann ist $\neg x_i$ wahr $\rightarrow$ 1 - 0 = 1)
\end{itemize}

\clearpage
\textbf{Nebenbedingungen pro Klausel}

Für jede Klausel $C_j$:
$$
(\text{Summe der Literal-Werte in } C_j) \geq y_j
$$

Warum klappt das?

\begin{itemize}
    \item Wenn alle Literale 0 sind, ist die Summe 0 $\rightarrow$ dann muss $y_j = 0$ sein.
    \item Wenn mindestens ein Literal 1 ist, ist die Summe $\geq 1 \rightarrow$ dann darf $y_j = 1$ werden.\newline
    Da wir gleich $y_j$ maximieren, wird es bei erfüllbaren Klauseln \enquote{hochgezogen}.
\end{itemize}

\textbf{Zielfunktion}
$$
\max \sum^{m}_{j = 1} y_j
$$

Das maximiert die Anzahl erfüllter Klauseln.

\section{Schritt 2: Relaxation}

Das 0-1-LP ist (wie MAX-SAT) schwer, weil $x_i, y_j \in \{0, 1\}$ gefordert ist.

Relaxation: Ersetze $\{0, 1\}$ durch das Intervall $[0, 1]$:
$$
0 \leq x_i \leq 1, \quad 0 \leq y_j \leq 1
$$
Jetzt dürfen z.B. $x_i = 0.7$ sein. Das ist nicht \enquote{wahr/falsch}, aber mathematisch super, weil:
\begin{itemize}
    \item LP (lineare Programmierung) kann effizient (polynomiell) gelöst werden.
    \item Die optimale LP-Lösung ist eine Art \enquote{beste fraktionale Schätzung}
\end{itemize}

\clearpage
\section{Algorithmus RZR}

Ziel: Aus der fraktionalen LP-Lösung wieder eine echte Belegung $x_i \in \{0, 1\}$ machen.

\textbf{RZR Schritt-für-Schritt}

\begin{enumerate}
    \item Baue das LP (wie oben) und löse es.\newline
    Ergebnis: optimale Werte $x_i^{*} \in [0, 1]$ und $y_j^{*} \in [0, 1]$.
    \item Runde zufällig: für jede Variable $x_i$
    \begin{itemize}
        \item setze $x_i = 1$ mit Wahrscheinlichkeit $x_i^{*}$
        \item setze $x_i = 0$ mit Wahrscheinlichkeit $1 - x_i^{*}$
    \end{itemize}
\end{enumerate}

Interpretation: $x_i^{*}$ ist wie eine \enquote{gezinkte Münze} für $x_i$.

\section{Analyse von RZR}

Betrachten wir eine Klausel mit $k$ Literalen (z.B. $x_1 \lor x_2 \lor \dots \lor x_k$). Nach dem Runden sind die Variablen unabhängig gesetzt.

Die Wahrscheinlichkeit, dass die Klausel NICHT erfüllt ist: Sie ist genau dann nicht erfüllt, wenn alle $k$ Literale 0 werden:
$$
Pr[\text{Klausel nicht erfüllt}] = \prod_{i=1}^{k} (1 - x_i^{*})
$$
Also:
$$
Pr[\text{Klausel erfüllt}] = 1 - \prod_{i=1}^{k}(1- x_i^{*})
$$

Jetzt kommt der entscheidende \enquote{Worst-Case}-Trick:

\begin{itemize}
    \item Aus dem LP weiss man grob: Summe der Literal-Werte ist mindestens $y_j^{*}$ (genau das steht in der Klausel-Nebenbedingung).
    \item Die Klausel ist am \enquote{schwierigsten} zu erfüllen, wenn die Wahrscheinlichkeiten gleichmässig verteilt sind (Intuition: viele kleine Chancen statt einer grossen).
    \item Dann gilt (Worst-Case):
\end{itemize}
$$
Pr[\text{Klausel erfüllt}] \geq 1 - \left( 1 - \frac{y_j^{*}}{k}\right)^k
$$
Wenn man (zur Intuition) $y_j^{*} = 1$ setzt, wird daraus:
$$
Pr[\text{erfüllt}] \geq 1 - \left( 1 - \frac{1}{k} \right)^k
$$
Und für grosse $k$ gilt der bekannte Grenzwert:
$$
\left( 1 - \frac{1}{k} \right)^k \rightarrow \frac{1}{e} \quad \Rightarrow \quad Pr[\text{erfüllt}] \gtrsim 1 - \frac{1}{e} \approx 0.632
$$

\begin{multicols}{2}
    Das heisst: Im Erwartungswert erfüllt RZR mindestens ca. 63.2\% der optimal möglichen Klauselzahl.

    \begin{figure}[H]
        \img[width=0.45\textwidth]{figures/rzr-prob.png}
        {RZR}
        \label{fig:rzr}
    \end{figure}
\end{multicols}

\textbf{Eigenschaft}

\begin{itemize}
    \item Laufzeit: polynomiell (LP lösen dominiert, Runden ist linear).
    \item Erwartete Approximationsgarantie: mindestens $1 - \frac{1}{e}$ vom Optimum (also Approximationsfaktor $\leq \frac{e}{e - 1} \approx 1.582$).
\end{itemize}

\subsection{Schwachstelle von RZR (warum reicht RZR allein nicht immer?)}

Bei sehr kurzen Klauseln (insbesondere Länge 1) kann RZR \enquote{zu vorsichtig} sein:

Beispiel: Klausel $C = (x_1)$. Wenn das LP $x_1^{*} = 0.1$ liefert, dann setzt RZR $x_1 = 1$ nur mit 10\% $\rightarrow$ diese Klausel geht oft verloren.

Hier ist eine simple 50/50-Zufallsbelegung manchmal besser, weil sie kurze Klauseln viel öfter trifft.

\clearpage
\section{Algorithmus KOMB - Kombination aus \enquote{Simple Random} und RZR}

\textbf{Simple Random (wie STICH / naive Zufallsbelegung)}

\begin{itemize}
    \item Setze jede Variable unabhängig mit Wahrscheinlichkeit $1/2$ auf $1$.
\end{itemize}

Dann gilt für eine Klausel mit $k$ Literalen:
\begin{itemize}
    \item Sie ist nur dann falsch, wenn alle $k$ Literale falsch sind $\rightarrow$ Wahrscheinlichkeit $2^{-k}$.
    \item Also:\newline
    $$
    Pr[\text{Klausel erfüllt}] = 1 - 2^{-k}
    $$
\end{itemize}

KOMB-Idee: Nutze das Beste aus beiden Welten. Es gibt zwei gleichwertige Sichtweisen (beide sind korrekt):

\begin{enumerate}
    \item Führe beide aus (Simple Random und RZR) und nimm die bessere Belegung (mehr erfüllte Klauseln).
    \item Wähle zufällig (z.B. Münzwurf), ob Simple Random oder RZR benutzt wird. Dann ist die erwartete Leistung der Durchschnitt beider Garantien.
\end{enumerate}

\subsection{Warum garantiert KOMB mindestens $3/4$}

Für Klausellänge $k$ haben wir zwei (untere) Schranken:

\begin{itemize}
    \item Simple Random: $1 - 2^{-k}$
    \item RZR: $1 - (1 - \frac{1}{k})^k$ (bzw. die passende Version mit $y_j^{*}$)
\end{itemize}

Wenn man im Mittel beide Strategien kombiniert (oder beide laufen lässt und das bessere nimmt), bekommt man mindestens:
$$
\frac{(1-2^{k}) + (1 - (1 - \frac{1}{k})^k)}{2}
$$

Und diese Grässe ist für jedes $k \geq 1$ mindestens $\frac{3}{4}$.

\clearpage
Intuition:

\begin{itemize}
    \item Bei $k = 1$: Simple Random ist $0.5$, RZR kann schlecht sein $\rightarrow$ Mittelwert $\geq 0.75$ (weil RZR bei $k = 1$ im LP-Setup effektiv \enquote{stark} wirken kann, und in der Kombi zählt die sichere Untergrenze über alle $k$).
    \item Bei $k = 2$: Simple Random ist $0.75 \rightarrow$ schon top.
    \item Bei grossen $k$: RZR nähert sich $0.632$, Simple Random nähert sich $1 \rightarrow$ Mittelwert bleibt $\geq 0.755$.
\end{itemize}

Ergebnis: KOMB erfüllt im Erwartungswert mindestens 75\% der optimal erfüllbaren Klauseln. Das entspricht Approximationsfaktor $\leq 4/3 \approx 1.333$.

\clearpage
\section{Summary}
\begin{enumerate}[label=(\alph*)]
    \item Beispiel:
    \begin{enumerate}[label=(\roman*)]
        \item Kann ich das MAX-SAT Problem in ein LP Problem übersetzen?\newline
        \textbf{Antwort}: Definiere $x_i \in \{0, 1\}$ (Variablen) und $y_j \in \{0, 1\}$ (Klausel erfüllt). Für jede Klausel $C_j$: $\sum$ (Literale in $C_j$) $\geq y_j$, wobei $\neg x_i$ als $1 - x_i$ modelliert wird. Ziel: $\max \sum_j y_j$. Dann relaxieren zu $x_i, y_j \in [0, 1] \rightarrow$ LP.
        \item Kenne ich die zwei Algorithmen RZR und KOMB und ihre Eigenschaften?\newline
        \textbf{Antwort}:
        \begin{itemize}
            \item RZR: Löse das LP, setze jedes $x_i = 1$ mit Wahrscheinlichkeit $x_i^{*}$. Erwartete Garantie für MAX-SAT: mindestens $1 - \frac{1}{e} \approx 0.632$ vom Optimum.
            \item KOMB: Kombiniert RZR mit Simple Random (jedes $x_i$ mit $1/2$). Erwartete Garantie: mindestens $3/4 = 0.75$ vom Optimum.
        \end{itemize}
    \end{enumerate}
\end{enumerate}

\chapter{Verzeichnisse}
\raggedright
\section{Literaturverzeichnis}

\printbibliography[heading=none]

\clearpage
\section{Abbildungsverzeichnis}
\onlylof

\end{document}
